\section{Расширения}\label{HS.Ext} 

\subsection{Назначение и перечень}\label{HS.Ext.List} 

С помощью расширений клиент и сервер согласовывают параметры, обмениваются 
ключевым материалом.
%
Расширение представляет собой структуру типа \code{Extension}, который 
определяется следующим образом:
%
\begin{codeblock}
struct {
  ExtensionType extension_type;
  opaque extension_data<0..2^16-1>;
} Extension;
\end{codeblock}

Поля \code{Extension} имеют следующее значение:
\begin{itemize}
\item
\code{extension_type}~--- тип расширения;
\item
\code{extension_data}~--- данные расширения, определяемые в зависимости от типа.
\end{itemize}

Вложенный тип \code{ExtensionType} определяется следующим образом:
\begin{codeblock}
enum {
  server_name(0),
  max_fragment_length(1),
  status_request(5),
  supported_groups(10),
  signature_algorithms(13),
  use_srtp(14),
  heartbeat(15),
  application_layer_protocol_negotiation(16),
  signed_certificate_timestamp(18),
  client_certificate_type(19),
  server_certificate_type(20),
  padding(21),
  pre_shared_key(41),
  early_data(42),
  supported_versions(43),
  cookie(44),
  psk_key_exchange_modes(45),
  certificate_authorities(47),
  oid_filters(48),
  post_handshake_auth(49),
  signature_algorithms_cert(50),
  key_share(51),
  (65535)
} ExtensionType;
\end{codeblock}

Большинство расширений имеют две формы: запрос и ответ.
%
Имеются расширения информационного характера, которые не предполагают ответа.
%
Запрос клиента отправляется в сообщении \token[HS.CH]{ClientHello}, 
соответствующий ответ сервера~--- в \token[HS.SH]{ServerHello}, 
\token[HS.EE]{EncryptedExtensions}, \token[HS.HRR]{HelloRetryRequest} или 
\token[HS.CT]{Certificate}.
%
В обратную сторону, запрос сервера отправляется в 
\token[HS.CR]{CertificateRequest}, клиент может дать на него ответ в 
\token[HS.CT]{Certificate}.
%
Сервер может отправить расширение, не предполагающее ответа, 
в~\token[HS.NST]{NewSessionTicket}.

Сторона TLS не должна отправлять расширение-ответ, если не получено
расширение-запрос. При получении ответа без запроса протокол Handshake должен 
быть прерван с оповещением \token[ALERT.Err.um]{unexpected_message}.
Исключением из данного правила является отправка расширения 
\token[HS.Ext.c]{cookie} в сообщении \token[HS.HRR]{HelloRetryRequest} 
(см.~\ref{HS.Ext.c}).
%
В таблице~\ref{Table.HS.Exts} перечислены расширения TLS версии 1.3.
Часть расширений вводится в настоящем стандарте, остальные~--- 
в спецификациях, ссылки на которые даны в квадратных скобках.
В таблице~\ref{Table.HS.Exts} указаны сообщения Handshake, в которых 
расширения пересылаются. 
%
Использованы сокращения: 
CH~--- \token[HS.CH]{ClientHello},
SH~--- \token[HS.SH]{ServerHello},
EE~--- \token[HS.EE]{EncryptedExtensions}, 
CT~--- \token[HS.CT]{Certificate},
CR~--- \token[HS.CR]{CertificateRequest}, 
NST~--- \token[HS.NST]{NewSessionTicket} и
HRR~--- \token[HS.HRR]{HelloRetryRequest}. 
%
Расширения перечисляются в порядке возрастания их типов
в~\code{ExtensionType}.

Следующие расширения не вводятся в настоящем стандарте, но учитываются при 
определении механизмов TLS.

\begin{enumerate}
\item\label{HS.Ext.sn}
В расширении \token{server_name} клиент передает имя сервера, с которым он 
собирается взаимодействовать. Имя учитывается при выборе сертификата для 
аутентификации. Расширение оказывается полезным в ситуации, когда один 
TLS-сервер обслуживает несколько подчиненных прикладных серверов (возможно  
виртуальных).

% info: https://security.stackexchange.com/questions/269301

\item\label{HS.Ext.mfl}
Расширение \token{max_fragment_length} используется для согласования
максимальной длины фрагментов данных при их передаче по протоколу Record.

\item\label{HS.Ext.alpn}
Расширение \token{application_layer_protocol_negotiation} используется  
для согласования прикладного протокола, который будет выполняться поверх TLS.

\item\label{HS.Ext.ct}
Расширения \token{client_certificate_type} и \token{server_certificate_type} 
используются для согласования типа сертификата для аутентификации соответственно 
клиента и сервера. Поддерживаются два типа сертификатов: обычные сертификаты 
X.509, определенные в СТБ 34.101.19, и контейнеры типа 
\code{SubjectPublicKeyInfo} АСН.1, также определенного в СТБ 34.101.19. 
%
Контейнер содержит открытый ключ, но не содержит ни идентификационных данных, 
ни срока действия, ни подписи УЦ. Обмен открытыми ключами в облегченных 
контейнерах востребован в некоторых сценариях использования TLS 
(см.~\cite{RFC7250}).

\item\label{HS.Ext.p}
Расширение \token{padding} используется для искусственного увеличения размера 
сообщения \token[HS.CH]{ClientHello} через включение в расширение незначащих
нулевых байтов.
%
Известны реализации TLS, которые некорректно обрабатывают
\token[HS.CH]{ClientHello} определенных длин. Увеличение размера сообщений
помогает избежать ошибок при выполнении TLS с такими реализациями.
\end{enumerate}

\begin{table}[hbt]
\caption{Расширения}\label{Table.HS.Exts}
\begin{tabular}{|l|c|l|}
\hline
Расширение & Определение & Сообщения Handshake\\
\hline
\hline
\code|server_name| & \cite{RFC6066} & 
CH, EE\\ 
%
\code|max_fragment_length| & \cite{RFC6066} & 
CH, EE\\ 
%
\code|status_request| & \ref{HS.Ext.sr} & 
CH, CR, CT\\ 
% todo: \cite{RFC6066}?
%
\code|supported_groups| & \ref{HS.Ext.sg} & 
CH, EE\\ 
% todo: \cite{RFC7919}?
%
\code|signature_algorithms| & \ref{HS.Ext.sa} &
CH, CR\\ 
%
\code|use_srtp| & \cite{RFC5764} & 
CH, EE\\ 
%
\code|heartbeat| & \cite{RFC6520} & 
CH, EE\\ 
%
\code|application_layer_protocol_negotiation| & \cite{RFC7301} & 
CH, EE\\ 
%
\code|signed_certificate_timestamp| & \cite{RFC6962} & 
CH, CR, CT\\ 
%
\code|client_certificate_type| & \cite{RFC7250} & 
CH, EE\\ 
%
\code|server_certificate_type| & \cite{RFC7250} & 
CH, EE\\ 
%
\code|padding| & \cite{RFC7685} & 
CH\\ 
%
\code|pre_shared_key| & \ref{HS.Ext.psk} &
CH, SH\\ 
%
\code|early_data| & \ref{HS.Ext.ed} &
CH, EE, NST\\ 
%
\code|supported_versions| & \ref{HS.Ext.sv} &
CH, SH, HRR\\ 
%
\code|cookie| & \ref{HS.Ext.c} &
CH, HRR\\ 
%
\code|psk_key_exchange_modes| & \ref{HS.Ext.pkem} &
CH\\ 
%
\code|certificate_authorities| & \ref{HS.Ext.ca} &
CH, CR\\
%
\code|oid_filters| & \ref{HS.Ext.of} &
CR\\ 
%
\code|post_handshake_auth| & \ref{HS.Ext.pha} &
CH\\ 
%
\code|signature_algorithms_cert| & \ref{HS.Ext.sa} &
CH, CR\\ 
%
\code|key_share| & \ref{HS.Ext.ks} &
CH, SH, HRR\\ 
\hline
\end{tabular}
\end{table}

Если сторона получает расширение в непредназначенном для него сообщении,
то она должна прервать Handshake с оповещением 
\token[ALERT.Err.ip]{illegal_parameter}. 

Если сообщение содержит несколько расширений, то они могут идти в любом 
порядке. Исключение составляет расширение \token[HS.Ext.psk]{pre_shared_key}, 
которое должно быть последним в \token[HS.CH]{ClientHello} (хотя может 
находиться в произвольном месте списка расширений \token[HS.SH]{ServerHello}).

Согласование параметров с помощью расширений выполняется в каждом сеансе 
Handshake, в том числе при возобновлении связи. 
%
Однако параметры секрета, который будет использоваться в механизме 0-RTT, 
изменить нельзя~--- они были согласованы в предыдущем соединении. При 
несоответствии параметров секрета параметрам текущего соединения механизм 0-RTT 
может быть отменен (см.~\ref{HS.Ext.ed}).

Перечень расширений может расширяться. При разработке новых расширений 
следует учитывать следующее:

\begin{itemize}
\item
сервер может не принять расширение, предложенное клиентом, как из-за ошибки, 
так и просто отказав в поддержке определенных функциональных возможностей. 
В первом случае следует прерывать Handshake с критическим оповещением, во 
втором~--- объяснять отказ в ответной части расширения;
\item
вплоть до завершения Handshake противник может манипулировать сообщениями 
протокола, вставляя, удаляя или перемещая расширения. Расширения следует 
проектировать так, чтобы противник не мог навязать использование (или наоборот 
отключение) определенных функциональных возможностей TLS. 
\end{itemize}

\subsection{Расширение \token{supported_versions}}\label{HS.Ext.sv} 

Клиент использует расширение \token{supported_versions} для информирования о 
поддерживаемых версиях TLS, сервер~--- для информирования о выбранной версии. 

Данные расширения \token{supported_versions} описываются типом 
\code{SupportedVersions}, который определяется следующим образом:

\begin{codeblock}
struct {
  select (Handshake.msg_type) {
    case client_hello:
      ProtocolVersion versions<2..254>;
    case server_hello: /* and HelloRetryRequest */
      ProtocolVersion selected_version;
  };
} SupportedVersions;
\end{codeblock}

В поле \code{versions} клиент перечисляет поддерживаемые версии в порядке 
убывания предпочтения. Сервер указывает выбранную версию в 
\code{selected_version}.

Клиент должен включить расширение \token{supported_versions} в сообщение 
\token[HS.CH]{ClientHello}, указав в расширении список версий TLS, которые он 
готов использовать. Версия 1.3, представленная числом \code{0x0304}, должна 
входить в список.

Если сервер получил \token[HS.CH]{ClientHello} без расширения 
\token{supported_versions} и если сервер поддерживает TLS версии 1.2, то 
для согласования версии он должен использовать схему, заданную в СТБ 34.101.65. 
%
Данное правило действует и тогда, когда поле \code{ClientHello.legacy_version} 
содержит значение \code{0x0304} или больше, т.~е. клиент предлагает 
использовать TLS версии 1.3 или выше, и при этом не отправляет 
\token{supported_versions}.
%
Сервер может прервать Handshake, если поле \code{ClientHello.legacy_version} 
содержит значение \code{0x0304} или больше.

Если сервер получил \token[HS.CH]{ClientHello} с расширением
\token{supported_versions}, то он должен игнорировать
\code{ClientHello.legacy_version} и при выборе версии использовать 
исключительно расширение. Сервер должен выбирать из версий, перечисленных в 
расширении, и при этом игнорировать версии, которые ему неизвестны,
\doubt{а также версии ниже 1.2}.
%
Допускается, что будет выбрана версия 1.2.
%
Сервер должен быть готов к обработке сообщения \token[HS.CH]{ClientHello}, в 
расширении~\token{supported_versions} которого отсутствует версия \code{0x0304}. 

Если сервер выбрал версию 1.2, то он должен указать эту версию в поле 
\code{ServerHello.legacy_version} и не должен включать в 
\token[HS.SH]{ServerHello} расширение \token{supported_versions}.
%
Если сервер выбрал версию 1.3, то расширение должно быть включено и в нем 
должна быть указана выбранная версию (\code{0x0304}). Кроме этого, сервер 
должен установить в \code{ServerHello.legacy_version} значение \code{0x0303} 
(TLS 1.2).

Клиент должен проверять наличие \token{supported_versions} в 
\token[HS.SH]{ServerHello}. Если расширение присутствует, то клиент должен 
игнорировать \code{ServerHello.legacy_version} и при определении версии 
использовать исключительно расширение. Если в расширении указана версия ниже 
1.3 или версия, которую клиент не предлагал, то клиент должен прервать 
Handshake с оповещением \token[ALERT.Err.ip]{illegal_parameter}.

При перезапуске Handshake сервер должен проинформировать о выборе версии в
сообщении \token[HS.HRR]{HelloRetryRequest} точно так же, как он это делает в
\token[HS.SH]{ServerHello}.

% todo: перечитать

\subsection{Расширение \token{cookie}}\label{HS.Ext.c}

Куки (cookie)~--- это данные, которые сервер отправляет клиенту на хранение. 
%
В куки сервер передает клиенту состояние, избегая необходимости хранить его 
самостоятельно.
%
Кроме этого, сервер может проверить достижимость сетевого адреса клиента 
(защита от атак <<отказ в обслуживании>>), предлагая тому возвратить присланные 
куки.

Данные расширения \token{cookie} описываются типом \code{Cookie}, который 
определяется следующим образом:

\begin{codeblock}
struct {
  opaque cookie<1..2^16-1>;
} Cookie;
\end{codeblock}

В поле \code{cookie} содержатся куки.

Сервер может отправить расширение \token{cookie} в сообщении
\token[HS.HRR]{HelloRetryRequest}. 
%
После получения \token[HS.HRR]{HelloRetryRequest} с расширением \token{cookie} 
клиент должен перенести расширение в повторное сообщение 
\token[HS.CH]{ClientHello}.
%
Клиент не должен включать расширение в первое \token[HS.CH]{ClientHello} в 
последующих соединениях с сервером.

\begin{note}
Отправка \token{cookie} в \token[HS.HRR]{HelloRetryRequest} является исключением 
из общего правила, согласно которому сервер может отправлять клиенту только те 
расширения, которые были включены в \token[HS.CH]{ClientHello}
(см.~\ref{HS.Ext.List}).
\end{note}

Сервер может указать в \token{cookie} защищенное хэш-значение первого 
сообщения \token[HS.CH]{ClientHello}. Получив повторное 
\token[HS.CH]{ClientHello}, сервер снимает защиту, определяет хэш-значение и   
использует его при обработке стенограммы Handshake (см.~\ref{CRYPTO.Transcr}).
%
При такой схеме взаимодействия серверу не нужно хранить первое 
\token[HS.CH]{ClientHello}. 

\begin{note}
\doubt{При определенных условиях сервер может реконструировать первое 
\token[HS.CH]{ClientHello} по повторному, а затем проверить, что 
хэш-значение реконструированного сообщения совпадает с присланным 
в~\token{cookie}.}
\end{note}

% info: https://mailarchive.ietf.org/arch/msg/tls/7rLQEHrqKaYo7u31iNBNsPI_OVg/
% info: https://mailarchive.ietf.org/arch/msg/tls/uOQBbe28_cIhQTmLgyPyPqtUvqk/
% info: https://mailarchive.ietf.org/arch/msg/tls/dPKZI5RXdAAb6mH8PhuOWnw1-1o/

Если сервер работает без состояния, то он не может отличить первое сообщение 
\token[HS.CH]{ClientHello} от повторного. Между этими сообщениями сервер может 
получить незащищенную запись типа \code{change_cipher_spec}. Сервер без 
состояния должен игнорировать эту запись (см.~\ref{RECORD.Rec}).

\subsection{Расширения \token{signature_algorithms} и 
\token{signature_algorithms_cert}}\label{HS.Ext.sa} 

Клиент и сервер используют расширения \token{signature_algorithms} и 
\token{signature_algorithms_cert} для информирования о поддерживаемых алгоритмах 
ЭЦП. Первое расширение касается подписи в сообщении 
\token[HS.CV]{CertificateVerify}, второе~--- подписей сертификатов.

Данные расширений \token{signature_algorithms} и 
\token{signature_algorithms_cert} описываются типом \code{SignatureSchemeList}, 
который определяется следующим образом:

\begin{codeblock}
struct {
  SignatureScheme supported_signature_algorithms<2..2^16-2>;
} SignatureSchemeList;
\end{codeblock}

В поле \code{supported_signature_algorithms} перечисляются идентификаторы 
алгоритмов ЭЦП в порядке убывания предпочтения.

Идентификаторы описываются типом \code{SignatureScheme}, который определяется 
следующим образом:

\begin{codeblock}
enum { (0xFFFF) } SignatureScheme;
\end{codeblock}

В~\cite{RFC8446} в~\code{SignatureScheme} включены идентификаторы некоторых
алгоритмов ЭЦП. Дополнительные идентификаторы включаются в~\ref{BSUITES.DS}.

Если аутентификация сервера проводится по его сертификату, т.~е. основана на 
проверке владения соответствующим личным ключом, то клиент должен включить 
\token{signature_algorithms} в сообщение \token[HS.CH]{ClientHello}.
%
Аналогичным образом, если сервер запрашивает аутентификацию клиента по 
сертификату, высылая тому сообщение \token[HS.CR]{CertificateRequest}, то это 
сообщение должно содержать расширение~\token{signature_algorithms}.

Если сторона аутентифицируется по сертификату, но не получила 
\token{signature_algorithms} в \token[HS.CH]{ClientHello} или 
\token[HS.CR]{CertificateRequest}, то она должна прервать  
Handshake с оповещением~\token[ALERT.Err.me]{missing_extension}.

Открытый ключ сертификата, который используется для аутентификации, должен 
соответствовать одному из алгоритмов, указанному 
в~\token{signature_algorithms}. 

Расширение~\token{signature_algorithms} может сопровождаться 
расширением~\token{signature_algorithms_cert}. В последнем расширении 
перечисляются алгоритмы, с помощью которых проверяются подписи в цепочке 
сертификатов аутентифицирируемой стороны. Если 
расширение~\token{signature_algorithms_cert} опущено, то перечень алгоритмов   
определяется по~\token{signature_algorithms}. 

Алгоритмы проверки подписей в цепочке сертификатов должны входить в 
перечень~\token{signature_algorithms_cert} или, если это расширение опущено,  
в перечень~\token{signature_algorithms}. 
%
Требование не распространяется на начальные сертификаты цепочек (точки доверия, 
как правило, корневые сертификаты), подписи которых могут не проверяться.

\subsection{Расширение \token{certificate_authorities}}\label{HS.Ext.ca} 

Расширение \token{certificate_authorities} используется для информирования
об УЦ, которым отправитель доверяет. Получателю следует учитывать 
информацию при выборе сертификата для аутентификации.

Данные расширения \token{certificate_authorities} описываются типом 
\code{CertificateAuthorities}, который определяется следующим образом:

\begin{codeblock}
opaque DistinguishedName<1..2^16-1>;

struct {
  DistinguishedName authorities<3..2^16-1>;
} CertificateAuthoritiesExtension;
\end{codeblock}

В поле \code{authorities} указывается список имен доверенных УЦ. Имя 
описывается типом~\code{Name} АСН.1, который определен в СТБ 34.101.19,
и кодируется по правилам~DER.
%
В качестве доверенного УЦ может указываться как корневой центр (точка 
доверия), так и промежуточный.

Клиент может отправить расширение \token{certificate_authorities}
в сообщении~\token[HS.CH]{ClientHello}, сервер~---
в сообщении~\token[HS.CR]{CertificateRequest}. 

\subsection{Расширение \token{oid_filters}}\label{HS.Ext.of} 

В расширении \token{oid_filters} сервер указывает требования к сертификатам, 
которые используются для аутентификации перед ним. Требования касаются 
расширений сертификатов, определенных в СТБ 34.101.19, 
и формулируются в виде пар <<идентификатор расширения~--- 
требуемое содержание>>.

Данные расширения \token{oid_filters} описываются типом 
\code{OIDFilterExtension}, который определяется следующим образом: 

\begin{codeblock}
struct {
  opaque certificate_extension_oid<1..2^8-1>;
  opaque certificate_extension_values<0..2^16-1>;
} OIDFilter;

struct {
  OIDFilter filters<0..2^16-1>;
} OIDFilterExtension;
\end{codeblock}

В поле \code{filters} указывается список элементов типа \code{OIDFilter}. 
%
Каждый элемент касается отдельного расширения сертификата и состоит из двух 
полей:
\begin{itemize}
\item
\code{certificate_extension_oid}~--- идентификатор расширения.
Описывается типом <<идентификатор объекта>> (OBJECT IDENTIFIER) АСН.1;

\item
\code{certificate_extension_values}~--- список требуемых значений расширения.
Определяется в соответствии со спецификацией расширения в СТБ 34.101.19.
\end{itemize}

Оба поля кодируются по правилам DER.

Если сервер использует расширение~\token{oid_filters}, то он должен 
включить его в сообщение~\token[HS.CR]{CertificateRequest}. 
%
Идентификаторы расширений сертификата в~\token{oid_filters} не должны 
повторяться.

Если сервер включил~\token{oid_filters} в~\token[HS.CR]{CertificateRequest}, то 
ответный сертификат клиента должен содержать все расширения, перечисленные
в~\token{oid_filters} и распознаваемые клиентом.
%
В этих расширениях должны присутствовать все требуемые значения.
%
Клиент должен игнорировать перечисленные в~\token{oid_filters} расширения 
сертификатов, идентификаторы которых он не распознал.
%
Если клиент проигнорировал некоторые расширения сертификатов и представил 
сертификат, который не полностью соответствует требованиям~\token{oid_filters}, 
то сервер может либо продолжить Handshake без аутентификации клиента, либо 
прервать протокол с оповещением \token[ALERT.Err.uc]{unsupported_certificate}.

Способ проверки требований к расширению сертификата меняется в зависимости от 
типа расширения. В настоящем стандарте определяются правила проверки следующих 
расширений.

\begin{enumerate}
\item
Расширение \code{KeyUsage} в сертификате клиента соответствует 
требованиям~\token{oid_filters}, если каждый бит назначения ключа, установленный 
в~\token{oid_filters}, установлен также в сертификате.
\item
Расширение \code{ExtKeyUsage} (название в соответствии с СТБ 34.101.78) в 
сертификате клиента соответствует требованиям~\token{oid_filters}, если каждый 
идентификатор области применения, указанный в~\token{oid_filters}, указан также 
в сертификате.
%
В~\token{oid_filters} не должен указываться идентификатор 
\code{anyExtendedKeyUsage}.
\end{enumerate}

За пределами настоящего стандарта могут устанавливаться 
правила проверки требований к другим расширениям сертификата. 

\subsection{Расширение \token{status_request}}\label{HS.Ext.sr} 

Расширение \token{status_request} используется для проверки статуса сертификата, 
используемого для аутентификации. Статус проверяется с помощью протокола 
OCSP, установленного в СТБ 34.101.26.

Данные расширения \token{status_request} описываются типом 
\code{StatusRequestExtension}, который определяется следующим образом: 

\begin{codeblock}
enum { ocsp(1), (255) } CertificateStatusType;

opaque ResponderID<1..2^16-1>;
opaque Extensions<0..2^16-1>;

struct {
  ResponderID responder_id_list<0..2^16-1>;
  Extensions request_extensions;
} OCSPStatusRequest;

struct {
  CertificateStatusType status_type = ocsp;
  OCSPStatusRequest request;
} CertificateStatusRequest;

opaque OCSPResponse<1..2^24-1>;

struct {
  CertificateStatusType status_type = ocsp;
  OCSPResponse response;
} CertificateStatus;

struct {} Empty;

struct {
  select (Handshake.msg_type) {
    case client_hello:
      CertificateStatusRequest request;
    case certificate_request: Empty;
    case certificate:
      CertificateStatus status;
  };
} StatusRequestExtension;
\end{codeblock}

Вложенный тип \code{OCSPStatusRequest} описывает OCSP-запросы,
тип \code{OCSPResponse}~--- OCSP-ответ.
%
Поля\code{OCSPStatusRequest} имеют следующее значение:
%
\begin{itemize}
\item
\code{responder_id_list}~--- список имен доверенных OCSP-серверов. Пустой 
список означает, что доверенные OCSP-серверы неявно известны, например, 
предварительно согласованы;
\item
\code{request_extensions}~--- OCSP-запросы.
\end{itemize}

OCSP-серверы, запросы и ответы описываются типами~\code{ResponderID}, 
\code{OCSPRequest} и \code{OCSPResponse} АСН.1, определенными в СТБ~34.101.26.
%
Данные АСН.1 кодируются по правилам DER.

Если клиент использует расширение \token{status_request}, то он должен 
включить его в сообщение \token[HS.CH]{ClientHello}.

Получив расширение \token{status_request}, сервер в 
сообщении~\token[HS.CT]{Certificate} вместе со своим сертификатом 
может выслать OCSP-ответ со сведениями о его статусе.
%
Для этого сервер формирует расширение \token{status_request} с OCSP-ответом 
и включает его в перечень расширений своего сертификата.
%
При подготовке OCSP-ответа серверу следует использовать сведения 
в присланном расширении \token{status_request}: перечни OCSP-серверов и 
OCSP-запросов. 

% info: https://mailarchive.ietf.org/arch/msg/tls/CB5_f1s4i3KAJDk1ZmOR2zeDIOg/
% * the status_request TLS extension doesn't provide a mechanism for stapling 
%   OCSP for intermediate certs

Сервер может запросить статус сертификата клиента, включив пустое расширение 
\token{status_request} в сообщение \token[HS.CR]{CertificateRequest}.
%
Получив запрос, клиент в сообщении~\token[HS.CT]{Certificate} вместе со 
своим сертификатом может выслать расширение \token{status_request} с 
OCSP-ответом о статусе сертификата.

\subsection{Расширение \token{post_handshake_auth}}\label{HS.Ext.pha} 

Расширение \token{post_handshake_auth} используется для информирования о том,
что клиент готов выполнить аутентификацию после Handshake 
(см.~\ref{HS.Post.Auth}). 

Расширение \token{post_handshake_auth} не содержит данные. Формально данные 
описываются пустым типом \code{PostHandshakeAuth}: 

\begin{codeblock}
struct {} PostHandshakeAuth;
\end{codeblock}

Если клиент использует расширение \token{post_handshake_auth}, то он должен 
включить его в сообщение \token[HS.CH]{ClientHello}.
%
Если сервер получил \token[HS.CH]{ClientHello} без расширения, то он
не должен отправлять клиенту сообщение \token[HS.CR]{CertificateRequest} 
по завершении Handshake. 

Сервер не должен отправлять расширение \token{post_handshake_auth}.

\subsection{Расширение \token{supported_groups}}\label{HS.Ext.sg} 

В расширении \token{supported_groups} отправитель перечисляет циклические 
группы, которые он может использовать в протоколе Диффи~-- Хеллмана.

Данные расширения \token{supported_groups} описываются типом 
\code{NamedGroupList}, который определяется следующим образом: 
%
\begin{codeblock}
struct {
  NamedGroup named_group_list<2..2^16-2>;
} NamedGroupList;
\end{codeblock}

% use: https://www.rfc-editor.org/errata/eid8411

В поле \code{named_group_list} указываются имена поддерживаемых групп
в порядке убывания предпочтения.

Имена групп описываются типом \code{NamedGroup}, который определяется 
следующим образом:

\begin{codeblock}
enum { (0xFFFF) } NamedGroup;
\end{codeblock}

В~\cite{RFC8446} в~\code{NamedGroup} включены имена некоторых групп. 
Дополнительные имена включаются в~\ref{BSUITES.Groups}. 

Клиент включает расширение \token{supported_groups} в сообщение 
\token[HS.CH]{ClientHello}, если он планирует использовать протокол 
Диффи~-- Хеллмана (режим DHE, см.~\ref{HS.Negot}) для формирования общих 
ключей.

Сервер может включить расширение \token{supported_groups} в сообщение
\token[HS.EE]{EncryptedExtensions}. Клиент не должен использовать сведения
в расширении сервера вплоть до завершения сеанса Handshake, но может использовать 
их в следующих сеансах для выбора групп в своих расширениях 
\token[HS.Ext.ks]{key_share}.
%
Если у сервера есть более предпочтительные группы, нежели те, что указаны в  
\token[HS.Ext.ks]{key_share} клиента, то серверу следует выслать 
расширение \token{supported_groups}, проинформировав тем самым клиента о своих 
предпочтениях. При этом серверу следует указать в \token{supported_groups} все 
поддерживаемые группы, независимо от того, поддерживаются они клиентом или нет. 

\subsection{Расширение \token{key_share}}\label{HS.Ext.ks} 

Расширение \token{key_share} содержит данные обмена между сторонами
в процессе выполнения протокола Диффи~-- Хеллмана.

Единица данных описывается типом \code{KeyShareEntry}, который определяется 
следующим образом:
%
\begin{codeblock}
struct {
  NamedGroup group;
  opaque key_exchange<1..2^16-1>;
} KeyShareEntry;
\end{codeblock}

Поля \code{KeyShareEntry} имеют следующее значение:
\begin{itemize}
\item
\code{group}~--- имя циклической группы, которая используется в протоколе 
Диффи~-- Хеллмана;
\item
\code{key_exchange}~--- одноразовый открытый ключ протокола.
%
Формат открытого ключа определяется в зависимости от используемой циклической 
группой. Форматы открытых ключей для некоторых известных групп 
определены в~\cite{RFC8446}, а также в~\ref{BSUITES.DH}.
\end{itemize}

Клиент отправляет расширение \token{key_share} в сообщении 
\token[HS.CH]{ClientHello} вместе с расширением 
\token[HS.Ext.sg]{supported_groups} (см.~\ref{HS.CH}).

Данные расширения \token{key_share} в \token[HS.CH]{ClientHello} описываются 
типом \code{KeyShareClientHello}, который определяется следующим образом:
%
\begin{codeblock}
struct {
  KeyShareEntry client_shares<0..2^16-1>;
} KeyShareClientHello;
\end{codeblock}

В поле \code{client_shares} клиент указывает перечень элементов 
типа \code{KeyShareEntry} в порядке убывания предпочтения.
%
Элементы должны соответствовать группам, перечисленным в 
\token[HS.Ext.sg]{supported_groups}, и должны идти в той же очередности.
%
При этом некоторые группы, в том числе самые предпочтительные, могут быть не 
представлены элементами \code{client_shares}.
%
Такая ситуация возникает, например, тогда, когда предпочтительная группа
является новой, скорее всего не будет поддержана сервером и поэтому клиенту 
нецелесообразно заранее вычислять открытый ключ для этой группы.

\begin{note*}
Тот факт, что клиент включил в \code{KeyShareClientHello} 
открытый ключ для группы $A$ и не включил для группы $B$, не означает,
что клиент предпочитает группу~$A$ группе~$B$. 
%
Поэтому выбор группы на основании \code{KeyShareClientHello} может дать в 
качестве результата не самую предпочтительную группу из тех, которые клиент и 
сервер одновременно поддерживают.
%
Если серверу требуется максимально точно учесть предпочтения клиента, то ему 
следует сначала выбрать группу на основании \token{supported_groups}, а затем,
в зависимости от того, поддержана ли группа открытым ключом в 
\code{KeyShareClientHello}, либо предложить использовать группу через 
\token[HS.HRR]{HelloRetryRequest}, либо выслать свой открытый ключ для  
группы в \token[HS.SH]{ServerHello}.
\end{note*}

Клиент может включить в \code{KeyShareClientHello} открытый ключ для каждой  
группы, указанной в \token{supported_groups}. Открытые ключи должны 
генерироваться независимо друг от друга. 
%
Клиент не должен включать в \code{KeyShareClientHello} несколько открытых 
ключей для одной и той же группы. 
%
Клиент не должен включать в \code{KeyShareClientHello} открытые ключи для 
групп, не указанных в \token{supported_groups}.
%
Сервер может проверять эти требования и прерывать Handshake с оповещением 
\token[ALERT.Err.ip]{illegal_parameter} при нарушении любого из них.

Клиент может отправить пустой список \code{KeyShareClientHello.client_shares}, 
запрашивая тем самым перезапуск Handshake с выбором группы со стороны сервера.
%
Сервер инициирует перезапуск самостоятельно, если ни одна из групп, указанных в
списке, его не устраивает. Перезапуск реализуется отправкой сообщения 
\token[HS.HRR]{HelloRetryRequest} с включением в него 
расширения~\token{key_share}.

Данные расширения \token{key_share} в \token[HS.HRR]{HelloRetryRequest} 
описываются типом \code{KeyShareHelloRetryRequest}, который определяется 
следующим образом: 
%
\begin{codeblock}
struct {
  NamedGroup selected_group;
} KeyShareHelloRetryRequest;
\end{codeblock}

В поле \code{selected_group} сервер указывает имя подходящей группы, 
выбирая среди групп, указанных клиентом в 
расширении~\token[HS.Ext.sg]{supported_groups}.

Получив расширение \token{key_share} в \token[HS.HRR]{HelloRetryRequest}, 
клиент должен проверить два условия:
\begin{enumerate}[label=\arabic*)]
\item
в \code{selected_group} указана одна из групп списка 
\token[HS.Ext.sg]{supported_groups};
\item
в \code{selected_group} не указана группа, для которой в расширении 
\token{key_share} предыдущего сообщения \token[HS.CH]{ClientHello}
был выслан открытый ключ.
\end{enumerate}

При нарушении любого из условий клиент должен прервать Handshake с оповещением 
\token[ALERT.Err.ip]{illegal_parameter}. 
%
Если условия соблюдены, то клиент отправляет повторное сообщение 
\token[HS.CH]{ClientHello}. В нем клиент должен изменить расширение
\token{key_share}. В новую редакцию расширения должен быть включен только один 
элемент типа \code{KeyShareEntry}, и он должен содержать открытый ключ для 
группы, указанной в \code{KeyShareHelloRetryRequest.selected_group}.

Если расширение \token{key_share} клиента устраивает сервер и он может 
продолжить Handshake без перезапуска, то сервер отвечает сообщением 
\token[HS.SH]{ServeHello}, также включая в него \token{key_share}.

Данные расширения \token{key_share} в \token[HS.SH]{ServerHello} 
описываются типом \code{KeyShareServerHello}, который определяется 
следующим образом: 

\begin{codeblock}
struct {
  KeyShareEntry server_share;
} KeyShareServerHello;
\end{codeblock}

В поле \code{server_share} сервер указывает свой одноразовый открытый ключ.

Открытый ключ сервера должен принадлежать одной из групп, для которых клиент 
представил открытый ключ в своем расширении \token{key_share}.
%
\doubt{Сервер не должен отправлять открытый ключ для групп, не включенных 
клиентом в расширение \token[HS.Ext.sg]{supported_groups}.}
%
% todo: но ведь это следствие требования в предыдущем абзаце
%
Сервер не должен отправлять открытый ключ, если клиент включил в
\token[HS.CH]{ClientHello} расширение
\token[HS.Ext.pkem]{psk_key_exchange_modes}, и в этом расширении установлен
режим \code{psk_ke} (cм. \ref{HS.Ext.pkem}).
%
Если выполнялся перезапуск Handshake и до \token[HS.SH]{ServerHello} сервер
отправлял сообщение \token[HS.HRR]{HelloRetryRequest}, то клиент должен
проверить, что открытый ключ сервера в \code{KeyShareServerHello} принадлежит
группе, указанной в \code{KeyShareHelloRetryRequest}. При нарушении этого
условия клиент должен прервать Handshake с оповещением
\token[ALERT.Err.ip]{illegal_parameter}.

\subsection{Расширение \token{psk_key_exchange_modes}}\label{HS.Ext.pkem} 

В расширении \token{psk_key_exchange_modes} клиент сообщает, какие режимы 
формирования общих ключей с участием PSK он поддерживает.
%
Расширение касается как внешнего PSK, согласованного за пределами TLS, 
так и PSK для возобновления связи, полученного ранее от сервера в сообщении  
\token[HS.NST]{NewSessionTicket}.

Данные расширения \token{psk_key_exchange_modes} описываются типом 
\code{PskKeyExchangeModes}, который определяется следующим образом: 

\begin{codeblock}
enum { psk_ke(0), psk_dhe_ke(1), (255) } PskKeyExchangeMode;

struct {
    PskKeyExchangeMode ke_modes<1..255>;
} PskKeyExchangeModes;
\end{codeblock}

В поле \code{ke_modes} указываются допустимые режимы:
\begin{itemize}
\item
\code{psk_ke}~--- режим PSK. В этом режиме сервер не должен отправлять 
расширение \token[HS.Ext.ks]{key_share}; 
\item
\code{psk_dhe_ke}~--- режим PSK+DHE. В этом режиме клиент и сервер должны 
обмениваться расширениями \token[HS.Ext.ks]{key_share}.
\end{itemize}

За пределами настоящего стандарта могут определяться дополнительные режимы.
Это должно быть сделано так, чтобы по сообщениям Handshake можно было 
однозначно определить, какой режим выбран. 

\begin{note*}
Выбор \code{psk_dhe_ke} подтверждается включением расширения 
\token[HS.Ext.ks]{key_share} в \token[HS.SH]{ServerHello}, выбор 
\code{psk_ke}~--- невключением.
\end{note*}

Если клиент включает в \token[HS.CH]{ClientHello} расширение 
\token[HS.Ext.psk]{pre_shared_key}, то он должен сопроводить его 
расширением \token{psk_key_exchange_modes}.
%
Если сервер получает \token[HS.CH]{ClientHello} с 
\token[HS.Ext.psk]{pre_shared_key}, но без \token{psk_key_exchange_modes}, 
то он должен прервать Handshake. 
%
Сервер не должен использовать режим формирования общих ключей,
отличный от заданных клиентом. 
%
Серверу не следует отправлять в \token[HS.NST]{NewSessionTicket}
секрет PSK, предназначенный для использования в режимах, которые не 
поддерживает клиент. Попытка возобновления связи с таким секретом
приведет к ошибке.

% info: https://mailarchive.ietf.org/arch/msg/tls/qNQibGvRsBv55KlpFuUkDWZUcIs/
% * What this text is supposed to mean is that the server shouldn't send
%   tickets if it doesn't like the client's psk_ke modes. So, say that the 
%   client only supports psk_dhe_ke and the server only supports psk_ke, 
%   then the server shouldn't advertise tickets, because future resumption will
%   be useless. 

Сервер не должен отправлять расширение \token{psk_key_exchange_modes}.

\subsection{Расширение \token{early_data}}\label{HS.Ext.ed} 

Расширение \token{early_data} используется для информирования о применении 
механизма 0-RTT (см.~\ref{COMMON.ZeroRTT}) и для передачи его настроек.

Данные расширения \code{extension_data} описываются типом 
\code{EarlyDataIndication}, который определяется следующим образом: 

\begin{codeblock}
struct {} Empty;

struct {
  select (Handshake.msg_type) {
    case new_session_ticket:
      uint32 max_early_data_size;
    case client_hello: Empty;
    case encrypted_extensions: Empty;
  };
} EarlyDataIndication;
\end{codeblock}

Если имеется секрет PSK, который годится для использования 
в механизме 0-RTT, то клиент может применить механизм.
%
Для этого клиент должен включить в \token[HS.CH]{ClientHello} расширения 
\token[HS.Ext.psk]{pre_shared_key} и \token{early_data}. В первом расширении 
клиент описывает секрет, во втором информирует об его использовании для 
защиты ранних прикладных данных. 
%
В \token[HS.Ext.psk]{pre_shared_key} используемый секрет должен быть описан 
первым.

Параметры 0-RTT определяются атрибутами используемого секрета. 
%
При согласовании секрета за пределами TLS атрибуты определяются в момент 
согласования.
%
Если секрет был согласован в предыдущем соединении, то атрибуты определяются 
параметрами и сообщениями этого соединения.

При согласовании секрета в предыдущем соединении сервер высылает клиенту 
сообщение \token[HS.NST]{NewSessionTicket}. Это сообщение должно  
содержать расширение \token{early_data}. В поле 
\code{EarlyDataIndication.max_early_data_size} расширения должен быть  
указан лимит на объем ранних прикладных данных (см. \ref{HS.NST}).

При использовании механизма 0-RTT сообщения Handshake и прикладные данные 
первой (ранней) очереди имеют тот же формат и защищаются так же, как сообщения 
и данные последующих очередей.
%
Единственное отличие состоит в том, что при защите ранних прикладных данных 
используются специальные ключ и начальная синхропосылка, построенные
по \code{client_early_traffic_secret} (см.~\ref{CRYPTO.Schedule}).
%
Если сервер согласился с использованием механизма 0-RTT, то после приема его 
сообщения \token[HS.F]{Finished} клиент отправляет в ответ сообщение
\token[HS.EOED]{EndOfEarlyData}, информируя тем самым о завершении ранней 
защиты и смене ключей. Сообщение \token[HS.EOED]{EndOfEarlyData} защищается так 
же, как ранние прикладные данные.

Сервер, который получает сообщение \token[HS.CH]{ClientHello} с расширением 
\token{early_data}, должен поступить одним из следующих способов.

\begin{enumerate}
\item
Сервер игнорирует расширение \token{early_data} и действует по правилам 1-RTT,
пропуская ранние прикладные данные клиента.
%
Для этого сервер обрабатывает присылаемые защищенные записи Record, которым в 
сумме соответствует не более \code{max_early_data_size} байтов открытого текста, 
и снимает с них защиту на ключе, построенном
по~\code{client_handshake_traffic_secret} (см.~\ref{CRYPTO.Schedule}).
%
Сервер отбрасывает записи, при снятии защиты с которых происходят ошибки. Первая 
запись, с которой успешно снята защита, считается началом второй очереди 
сообщений клиента.
%
С этой записи сервер переходит на выполнение Handshake по схеме 1-RTT.

\item
Сервер предлагает клиенту исправить \token[HS.CH]{ClientHello}, отправляя
сообщение \token[HS.HRR]{HelloRetryRequest}.
%
Клиент не должен включать \token{early_data} в повторное сообщение 
\token[HS.CH]{ClientHello}, отказываясь тем самым от 0-RTT.
%
Сервер пропускает ранние прикладные данные клиента, отбрасывая записи Record с типом 
\code{application_data}. Это маскировочный тип, который означает, что записи защищены 
(см.~\ref{RECORD.Enc}). Сервер отбрасывает защищенные записи, которым в сумме 
соответствует не более \code{max_early_data_size} байтов открытого текста.
%
Сервер находит запись с повторным сообщением \token[HS.CH]{ClientHello} и с 
этого момента выполняет Handshake по схеме 2-RTT.

\item
Сервер сообщает о поддержке 0-RTT, отправляя клиенту сообщение 
\token[HS.EE]{EncryptedExtensions} с расширением \token{early_data}.
%
Поддержка означает, что сервер будет принимать все ранние прикладные данные 
клиента. При этом в момент отправки \token[HS.EE]{EncryptedExtensions} часть 
этих данных может быть в процессе доставки.
\end{enumerate}

Прежде чем поддержать 0-RTT, сервер должен принять расширение 
\token[HS.Ext.psk]{pre_shared_key} в сообщении \token[HS.CH]{ClientHello},
обработать его по правилам, заданным в~\ref{HS.Ext.psk}, и определить первый 
секрет, описанный в расширении.

% use: https://www.rfc-editor.org/errata/eid6147

После этого сервер должен проверить, что следующие параметры текущего соединения 
соответствуют атрибутам секрета:
\begin{itemize} 
\item
выбранная версия TLS;
%
% use:https://www.rfc-editor.org/errata/eid6146
%
\item
выбранный криптонабор;
\item
прикладной протокол, выбранный с помощью расширения 
\token[HS.Ext.alpn]{application_layer_protocol_negotiation} (см. 
таблицу~\ref{Table.HS.Exts}), если таковой имеется.
\end{itemize}
%
Если какая-либо из проверок завершена с ошибкой, то сервер должен пропустить ранние 
прикладные данные клиента по правилам, заданным выше, переходя таким образом 
к 1-RTT или 2-RTT.

% use: https://www.rfc-editor.org/errata/eid6138

Дополнительно сервер анализирует описание секрета в расширении
\token[HS.Ext.psk]{pre_shared_key} клиента. Описание представляет собой
структуру типа~\code{PskIdentity} (см.~\ref{HS.Ext.psk}). Если секрет был
согласован через билет, отосланный сервером в сообщении
\token[HS.NST]{NewSessionTicket} предыдущего соединения, то поле
\code{PskIdentity.obfuscated_ticket_age} содержит аддитивно зашумленное время 
жизни билета.
%
Сервер проверяет время жизни, используя сохраненные реквизиты билета,
в том числе величину зашумления, которая была отослана в поле 
\code{ticket_age_add} сообщения~\token[HS.NST]{NewSessionTicket} 
(см.~\ref{HS.NST}).
%
Сервер находит разность между \code{obfuscated_ticket_age} и 
\code{ticket_age_add} по модулю~$2^{32}$ и проверяет, что 
результат незначительно отличается от времени в миллисекундах с момента выпуска 
билета. При нарушении этого условия серверу следует продолжить Handshake, но 
без поддержки 0-RTT. Кроме этого, серверу не следует полагаться на то, что 
сообщение \token[HS.CH]{ClientHello} не просрочено.

Если сервер принял расширение \token{early_data}, то он должен обрабатывать 
записи с ранними прикладными данными по обычным правилам. В частности, при 
ошибке снятия защиты с записи сервер должен прервать Handshake с оповещением 
\token[ALERT.Err.brm]{bad_record_mac}.

При использовании механизма 0-RTT клиенту разрешается передавать ранние
прикладные данные вплоть до сообщения \token[HS.F]{Finished} от сервера
(см.~\ref{HS.EOED}). После получения \token[HS.F]{Finished} клиент отправляет
сообщение \token[HS.EOED]{EndOfEarlyData} и стороны переходят к завершению Handshake.
Чтобы избежать взаимной блокировки, сервер после приема \token[HS.CH]{ClientHello} должен
сразу же отправлять свою очередь сообщений, включая \token[HS.SH]{ServerHello}
с расширением \token{early_data}, не дожидаясь сообщения 
\token[HS.EOED]{EndOfEarlyData} от клиента.

Если сервер отклоняет \token{early_data}, то клиент может повторить отправку 
ранних прикладных данных, только теперь на обычной стадии~--- по завершении 
Handshake.
%
Прикладной протокол может по-разному интерпретировать соединение
до завершения Handshake и после завершения. Поэтому клиенту не следует 
автоматически, без участия прикладного протокола, повторно отправлять ранние 
прикладные данные.
%
Клиент не должен автоматически повторно отправлять данные, если в процессе 
выполнения Handshake изменяется прикладной протокол, согласуемый
через расширение \token[HS.Ext.alpn]{application_layer_protocol_negotiation}.

При определении расширений за пределами настоящего стандарта должны быть 
оговорены правила их применения относительно механизма 0-RTT.

\subsection{Расширение \token{pre_shared_key}}\label{HS.Ext.psk} 

Расширение \token{pre_shared_key} используется для согласования 
идентификатора секрета, который будет использоваться для формирования общих 
ключей в режимах PSK и PSK+DHE (см.~\ref{COMMON.Components}).

Данные расширения \token{pre_shared_key} описываются типом 
\code{PreSharedKeyExtension}, который определяется следующим образом:

\begin{codeblock}
struct {
  opaque identity<1..2^16-1>;
  uint32 obfuscated_ticket_age;
} PskIdentity;

opaque PskBinderEntry<32..255>;

struct {
  PskIdentity identities<7..2^16-1>;
  PskBinderEntry binders<33..2^16-1>;
} OfferedPsks;

struct {
  select (Handshake.msg_type) {
    case client_hello: OfferedPsks;
    case server_hello: 
      uint16 selected_identity;
  };
} PreSharedKeyExtension;
\end{codeblock}

Клиент отправляет расширение \token{pre_shared_key} в сообщении 
\token[HS.CH]{ClientHello}. Клиент указывает в расширении структуру типа 
\code{OfferedPsks}, поля которой имеют следующие значения:

\begin{itemize}
\item
\code{identities}~--- список структур типа \code{PskIdentity}, которые 
описывают секреты. При включении в \token[HS.CH]{ClientHello} расширения 
\token[HS.Ext.ed]{early_data} первым в списке должно идти описание секрета для 
механизма 0-RTT (см. \ref{HS.Ext.ed});

\item
\code{binders}~--- список PSK-скрепок. Каждый элемент \code{identities} должен 
сопровождаться скрепкой, порядок скрепок должен повторять порядок элементов.
Правила вычисления скрепок определяются в~\ref{CRYPTO.Binder}.
\end{itemize}

Поля \code{PskIdentity} имеют следующее значение:

\begin{itemize}
\item
\code{identity}~--- идентификатор секрета. В его качестве может выступать 
билет, высылаемый в сообщении \token[HS.NST]{NewSessionTicket}, или имя, 
согласуемое за пределами TLS;
\item
\code{obfuscated_ticket_age}~--- зашумленное время жизни секрета. 
%
Если секрет согласован в предыдущем соединении через билет сообщения
\token[HS.NST]{NewSessionTicket}, то в поле должна быть указана сумма по 
модулю~$2^{32}$ числа миллисекунд, прошедших с момента получения билета, 
и величины зашумления \code{ticket_age_add}, указанного в сообщении 
(см.~\ref{HS.NST}). 
%
Клиент не должен использовать секрет, если с момента получения билета прошло 
больше \code{ticket_lifetime} секунд, где \code{ticket_lifetime}~--- величина, 
указанная в \token[HS.NST]{NewSessionTicket}.
%
Если секрет согласован за пределами TLS, то в поле \code{obfuscated_ticket_age}
клиенту следует задавать значение~$0$, а сервер должен игнорировать поле.
\end{itemize}

\begin{note*}
Зашумление времени жизни билета не позволяет пассивному противнику определить, 
в каком из предыдущих соединений билет был выпущен.
%
Срок действия билета ограничен неделей и поэтому укладывается в 32-битовое 
поле \code{ticket_lifetime}.
\end{note*}

Сервер отправляет расширение \token{pre_shared_key} в сообщении 
\token[HS.SH]{ServerHello}. Сервер указывает в расширении номер выбранного 
секрета \code{selected_identity}. Это порядковый номер в списке 
\code{OfferedPsks.identities}, предложенным клиентом. Нумерация в списке  
начинается с~$0$.

Каждый секрет ассоциирован с определенным алгоритмом хэширования. 
%
Если секрет согласован в предыдущем соединении, то этим алгоритмом является
алгоритм хэширования криптонабора, который использовался в соединении.
%
Если секрет согласовывается за пределами TLS, то алгоритм хэширования должен 
определяться в момент согласования.
%
При выборе секрета сервер должен обеспечить его совместимость с криптонабором 
текущего соединения: алгоритм хэширования криптонабора должен совпадать с алгоритмом
хэширования, который ассоциирован с секретом.

% skip: the Hash algorithm MUST be set when the PSK is established or default 
% to SHA-256 if no such algorithm is defined

% skip: In TLS versions prior to TLS 1.3, the Server Name Indication (SNI) 
% value was intended to be associated with the session (Section 3 of [RFC 
% 6066]), with the server being required to enforce that the SNI value 
% associated with the session matches the one specified in the resumption 
% handshake. However, in reality the implementations were not consistent on 
% which of two supplied SNI values they would use, leading to the consistency 
% requirement being de facto enforced by the clients. In TLS 1.3, the SNI value 
% is always explicitly specified in the resumption handshake, and there is no 
% need for the server to associate an SNI value with the ticket. Clients, 
% however, SHOULD store the SNI with the PSK to fulfill the requirements of 
% Section 4.6.1.

% use: https://www.rfc-editor.org/errata/eid5627

% skip: Implementor's note: When session resumption is the primary use case of 
% PSKs, the most straightforward way to implement the PSK/cipher suite matching 
% requirements is to negotiate the cipher suite first and then exclude any 
% incompatible PSKs. Any unknown PSKs (e.g., ones not in the PSK database or 
% encrypted with an unknown key) SHOULD simply be ignored. If no acceptable PSKs 
% are found, the server SHOULD perform a non-PSK handshake if possible. If 
% backward compatibility is important, client-provided, externally established 
% PSKs SHOULD influence cipher suite selection.

Перед выбором секрета сервер должен проверить соответствующую PSK-скрепку.
Правила проверки определены в~\ref{CRYPTO.Binder}. Если скрепка отсутствует или 
оказывается некорректной, то сервер должен прервать Handshake. Серверу не следует 
проверять сразу несколько скрепок~--- следует выбрать единственный секрет и 
проверить только его скрепку.

% skip: See Section 8.2 and Appendix E.6 for the security rationale for this 
% requirement. In order to accept PSK key establishment, the server sends a 
% "pre_shared_key" extension indicating the selected identity.

Получив сообщение \token[HS.SH]{ServerHello}, клиент находит расширение 
\token{pre_shared_key} и проверяет, что указанный в нем номер 
\code{selected_identity} не выходит за границы списка 
\code{OfferedPsks.identities}, предложенного клиентом.
%
Кроме этого, клиент проверяет, что выбранные сервером криптонабор и секрет 
совместимы друг с другом.
%
Наконец, клиент проверяет, что \token[HS.SH]{ServerHello} содержит 
расширение \token[HS.Ext.ks]{key_share}, если этого требовало расширение 
\token[HS.Ext.pkem]{psk_key_exchange_modes}, отправленное клиентом 
в~\token[HS.CH]{ClientHello}.
%
При нарушении любого из условий клиент должен прервать Handshake с оповещением
\token[ALERT.Err.ip]{illegal_parameter}.

Если сервер включил в \token[HS.SH]{ServerHello} расширение
\token[HS.Ext.ed]{early_data}, то клиент должен убедиться, что выбранный 
сервером номер \code{selected_identity} равняется 0. При нарушении этого 
условия клиент должен прервать Handshake с оповещением 
\token[ALERT.Err.ip]{illegal_parameter}.

Чтобы упростить построение PSK-скрепок, расширение \token{pre_shared_key} должно 
идти последним в списке расширений \token[HS.CH]{ClientHello}. Сервер должен 
проверять данное условие и прерывать Handshake с оповещением 
\token[ALERT.Err.ip]{illegal_parameter} в случае его нарушения.

