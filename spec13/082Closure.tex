\section{Оповещения о закрытии соединения}\label{ALERT.Closure}

Клиент и сервер должны информировать друг друга о закрытии соединения, чтобы
избежать ситуации, в которой одна сторона считает, что соединение закрыто, 
а вторая сторона считает наоборот.

Определены следующие оповещения о закрытии соединения:

\begin{itemize}
\item\label{ALERT.Closure.cn}
\code{close_notify}~--- отправитель информирует получателя о том, что больше не 
будет посылать сообщения в данном соединении. Предполагается, что оповещение
будет доставлено прежде чем соединение будет закрыто на транспортном уровне; 

% info: https://mailarchive.ietf.org/arch/msg/tls/yoenkuuNE70yGmeONa2VIOtVSKY/
% * It is assumed that closing the write side of a connection reliably 
%   delivers pending data before destroying the transport. 
% * It means that you deliver the close_notify before sending FIN (if on TCP).

\item\label{ALERT.Closure.uc}
\code{user_canceled}~--- отправитель информирует получателя о том, что он 
отказывается от продолжения Handshake по причине, не связанной с ошибкой.
%
Оповещение \code{user_canceled} следует использовать только в Handshake. 
Для информирования об отказе по завершении Handshake следует 
использовать~\code{close_notify}.
%
После \code{user_canceled} следует отправлять \code{close_notify}.
\end{itemize}

Указанные оповещения являются предупредительными. 

% use: https://www.rfc-editor.org/errata/eid7303

Закрытие соединения может инициировать любая сторона. При закрытии она должна 
отправить оповещение \token[ALERT.Closure.cn]{close_notify}, только если не 
отправляла ранее критическое оповещение.
%
Любые данные, полученные после \token[ALERT.Closure.cn]{close_notify}, должны 
игнорироваться.

Если соединение разорвано на транспортном уровне до 
приема~\token[ALERT.Closure.cn]{close_notify}, то получатель не может быть 
уверен, что он получил все предназначенные ему данные. Тем не менее, получатель 
не обязан дожидаться~\token[ALERT.Closure.cn]{close_notify} перед закрытием 
соединения со своей стороны.

Если TLS используется в прикладном протоколе и по завершении TLS предполагается 
продолжение передачи данных на транспортном уровне, то сторона TLS должна  
дождаться~\token[ALERT.Closure.cn]{close_notify} перед тем, как информировать 
прикладной протокол о завершении. Если продолжение не предполагается, то 
сторона может действовать по своему усмотрению.
