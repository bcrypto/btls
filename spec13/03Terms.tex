\chapter{Термины и определения}\label{TERMS}

В настоящем стандарте применяют термины, установленные в СТБ~34.101.19, 
СТБ~34.101.31, СТБ~34.101.45, СТБ~34.101.47 и СТБ~34.101.78, а также следующие 
термины с соответствующими определениями: 

{\bf \thedefctr~аутентификация}:
Проверка подлинности стороны.

{\bf \thedefctr~билет (ticket)}:
Ссылка на секрет, согласованный в соединении TLS.

{\bf \thedefctr~возобновление связи (resumption)}: 
Ускоренный вариант Handshake, в котором стороны используют секрет, 
согласованный в предыдущем соединении. 

{\bf \thedefctr~запись (record)}:
Единица данных в протоколе Record.

{\bf \thedefctr~клиент (client)}:
Сторона, которая инициирует выполнение протокола TLS.

% RFC8446: The endpoint initiating the TLS connection.

{\bf \thedefctr~криптонабор (cipher suite)}:
Алгоритмы аутентифицированного шифрования в связке с алгоритмом хэширования,
которые используются для защиты пересылаемых данных, построения ключей, 
аутентификации.

{\bf \thedefctr~одноразовый ключ (ephemeral key)}:
Ключ, который создается, используется и уничтожается в течение одного сеанса 
протокола. 

% СТБ 34.101.66

{\bf \thedefctr~оповещение (alert)}:
Сообщение протокола Alert.

{\bf \thedefctr~параметры защиты (security parameters)}:
Параметры, которые определяют правила защиты соединения, в том числе: 
версия TLS, криптонабор, параметры протокола Диффи~--- Хеллмана, перечни 
алгоритмов электронной цифровой подписи и предварительно согласованных 
секретов, режим формирования общих ключей.

{\bf \thedefctr~предварительно согласованый секрет; PSK (pre-shared secret)}:
Секрет, согласованный сторонами до установки соединения: за пределами TLS или в 
предыдущем соединении.

{\bf \thedefctr~прикладной протокол (application protocol)}:
Протокол, который выполняется поверх протокола TLS.

{\bf \thedefctr~прикладные данные (application data)}:
Данные прикладных протоколов, которые пересылаются с помощью TLS.

{\bf \thedefctr~протокол защиты транспортного уровня; протокол TLS 
(transport layer security protocol)}: 
Определяемый в настоящем стандарте криптографический протокол, который
обеспечивает аутентификацию сторон, конфиденциальность и контроль целостности и
подлинности данных, передаваемых между сторонами на транспортном
коммуникационном уровне.

\doubt{\bf \thedefctr~протокол Диффи~--- Хеллмана}:
Криптографический протокол, с помощью которого две стороны формируют 
общий секретный ключ, используя циклическую группу, обмениваясь открытыми 
ключами, элементами этой группы, и держа в секрете соответствующие личные 
ключи. 

\begin{note}
Имеется несколько вариантов протокола в зависимости от того, какие ключи 
использует та или иная сторона: долговременные (статические) или 
одноразовые (эфемерные). В настоящем стандарте у обеих сторон протокола 
Диффи~--- Хеллмана ключи одноразовые.
\end{note}

\begin{note}
В протоколе Диффи~--- Хеллмана используется конечная циклическая группа большого
(как правило простого) порядка. Обычно это либо подгруппа мультипликативной
группы конечного поля, либо группа точек эллиптической кривой.
%
Редакции протокола, в которых используются группы первого типа, принято снабжать
аббревиатурой FF (finite field), второго~--- аббревиатурой EC (elliptic curve).
%
В настоящем стандарте группы различных типов не дифференцируются и аббревиатуры 
не используются.
\end{note}

{\bf \thedefctr~протокол Alert}:
Протокол обмена сообщениями о закрытии соединения и об ошибках;
часть протокола TLS.

{\bf \thedefctr~протокол Handshake}:
Протокол согласования параметров, формирования общих ключей и аутентификации; 
часть протокола TLS. 

% RFC8446: An initial negotiation between client and server that establishes 
% the parameters of their subsequent interactions within TLS.

{\bf \thedefctr~протокол Record}:
Протокол передачи данных, в том числе в защищенном виде; 
часть протокола TLS. 

{\bf \thedefctr~расширение (extension)}:
Стуктура данных, которая входит в сообщение Handshake и служит для 
согласования параметров, обмена ключевым материалом, аутентификации.

{\bf \thedefctr~сервер (server)}:
Сторона, которая выполняет протокол TLS по запросу клиента.

% RFC8446: The endpoint that did not initiate the TLS connection.

{\bf \thedefctr~соединение (connection)}:
Cвязь между сторонами на транспортном коммуникационном уровне. 

% RFC8446: A transport-layer connection between two endpoints.

{\bf \thedefctr~стенограмма протокола; стенограмма (transcript)}: 
Последовательность сообщений протокола, переданных между сторонами к 
определенному моменту.

{\bf \thedefctr~циклическая группа}:
Алгебраическая группа, все элементы которой являются степенями или кратными
одного элемента.

\begin{note*}
Степенями при мультипликативной записи группы и кратными при аддитивной.
\end{note*}

{\bf \thedefctr~PSK-скрепка (PSK-binder)}:
Данные, которые связывают предварительно согласованный секрет с текущим 
соединением и возможно одним из предыдущих.

