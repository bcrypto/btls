\section{Сообщения}\label{HS.Msg}

% skip: Протокол Handshake отвечает за согласование параметров TLS, 
% формирование общих ключей и аутентификацию.

Сообщения Handshake описываются одноименным типом, который определяется
следующим образом:

\begin{codeblock}
enum {
  client_hello(1),
  server_hello(2),
  new_session_ticket(4),
  end_of_early_data(5),
  encrypted_extensions(8),
  certificate(11),
  certificate_request(13),
  certificate_verify(15),
  finished(20),
  key_update(24),
  message_hash(254),
  (255)
} HandshakeType;

struct {
  HandshakeType msg_type;
  uint24 length;
  select (Handshake.msg_type) {
    case client_hello:          ClientHello;
    case server_hello:          ServerHello;
    case end_of_early_data:     EndOfEarlyData;
    case encrypted_extensions:  EncryptedExtensions;
    case certificate_request:   CertificateRequest;
    case certificate:           Certificate;
    case certificate_verify:    CertificateVerify;
    case finished:              Finished;
    case new_session_ticket:    NewSessionTicket;
    case key_update:            KeyUpdate;
  };
} Handshake;
\end{codeblock}

Поля \code{Handshake} имеют следующее значение:
\begin{itemize}
\item
\code{msg_type}~--- тип сообщения. Описывается перечислением 
\lstinline{HandshakeType};

\item
\code{length}~--- длина в байтах следующего поля.
\end{itemize}

В поле, которое следует за \code{length}, задается тело сообщения.
Его формат определяется типом \code{msg_type}.

Сообщения Handshake передаются с помощью протокола Record.
В зависимости от типа сообщения, оно может передаваться в открытом виде или 
быть защищено.

Сообщения Handshake должны отправляться в порядке, который указан 
в~\ref{CRYPTO.Transcr} и представлен на рисунках~\ref{Fig.COMMON.Phases}~--- 
\ref{Fig.COMMON.ZeroRTT}.
%
Сторона, которая получает сообщение с нарушением порядка, должна прервать 
Handshake с оповещением \token[ALERT.Err.um]{unexpected_message}.

% skip: New handshake message types are assigned by IANA as described in 
% Section 11. 
