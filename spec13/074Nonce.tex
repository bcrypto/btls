\section{Синхропосылки}\label{RECORD.Nonce}

При передаче и приеме записей используются независимые 64-битовые порядковые номера.
Номер должен устанавливаться равным~$0$ в самом начале обмена данными и после 
смены ключей. Номер должен увеличиваться на~$1$ после обработки (передачи 
или приема) каждой записи. 

Число всевозможных порядковых номеров, $2^{64}$, весьма велико и поэтому при 
обработке записей номер не переполнится. Для принудительного сброса 
номера следует обновить ключи (см.~\ref{HS.KU}) или разорвать соединение.

Алгоритмы аутентифицированного шифрования, которые используются для защиты 
записей, характеризуются минимальной и максимальной длиной синхропосылок в байтах.
Алгоритмы, у которых максимальная длина меньше 8, не должны использоваться в TLS. 
Длина синхропосылки для защиты записи, \code{iv_length}, должна выбираться 
как максимум из минимальной длины и $8$.

Синхропосылка формируется следующим образом: 
\begin{enumerate}
\item 
Порядковый номер записи кодируется строкой байтов по правилу <<от старших к
младшим>> (big-endian, сетевой порядок байтов), а затем дополняется слева 
нулями до \code{iv_length} байтов.

\item 
Полученная строка складывается поразрядно по модулю 2 с начальной 
синхропосылкой \code{client_write_iv} или \code{server_write_iv} в зависимости 
от того, кто является отправителем: клиент или сервер.
Начальная синхропосылка формируется по правилам, заданным в~\ref{CRYPTO.TKeys}.

\item 
Результат сложения объявляется искомой синхропосылкой.
\end{enumerate}
