\section{Аутентификация}\label{HS.Auth} 

\subsection{Сообщение \token{Certificate}}\label{HS.CT} 

В сообщении \token{Certificate} сторона передает свой сертификат для
аутентификации, а также другие сертификаты и данные, которые подтверждают его
подлинность.

Формат сообщения \token{Certificate} описывается одноименным типом:
%
\begin{codeblock}
enum {
  X509(0),
  RawPublicKey(2),
  (255)
} CertificateType;

struct {
  select (certificate_type) {
    case RawPublicKey:
      opaque ASN1_subjectPublicKeyInfo<1..2^24-1>;
    case X509:
      opaque cert_data<1..2^24-1>;
  };
  Extension extensions<0..2^16-1>;
} CertificateEntry;

struct {
  opaque certificate_request_context<0..2^8-1>;
  CertificateEntry certificate_list<0..2^24-1>;
} Certificate;
\end{codeblock}

Поля \code{Certificate} имеют следующее значение:
%
\begin{itemize}
\item
\code{certificate_request_context}~--- синхропосылка. Если сообщение является
ответом на запрос \token[HS.CR]{CertificateRequest}, то в поле должна быть 
повторена синхропосылка запроса. Иначе поле должно быть пустым (иметь нулевую 
длину); 

\item
\code{certificate_list}~--- список (цепочка) структур типа
\code{CertificateEntry}, каждая из которых содержит один сертификат 
и сопровождающий его набор расширений.
\end{itemize}

Если стороны не согласовывали тип сертификата 
с помощью расширения \token[HS.Ext.ct]{server_certificate_type}
или \token[HS.Ext.ct]{client_certificate_type}, или если согласован тип 
\code{CertificateType.X509}, то элементы списка \code{certificate_list} должны 
содержать сертификаты X.509 (СТБ 34.101.19). Сертификат кодируется по правилам 
DER, код размещается в поле \code{CertificateEntry.cert_data}. 
%
Сертификат отправителя сообщения \token{Certificate} должен быть 
первым в цепочке \code{certificate_list}. Следующие сертификаты следует 
располагать так, чтобы каждый из них удостоверял предыдущий.
%
Поскольку точки доверия (корневые сертификаты) согласуются за пределами TLS, 
они могут быть исключены из цепочек при условии, что стороны знают, как их 
восстановить.

% skip: Prior to TLS 1.3, "certificate_list" ordering required each certificate 
% to certify the one immediately preceding it; however, some implementations 
% allowed some flexibility. Servers sometimes send both a current and 
% deprecated intermediate for transitional purposes, and others are simply 
% configured incorrectly, but these cases can nonetheless be validated 
% properly. For maximum compatibility, all implementations SHOULD be prepared 
% to handle potentially extraneous certificates and arbitrary orderings from 
% any TLS version, with the exception of the end-entity certificate which MUST 
% be first.  

Если согласован тип сертификата \code{CertificateType.RawPublicKey}, то 
список \code{certificate_list} должен содержать не более одного 
элемента \code{CertificateEntry}. Если список непуст, то в поле 
\code{ASN1_subjectPublicKeyInfo} его единственного элемента размещается 
DER-код контейнера \code{SubjectPublicKeyInfo} (см.~\ref{HS.Ext.List}).

\begin{note*}
Контейнер \code{SubjectPublicKeyInfo} содержит открытый ключ, но не 
удостоверяющую подпись, обычную для сертификатов. Тем не менее, 
контейнер выполняет функции сертификата, если сторона TLS ведет список
доверенных контейнеров, согласованный за пределами TLS, и проверяет,
что присланный в \token{Certificate} контейнер входит в список.
\end{note*}

% skip: Note that if raw public keys [RFC 7250] or the cached information 
% extension [RFC 7924] are in use, then this message will not contain a 
% certificate but rather some other value corresponding to the server's 
% long-term key. 

% skip: The OpenPGP certificate type [RFC6091] MUST NOT be used with TLS 1.3.

Сопровождающие сертификат расширения задаются в поле 
\code{CertificateEntry.extensions}.
%
Они описывается типом \code{Extension}, который определен в~\ref{HS.Ext.List}.
%
Расширения в сообщении \token{Certificate} сервера должны соответствовать 
расширениям в \token[HS.CH]{ClientHello},  
а расширения в \token{Certificate} клиента~--- расширениям в 
\token[HS.CR]{CertificateRequest}.
%
Если расширение относится ко всей цепочке сертификатов, то оно должно быть 
указано в первом элементе списка \code{certificate_list}.
%
Настоящий стандарт разрешает использовать только расширение 
\token[HS.Ext.sr]{status_request} (см.~\ref{HS.Ext.sr}). Перечень допустимых 
расширений может дополняться за пределами стандарта.

% skip: and the SignedCertificateTimestamp extension [RFC6962]

Сервер должен отправить сообщение \token{Certificate}, если 
общие ключи были сформированы в режиме DHE, т.~е. без использования 
предварительно согласованных секретов.
%
Список сертификатов сервера должен быть непустым.

Клиент должен отправить сообщение \token{Certificate}, только и только если
сервер запрашивает аутентификацию с помощью сообщения 
\token[HS.CR]{CertificateRequest}.
%
Если сервер запрашивает аутентификацию, а у клиента отсутствует подходящий 
сертификат, то клиент должен отправить сообщение \token{Certificate} 
с пустым списком \code{certificate_list}, т.~е. без сертификатов.

% skip: A Finished message MUST be sent regardless of whether the Certificate 
% message is empty.

Сообщение \token{Certificate} защищается на ключе, построенном по 
\code{client_handshake_traffic_key} или \code{server_handshake_traffic_key}, 
в зависимости от отправителя (см. \ref{CRYPTO.Schedule}).

\subsection{Выбор и проверка сертификатов}\label{HS.Auth.Cert} 

Клиент и сервер должны использовать сертификаты X.509 (СТБ 34.101.19), если 
иной тип не согласован с помощью расширений 
\token[HS.Ext.ct]{client_certificate_type}  
и \token[HS.Ext.ct]{server_certificate_type} 
(см. таблицу~\ref{Table.HS.Exts}).

Открытый ключ сертификата должен быть совместим с одним из алгоритмов ЭЦП, 
указанных противоположной стороной в расширении 
\token[HS.Ext.sa]{signature_algorithms} (см.~\ref{HS.Ext.sa}).

\begin{note*}
Возможна ситуация, когда открытый ключ сертификата используется с одним 
алгоритмом ЭЦП, а сам сертификат подписан другим алгоритмом.
\end{note*}

Дополнительные требования к сертификатам X.509:
%
\begin{itemize}
\item
в сертификат должно быть включено расширение \code{KeyUsage}, и в нем должен 
быть установлен флаг \code{digitalSignature}; 

\item
если противоположная сторона выслала расширение 
\token[HS.Ext.ca]{certificate_authorities}, то в цепочку сертификатов 
\token[HS.CT]{Certificate} следует включить хотя бы один сертификат, который 
удовлетворяет требованиям расширения (см.~\ref{HS.Ext.ca});

\item
если сервер включил в \token[HS.CR]{CertificateRequest} расширение 
\token[HS.Ext.of]{oid_filters}, то сертификат клиента должен удовлетворять 
требованиям расширения (см.~\ref{HS.Ext.of});

\item
если сервер может использовать расширение \token{server_name} 
(см.~\ref{HS.Ext.sn}) при выборе своего сертификата, то клиенту следует 
высылать это расширение.
\end{itemize}

% use: https://www.rfc-editor.org/errata/eid6139

Все сертификаты X.509, включенные в \token[HS.CT]{Certificate}, должны быть 
подписаны с помощью алгоритмов, указанных в расширениях 
\token[HS.Ext.sa]{signature_algorithms} и 
\token[HS.Ext.sa]{signature_algorithms_cert} противоположной стороны. 
%
Требование не распространяется на точки доверия, подписи которых могут не 
проверяться.

Если отправителем \token[HS.CT]{Certificate} является сервер, и он не может 
подготовить цепочку сертификатов, в которой используются только запрошенные 
алгоритмы ЭЦП, то серверу следует продолжить Handshake, составив резервную 
цепочку по своему усмотрению.
%
В резервную цепочку могут входить сертификаты, для алгоритмов проверки подписей 
которых неизвестно, поддерживает ли их клиент или нет.

Если отправителем \token[HS.CT]{Certificate} является клиент и он также не 
может выполнить ограничения на алгоритмы ЭЦП, то он может либо выслать 
резервную цепочку, либо продолжить Handshake анонимно, т.~е. с пустым списком 
сертификатов в \token[HS.CT]{Certificate}.

Если получатель \token[HS.CT]{Certificate} не может по присланному списку 
сертификатов построить подходящую цепочку, то он должен прервать Handshake
с подходящим оповещением (\token[ALERT.Err.uc]{unsupported_certificate} по умолчанию).

Если у отправителя \token[HS.CT]{Certificate} есть несколько подходящих 
сертификатов, то он может выбрать любой из них, руководствуясь изложенными 
правилами и дополнительными критериями.

Цепочка сертификатов X.509 должна проверяться в соответствии с СТБ 34.101.19. 
%
Если проверку выполняет сервер, то в случае ошибки он может по своему 
усмотрению либо продолжить Handshake без аутентификации клиента, либо прервать 
его.

Если сервер выслал в \token[HS.CT]{Certificate} пустой список сертификатов,
то клиент должен прервать Handshake с оповещением 
\token[ALERT.Err.de]{decode_error}.  

Если пустой список сертификатов выслал клиент, то сервер может по своему 
усмотрению либо продолжить Handshake без аутентификации клиента, либо прервать 
его с оповещением \token[ALERT.Err.cr]{certificate_required}. 

\subsection{Сообщение \token{CertificateVerify}}\label{HS.CV} 

Сообщение \token{CertificateVerify} содержит подпись текущей стенограммы 
Handshake. С помощью \token{CertificateVerify} отправитель доказывает, что он
действительно владеет личным ключом, связанным с его сертификатом. Кроме этого, 
подпись в \token{CertificateVerify} подтверждает целостность и подлинность 
стенограммы. 

Формат сообщения \token{CertificateVerify} описывается одноименным типом:

\begin{codeblock}
struct {
  SignatureScheme algorithm;
  opaque signature<0..2^16-1>;
} CertificateVerify;
\end{codeblock}

Поля \code{CertificateVerify} имеют следующее значение:
%
\begin{itemize}
\item 
\code{algorithm}~--- идентификатор алгоритмов ЭЦП. Описывается типом 
\code{SignatureScheme}, который определен в \ref{HS.Ext.sa};
\item 
\code{signature}~--- подпись стенограммы Handshake.
\end{itemize}

Подпись вырабатывается по правилам, заданным в~\ref{CRYPTO.Auth}.
При выработке подписи используется личный ключ отправителя сообщения.

Сервер должен отправлять \token{CertificateVerify} при аутентификации по 
сертификату, после сообщения \token[HS.CT]{Certificate}.
%
Клиент должен отправлять \token{CertificateVerify} при аутентификации по 
сертификату, если сертификат указан в \token[HS.CT]{Certificate} (список 
сертификатов непуст).

Сообщение \token{CertificateVerify} должно отправляться непосредственно после 
\token[HS.CT]{Certificate} и непосредственно перед \token[HS.F]{Finished}.

Идентификатор алгоритмов ЭЦП в сообщении \token{CertificateVerify} сервера 
должен входить в список, указанный клиентом в расширении 
\token[HS.Ext.sa]{signature_algorithms} сообщения \token[HS.CH]{ClientHello}. 
%
% skip: unless no valid certificate chain can be produced without unsupported 
% algorithms (see Section 4.2.3).
%
Идентификатор алгоритмов ЭЦП в \token{CertificateVerify} клиента 
должен входить в список, указанный сервером в расширении 
\token[HS.Ext.sa]{signature_algorithms} сообщения 
\token[HS.CR]{CertificateRequest}.  
%
Открытый ключ в сертификате отправителя \token{CertificateVerify} должен 
соответствовать алгоритмам ЭЦП. 

Получатель \token{CertificateVerify} должен проверить подпись в поле
\code{signature}. При проверке должен использоваться открытый ключ из
сертификата отправителя, присланного в сообщении \token[HS.CT]{Certificate}.
%
Если проверка подписи завершена с ошибкой, то получатель должен прервать 
Handshake с оповещением \token[ALERT.Err.de2]{decrypt_error}.

Сообщение \token{CertificateVerify} защищается на ключе, построенном по 
\code{client_handshake_traffic_key} или \code{server_handshake_traffic_key}, 
в зависимости от отправителя.

\subsection{Сообщение \token{Finished}}\label{HS.F} 

В сообщении \token{Finished} пересылается имитовставка текущей стенограммы 
Handshake. Имитовставка вычисляется на одном из общих ключей. Ее корректность 
подтверждает знание общих ключей, целостность и подлинность стенограммы.

Формат сообщения \token{Finished} определяется одноименным типом:

\begin{codeblock}
struct {
  opaque verify_data[hash_length];
} Finished;
\end{codeblock}

В поле \code{verify_data} указывается имитовставка стенограммы. 
Имитовставка вычисляется по правилам, заданным в~\ref{CRYPTO.Auth}.

Получатель \token{Finished} должен проверить имитовставку.
В случае ошибки получатель должен прервать Handshake с 
оповещением \token[ALERT.Err.de2]{decrypt_error}. 

После того, как сервер отправил свое сообщение \token{Finished}, получил и 
проверил ответное \token{Finished}, он может начать отправку прикладных данных. 
%
Сервер может отправлять прикладные данные даже раньше, не дожидаясь \token{Finished} 
клиента. Однако при этом в момент отправки у сервера нет гарантий, что клиент 
действительно выполняет протокол, а не просто выслал 
\token[HS.CH]{ClientHello}.
%
Клиент может отправлять прикладные данные, не дожидаясь \token{Finished} 
сервера, в рамках механизма 0-RTT.

Сообщение \token{Finished} защищается на ключе, построенном по  
\code{client_handshake_traffic_key} или \code{server_handshake_traffic_key}, в 
зависимости от отправителя.

После \token{Finished} отправляемые данные должны защищаться на ключе,
построенном по \code{client_application_traffic_key} или 
\code{server_application_traffic_key}.
%
В частности, оповещения Alert, которые могут появиться при обработке сообщений 
\token[HS.CT]{Certificate} и \token[HS.CV]{CertificateVerify} клиента,
сервер защищает на ключе, построенном по \code{server_application_traffic_key}.
