\section{Другие ключи и секреты}\label{CRYPTO.Other}

\subsection{Экспортеры}\label{CRYPTO.Exp}

По завершении TLS стороны могут продолжить использовать общий ключевой материал 
с помощью экспортеров. Экспортер~--- это ключ, который строится по специальному 
общему ключу, согласованному сторонами в Handshake. Экспортер характеризуется 
меткой и строится с учетом определенного контекста. Изменяя метку или контекст, 
можно получить семейство экспортеров. При раскрытии одного из представителей 
семейства сохраняется секретность как исходного ключа, так и других 
представителей.

Алгоритм построения экспортера принимает на вход ключ~$K$, метку 
$L\in\{0,1\}^{8*}$, контекст~$C\in\{0,1\}^{8*}$ и натуральное~$m$~---
длину экспортера в байтах. Алгоритм возвращает экспортер~$Y\in\{0,1\}^{8m}$. 
%
В качестве~$K$ должен использоваться один из
ключей~\code{early_exporter_master_secret} или~\code{exporter_master_secret}
(см. таблицу~\ref{Table.CRYPTO.Schedule}). 

Шаги алгоритма:
\begin{enumerate}
\item
Установить $H\gets\code{Hash}(C)$.
\item
Установить $R\gets\code{Derive-Secret}(K,L,\perp)$.
\item
Установить $Y\gets\code{HKDF-Expand-Label}(R,\str{exporter}, H, m)$.
\item
Возвратить $Y$.
\end{enumerate}

При вызовах \code{Derive-Secret} и \code{HKDF-Expand-Label} должны соблюдаться 
ограничения этих алгоритмов (см.~\ref{CRYPTO.HKDF.IFace}). В частности длина 
метки~$L$ в байтах не должна превышать~249. Дополнительные ограничения на  
метки экспортеров установлены в~\cite{RFC5705}.

При выборе исходного ключа~$K$ стороны должны отдавать предпочтение 
\code{exporter_master_secret}, если иное не оговорено явно.
%
Ключ~\code{early_exporter_master_secret} введен для того, чтобы иметь 
возможность строить экспортеры на ранней стадии (в механизме 0-RTT), 
когда \code{exporter_master_secret} еще не сформирован.
%
В реализациях TLS рекомендуется выносить построение ранних экспортеров
в отдельный сервис.

\subsection{Секреты для возобновления связи}\label{CRYPTO.Resume}

Секрет для возобновления связи строится по ключу 
\code{resumption_master_secret} и синхропосылке \code{ticket_nonce},  
включенной в сообщение \token{NewSessionTicket} (см.~\ref{HS.NST}).

Секретом является следующее значение:
%
\begin{align*}
\code{HKDF-Expand-Label}(&\code{resumption_master_secret},\str{resumption},\\
&\code{ticket_nonce}, \code{hash_length}).
\end{align*}

% skip: Because the ticket_nonce value is distinct for each NewSessionTicket 
% message, a different PSK will be derived for each ticket.
