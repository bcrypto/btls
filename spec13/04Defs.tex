\chapter{Сокращения, соглашения и обозначения}\label{DEFS}

В настоящем стандарте применяют следующие сокращения:

АСН.1~--- абстрактно-синтаксическая нотация версии 1 (ГОСТ 34.973);

УЦ~--- удостоверяющий центр (СТБ 34.101.19, СТБ 34.101.78);

ЭЦП~---~электронная цифровая подпись (СТБ 34.101.45);

DER (distinguished encoding rules)~--- отличительные правила кодирования АСН.1
(СТБ 34.101.19 [приложение Б]);

DHE (Diffie-Hellman ephemeral)~--- протокол Диффи~--- Хеллмана с одноразовыми 
ключами;

OCSP (online certificate status protocol)~--- онлайновый протокол проверки 
статуса сертификата (СТБ 34.101.26);

RTT (round-trip time)~--- цикл приема-передачи данных в компьютерных сетях.

В настоящем стандарте для описания форматов данных используются соглашения 
и обозначения, определенные в приложении~\ref{SYNTAX}. Применяемый 
синтаксис основан на языке программирования Си~\cite{ISO9899} и правилах, 
заданных в~\cite{RFC4506}.

Для описания преобразований двоичных слов используются обозначения, введенные в 
СТБ 34.101.31.
%
Октеты двоичного слова могут кодироваться графическими символами базовой таблицы 
КОИ-7 (ASCII), определенной в ГОСТ~27463 и представленной в СТБ 34.101.47. 
Полученная в результате кодирования строка символов окаймляется двойными 
кавычками.  
%
Например, строка \str{tls13 } является кодовым представлением двоичного 
слова $\hex{746C73313320}$.
%
Двойные кавычки внутри строк не допускаются.
%
Пустая строка \str{} соответствует пустому слову~$\perp$.

В настоящем стандарте термины <<октет>> и <<байт>> взаимозаменяемы.

