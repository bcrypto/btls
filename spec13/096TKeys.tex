\section{Ключи защиты записей}\label{CRYPTO.TKeys}

Записи протокола Record защищаются с помощью алгоритма аутентифицированного 
шифрования, заданного в действующем криптонаборе.
%
Для защиты записей, отправляемых от клиента серверу, используются
ключ \code{client_write_key} и начальная синхропосылка \code{client_write_iv}. 
Ключ используется напрямую, а \code{client_write_iv} складывается с порядковым 
номером записи для получения актуальной синхропосылки (см.~\ref{RECORD.Nonce}).

Пары $(\code{client_write_key},\code{client_write_iv})$ строятся по каждому из 
ключей таблицы~~\ref{Table.CRYPTO.Schedule}, имя которого начинается с 
\code{client}. Назначение исходного ключа определяет, для защиты записей какого 
типа должна использоваться построенная по нему пара.  

Построение выполняется следующим образом:
\begin{align*}
\code{client_write_key}&\gets
\code{HKDF-Expand-Label}(K, \str{key}, \perp, \code{key_length}),\\
\code{client_write_iv}&\gets
\code{HKDF-Expand-Label}(K, \str{iv}, \perp, \code{iv_length}).
\end{align*}
Здесь $K$~--- исходный ключ (один из ключей \code{client_early_traffic_secret}, 
\code{client_handshake_traffic_secret} или \code{client_application_traffic_secret}), 
\code{key_length}~--- длина в байтах ключа алгоритмов аутентифицированного 
шифрования, \code{iv_length}~--- длина в байтах синхропосылки.

Ключ и начальная синхропосылка пересчитываются всякий раз, когда исходный 
ключ~$K$ обновляется (см.~\ref{CRYPTO.Upd}).

Введенные правила распространяются на ключи и начальные синхропосылки, 
которые используются для защиты записей, отправляемых от сервера клиенту.
Изменяются только имена объектов: префикс \code{server} вместо префикса 
\code{client}.
%
В частности, 
\begin{align*}
\code{server_write_key}&\gets
\code{HKDF-Expand-Label}(K, \str{key}, \perp, \code{key_length}),\\
\code{server_write_iv}&\gets
\code{HKDF-Expand-Label}(K, \str{iv}, \perp, \code{iv_length}),
\end{align*}
где~$K$~--- один из ключей \code{server_handshake_traffic_secret} или 
\code{server_application_traffic_secret}.
