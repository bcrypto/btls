\section{Параметры сервера}\label{HS.Params} 

\subsection{Сообщение \token{EncryptedExtensions}}\label{HS.EE} 

В сообщении \token{EncryptedExtensions} сервер пересылает защищенные расширения.

Формат сообщения описывается одноименным типом:
%
\begin{codeblock}
struct {
  Extension extensions<0..2^16-1>;
} EncryptedExtensions;
\end{codeblock}

В поле \code{extensions} указывается список расширений. Допустимые расширения 
перечислены в таблице~\ref{Table.HS.Exts}.

Сервер должен выслать \token{EncryptedExtensions} сразу после своего сообщения 
\token[HS.SH]{Finished}.

Сообщение \token{EncryptedExtensions} защищается на ключе, построенном по 
\code{server_handshake_traffic_secret} (см.~\ref{CRYPTO.Schedule}). Это первое 
сообщение, которое защищается на данном ключе.

Клиент должен проверить, что \token{EncryptedExtensions} содержит только 
допустимые расширения. При нарушении условия клиент должен прервать Handshake
с оповещением \token[ALERT.Err.ip]{illegal_parameter}.

\subsection{Сообщение \token{CertificateRequest}}\label{HS.CR} 

С помощью сообщения \token{CertificateRequest} сервер запрашивает сертификат 
клиента для его аутентификации. Сервер может выслать 
\token{CertificateRequest}, только если сам аутентифицируется по сертификату.

Формат сообщения \token{CertificateRequest} описывается одноименным типом:
%
\begin{codeblock}
struct {
  opaque certificate_request_context<0..2^8-1>;
  Extension extensions<0..2^16-1>;
 } CertificateRequest;
\end{codeblock}

% use: https://www.rfc-editor.org/errata/eid5682

Поля \code{CertificateRequest} имеют следующее значение:

\begin{itemize}
\item
\code{certificate_request_context}~--- строка, которая 
идентифицирует запрос сертификата и которая будет дублироваться в сообщении  
\token[HS.CT]{Certificate} клиента.
%
Если в соединении высылается несколько сообщений \token{CertificateRequest},
то их строки \code{certificate_request_context} не должны повторяться.
%
Если \token{CertificateRequest} высылается в Handshake, а не по его завершении,
то строка должна иметь нулевую длину.
%
При отправке \token{CertificateRequest} после Handshake серверу следует выбирать 
строки так, чтобы клиент не мог их предсказать (например, генерируя случайным 
образом);

% skip: ...in order to prevent an attacker who has temporary access to the 
% client's private key from pre-computing valid CertificateVerify messages.

\item
\code{extensions}~--- перечень расширений, которые описывают параметры 
запрашиваемого сертификата. Перечень должен содержать расширение
\token[HS.Ext.sa]{signature_algorithms}, может содержать другие расширения.
%
Клиент должен игнорировать нераспознанные расширения.
\end{itemize}

Сообщение \token{CertificateRequest} не является обязательным. Если сервер 
высылает \token{CertificateRequest}, то он должен это сделать сразу после 
\token[HS.EE]{EncryptedExtensions}.

% skip: In prior versions of TLS, the CertificateRequest message carried a list 
% of signature algorithms and certificate authorities which the server would 
% accept. In TLS 1.3, the former is expressed by sending the 
% "signature_algorithms" and optionally "signature_algorithms_cert" extensions. 
% The latter is expressed by sending the "certificate_authorities" extension 
% (see Section 4.2.4).

Если сервер аутентифицируется с помощью PSK, согласованного в предыдущем 
соединении, то он не должен отправлять \token{CertificateRequest} в протоколе 
Handshake, хотя может это сделать по завершении протокола
(см.~\ref{HS.Post.Auth}) при условии, что клиент выслал в
\token[HS.CH]{ClientHello} расширение \token[HS.Ext.pha]{post_handshake_auth}
(см.~\ref{HS.Ext.pha}).
%
Если сервер аутентифицируется с помощью PSK, согласованного за пределами TLS, 
то он не должен отправлять \token{CertificateRequest} ни в протоколе Handshake, 
ни по его завершении.

% use: https://www.rfc-editor.org/errata/eid6205
% skip: Future specifications MAY provide an extension to permit this. 

% info: https://mailarchive.ietf.org/arch/msg/tls/TugB5ddJu3nYg7chcyeIyUqWSbA/
% * Suppose a client Alice performs an initial handshake with Charlie. Charlie,
%   masquerading as Alice, subsequently performs a handshake with Bob. 
%   Following a PSK resumption, Bob requests authentication from Charlie 
%   (impersonating Alice). Charlie then requests authentication from Alice, and 
%   the returned signature will also be a valid signature for the session with 
%   Bob. 
% info: https://eprint.iacr.org/2016/711.pdf

Сообщение \token{CertificateRequest} защищается на ключе, построенном по 
\code{server_handshake_traffic_secret} (см.~\ref{CRYPTO.Schedule}). 
