\section{Обмен ключами}\label{HS.KeyExchange}

\subsection{Согласование параметров}\label{HS.Negot} 

На этапе обмена ключами стороны согласуют параметры защиты.
%
Согласование начинает клиент, высылая в своем сообщении 
\token[HS.CH]{ClientHello} устраивающие его параметры.

Имеется четыре перечня параметров:
\begin{enumerate}
\item
Перечень поддерживаемых криптонаборов.

\item
Перечень поддерживаемых циклических групп для протокола Диффи~--- Хеллмана.
Задается в расширении \token{supported_groups} (см.~\ref{HS.Ext.sg}). 
Сопровождается расширением \token{key_share} (см.~\ref{HS.Ext.ks}), в котором 
клиент указывает открытые ключи для некоторых или всех перечисленных групп.

\item
Перечень поддерживаемых алгоритмов ЭЦП. Задается в расширении 
\token{signature_algorithms} (см.~\ref{HS.Ext.sa}).
%
Может сопровождаться расширением \token[HS.Ext.sa]{signature_algorithms_cert} 
с перечнем поддерживаемых алгоритмов проверки подписей сертификатов.

\item
Список идентификаторов предварительно согласованных секретов.
Задается в расширении \token{pre_shared_key} (см.~\ref{HS.Ext.psk}).
%
Сопровождается расширением \token{psk_key_exchange_modes} 
(см.~\ref{HS.Ext.pkem}), где клиент перечисляет допустимые режимы формирования  
общих ключей с участием секретов.
\end{enumerate}

Если сервер решает не использовать предварительно согласованные секреты, то он 
может делать выбор в первых трех перечнях независимо друг от друга. А именно, 
сервер выбирает подходящие криптонабор, циклическую группу, алгоритмы ЭЦП. 
Одновременно с выбором алгоритмов ЭЦП сервер выбирает сертификат для 
аутентификации.
%
Если группы, предложенные клиентом в \token[HS.Ext.sg]{supported_groups}, не 
устраивают сервер, то он должен закрыть соединение с оповещением 
\token[ALERT.Err.hf]{handshake_failure} или 
\token[ALERT.Err.is]{insufficient_security}.

Если сервер решает использовать предварительно согласованный секрет, то он 
кроме всего прочего должен выбрать режим формирования общих ключей из 
списка, предложенного клиентом в \token[HS.Ext.pkem]{psk_key_exchange_modes}. 
Если сервер выбирает режим PSK, то отсутствие подходящих групп в 
\token[HS.Ext.sg]{supported_groups} не является критическим.

Если сервер выбрал группу, для которой клиент не предъявил в 
\token[HS.Ext.ks]{key_share} открытый ключ, то сервер должен выслать сообщение 
\token[HS.HRR]{HelloRetryRequest}.

Если сервер выбрал параметры, не отправляя \token[HS.HRR]{HelloRetryRequest}, 
то он указывает их в ответном сообщении \token[HS.SH]{ServerHello}.
%
Сообщение формируется по следующим правилам:

\begin{itemize}
\item
если сервер решил использовать предварительно согласованный секрет, то он 
включает в \token[HS.SH]{ServerHello} расширение 
\token[HS.Ext.psk]{pre_shared_key} с номером секрета в  
списке клиента;

\item
если сервер решил использовать протокол Диффи~--- Хеллмана, то он включает в 
\token[HS.SH]{ServerHello} расширение \token[HS.Ext.ks]{key_share}
с выбранной циклической группой и открытым ключом для нее. 
\end{itemize}

Если сервер решил не использовать PSK, то он должен включить 
\token[HS.Ext.ks]{key_share} в \token[HS.SH]{ServerHello}
и впоследствии должен аутентифицироваться по сертификату, высылая 
сообщения \token[HS.CT]{Ceritificate} и \token[HS.CV]{CertificateVerify} 
(см.~\ref{HS.Auth}).
%
Если сервер решил использовать PSK, то он не должен аутентифицироваться 
по сертификату.

Если сервер не может поддержать предложенные клиентом параметры, то он должен 
прервать Handshake с оповещением \token[ALERT.Err.hf]{handshake_failure} или 
\token[ALERT.Err.is]{insufficient_security}. 

\subsection{Сообщение \code{ClientHello}}\label{HS.CH} 

Сообщение \code{ClientHello} является первым сообщением клиента,
с его отправки начинается выполнение TLS.

Формат сообщения \code{ClientHello} описывается одноименным типом:
%
\begin{codeblock}
uint16 ProtocolVersion;
opaque Random[32];

uint8 CipherSuite[2];

struct {
  ProtocolVersion legacy_version = 0x0303;  /* TLS v1.2 */
  Random random;
  opaque legacy_session_id<0..32>;
  CipherSuite cipher_suites<2..2^16-2>;
  opaque legacy_compression_methods<1..2^8-1>;
  Extension extensions<8..2^16-1>;
} ClientHello;
\end{codeblock}

% use: https://www.rfc-editor.org/errata/eid8423

Поля \code{ClientHello} имеют следующее значение:
%
\begin{itemize}
\item
\code{legacy_version}~--- устаревшее поле для номера версии TLS. По соображениям
совместимости в поле должно быть установлено значение \code{0x0303}, которое
представляет версию 1.2. Согласование версии выполняется с помощью расширения
\token{supported_versions} (см.~\ref{HS.Ext.sv}).
%
Сервер должен проверить, что в поле установлено значение \code{0x0303}.
При нарушении условия сервер должен закрыть соединение с оповещением 
\token[ALERT.Err.pv]{protocol_version};

% info: https://datatracker.ietf.org/doc/html/rfc8996
   
\item
\code{random}~--- случайные данные клиента; 

% skip: generated by a secure random number generator. 
% TLS requires a cryptographically secure pseudorandom number generator 
% (CSPRNG) in most cases. 

% todo: pseudorandom, требования СТБ 34.101.27?

\item
\code{legacy_session_id}~--- устаревшее поле для идентификатора сеанса TLS. 
В TLS 1.2 поле использовалось для возобновления связи.  
%
Если клиент получил идентификатор сеанса от сервера TLS 1.2 и желает 
возобновить с сервером связь, то он должен указать идентификатор в поле 
\code{legacy_session_id}. 
%
В остальных случаях поле должны быть пустым (иметь нулевую длину);

% skip: In compatibility mode (see Appendix D.4), this field MUST be non-empty, 
% so a client not offering a pre-TLS 1.3 session MUST generate a new 32-byte 
% value. This value need not be random but SHOULD be unpredictable to avoid 
% implementations fixating on a specific value (also known as ossification). 

\item
\code{cipher_suites}~--- перечень криптонаборов, которые поддерживает клиент, 
в порядке убывания предпочтения. 
%
Криптонабор описывает алгоритмы аутентифицированного шифрования для защиты
записей протокола Record и алгоритм хэширования для построения ключей и других
криптографических операций.
%
Предварительно согласованные секреты связываются с алгоритмами хэширования.

Если клиент в расширении \token[HS.Ext.psk]{pre_shared_key} предлагает
использовать предварительно согласованный секрет, то ему следует включить в
перечень хотя бы один криптонабор, алгоритм хэширования которого ассоциирован с
секретом.
%
Сервер должен игнорировать криптонаборы, которые он не распознает, не
поддерживает или не желает использовать, и обрабатывать оставшиеся криптонаборы
без учета игнорирования;

\item
\code{legacy_compression_methods}~--- устаревшее поле со списком алгоритмов
сжатия, которые поддерживает клиент. Список должен быть задан одним байтом, 
и этот байт должен быть нулевым. При нарушении данного условия сервер должен 
закрыть соединение с оповещением \token[ALERT.Err.ip]{illegal_parameter};

% skip: Note that TLS 1.3 servers might receive TLS 1.2 or prior ClientHellos 
% which contain other compression methods and (if negotiating such a prior version) 
% MUST follow the procedures for the appropriate prior version of TLS.
 
\item
\code{extensions}~--- перечень расширений. Тип \code{Extension} определен 
в~\ref{HS.Ext.List}. Имеются обязательные расширения. В частности, в перечень 
должно входить расширение \token[HS.Ext.sv]{supported_versions}.
%
Сервер должен игнорировать нераспознанные расширения.

% skip: All versions of TLS allow an extensions field to optionally follow the  
% compression_methods field. TLS 1.3 ClientHello messages always contain  
% extensions (minimally "supported_versions", otherwise, they will be 
% interpreted as TLS 1.2 ClientHello messages). However, TLS 1.3 servers might 
% receive ClientHello messages without an extensions field from prior versions 
% of TLS. The presence of extensions can be detected by determining whether 
% there are bytes following the compression_methods field at the end of the 
% ClientHello. Note that this method of detecting optional data differs from 
% the normal TLS method of having a variable-length field, but it is used for 
% compatibility with TLS before extensions were defined. TLS 1.3 servers will 
% need to perform this check first and only attempt to negotiate TLS 1.3 if the 
% "supported_versions" extension is present. If negotiating a version of TLS 
% prior to 1.3, a server MUST check that the message either contains no data 
% after legacy_compression_methods or that it contains a valid extensions block 
% with no data following. If not, then it MUST abort the handshake with a 
% "decode_error" alert.

\end{itemize}

Клиент должен отправить \code{ClientHello} при подключении к серверу.
%
Клиент повторно отправляет \code{ClientHello}, если в ответ на первое сообщение 
он получил от сервера сообщение \token[HS.HRR]{HelloRetryRequest}. 
%
Повторное сообщение \code{ClientHello} должно быть получено из первого 
следующими корректировками.
%        
\begin{enumerate}
\item
Исправление расширения \token[HS.Ext.ks]{key_share}, если оно включено
в \token[HS.HRR]{HelloRetryRequest}. Исправленное расширение должно 
включать единственную запись \code{KeyShareEntry}, и в этой записи должен быть  
указан открытый ключ для выбранной сервером циклической группы.

\item
Исключение расширения \token{early_data} (см.~\ref{HS.Ext.ed}), если оно
включено. При перезапуске Handshake через \token[HS.HRR]{HelloRetryRequest}
защита ранних прикладных данных запрещена.

\item
Включение расширения \token{cookie} (см.~\ref{HS.Ext.c}), если оно включено в 
\token[HS.HRR]{HelloRetryRequest}.

\item
Обновление расширения \token{pre_shared_key} (см.~\ref{HS.Ext.psk}), если оно 
включено. В расширении пересчитываются поля \code{obfuscated_ticket_age} и 
PSK-скрепки. Могут исключаться идентификаторы PSK, алгоритмы хэширования которых не 
соответствует криптонабору, выбранному сервером.

\item
Включение или исключение расширения \token[HS.Ext.p]{padding} 
(см. таблицу~\ref{Table.HS.Exts}), а также изменение его размера.

\item
Корректировка в соответствии с расширением, которое определяется за пределами 
настоящего стандарта и которое включается в \token[HS.HRR]{HelloRetryRequest}.
\end{enumerate}

Другие корректировки \code{ClientHello} не допускаются.

Если сервер получает \code{ClientHello} в неоговоренные моменты времени, 
то он должен прервать Handshake с оповещением 
\token[ALERT.Err.um]{unexpected_message}.

Если сервер установил соединение TLS версии 1.2 и в рамках \doubt{переустановки} 
связи получает \token{ClientHello} версии 1.3, то сервер должен сохранить 
версию 1.2 соединения и \doubt{не переходить на версию 1.3}.

% todo: разобраться с "and receives a TLS 1.3 ServerHello" и renegotiation.

Если клиент запрашивает с помощью какого-либо расширения дополнительные 
функциональные возможности, а сервер эти возможности не предоставляет, то 
клиент может прервать Handshake.

После отправки \code{ClientHello} клиент ожидает сообщения
\token[HS.SH]{ServerHello} или \token[HS.HRR]{HelloRetryRequest}. Если
используется механизм 0-RTT, то в процессе ожидания клиент может передавать
ранние прикладные данные.

\subsection{Сообщение \code{ServerHello}}\label{HS.SH} 

Сервер отправляет сообщение \code{ServerHello}, если присланное 
\token[HS.CH]{ClientHello} устраивает сервер, позволяет ему
согласовать параметры и сформировать общие ключи.

Формат сообщения \code{ServerHello} описывается одноименным типом:
%
\begin{codeblock}
struct {
  ProtocolVersion legacy_version = 0x0303;    /* TLS v1.2 */
  Random random;
  opaque legacy_session_id_echo<0..32>;
  CipherSuite cipher_suite;
  uint8 legacy_compression_method = 0;
  Extension extensions<6..2^16-1>;
} ServerHello;
\end{codeblock}

Поля \code{ServerHello} имеют следующее значение:

\begin{itemize}
\item
\code{legacy_version}~---  устаревшее поле для номера версии TLS. В поле должно 
быть установлено значение \code{0x0303}.
%
Клиент должен проверить данное условие и закрыть соединение с оповещением 
\token[ALERT.Err.ip]{illegal_parameter} при его нарушении; 

% info: https://datatracker.ietf.org/doc/html/rfc8996

\item
\code{random}~--- случайные данные сервера. Должны генерироваться независимо от 
случайных данных клиента в \code{ClientHello.random}.
%
При выборе TLS версии 1.2 последние 8 байтов \code{random} должны быть 
заданы специальным образом, как это объясняется ниже;

\item
\code{legacy_session_id_echo}~--- повтор поля \code{legacy_session_id} из 
сообщения \token[HS.CH]{ClientHello} клиента.
%
% skip: Note that this field is echoed even if the client's value corresponded 
% to a cached pre-TLS 1.3 session which the server has chosen not to resume.
%
Клиент должен прервать Handshake с оповещением 
\token[ALERT.Err.ip]{illegal_parameter}, если \code{legacy_session_id_echo} 
отличается от \code{legacy_session_id};

\item
\code{cipher_suite}~--- криптонабор, выбранный сервером из перечня 
\code{ClientHello.cipher_suites}. 
%
Клиент должен прервать Handshake с оповещением 
\token[ALERT.Err.ip]{illegal_parameter}, если выбранный сервером криптонабор не 
был указан в перечне клиента;

\item
\code{legacy_compression_method}~--- список алгоритмов сжатия.
Должен быть задан одним байтом и этот байт должен быть нулевым. При 
нарушении данного условия клиент должен прервать Handshake с оповещением 
\token[ALERT.Err.ip]{illegal_parameter};

\item
\code{extensions}~--- перечень расширений. Должен содержать только те
расширения, которые нужны для согласования параметров и формирования общих
ключей. Другие расширения могут передаваться в сообщении
\token[HS.EE]{EncryptedExtensions}. Перечень должен включать расширение
\token[HS.Ext.sg]{supported_groups}, а также одно или оба расширения
\token[HS.Ext.psk]{pre_shared_key} и \token[HS.Ext.ks]{key_share}
(см.~\ref{HS.Negot}). 
% add:
В перечень не должны входить расширения, которые не были включены
в~\token[HS.CH]{ClientHello}.
\end{itemize}

Для обратной совместимости с промежуточными сетевыми устройствами (middleboxes)
сообщение \token[HS.HRR]{HelloRetryRequest} имеет такой же формат, как 
\code{ServerHello}. Единственное отличие состоит в том, что в поле \code{random}
сообщения \token[HS.HRR]{HelloRetryRequest} всегда устанавливается значение
$$
\hex{CF21AD74E59A6111BE1D8C021E65B891C2A211167ABB8C5E079E09E2C8A8339C}.
$$

\begin{note*}
Получено применением алгоритма хэширования SHA-256 к строке \str{HelloRetryRequest}.
\end{note*}

При приеме сообщения Handshake с типом \code{server_hello} клиент должен 
отличить \code{ServerHello} от \token[HS.HRR]{HelloRetryRequest} на основании 
поля \code{random}. Если в поле установлено указанное значение, то сообщение 
обрабатывается как \token[HS.HRR]{HelloRetryRequest}.

Если сервер согласовал использование TLS версии 1.2, то для защиты от атак по 
понижению версии он должен указать в последних 8 байтах 
\code{ServerHello.random} следующее значение:
$$
\hex{444F574E47524401}.
$$

% use: https://www.rfc-editor.org/errata/eid7774: Random -> random

% skip: If negotiating TLS 1.1 or below, TLS 1.3 servers MUST, and TLS 1.2 
% servers SHOULD, set the last 8 bytes of their ServerHello.Random value to the 
% bytes: 44 4F 57 4E 47 52 44 00

При согласовании TLS версии 1.2 клиент должен проверить, что последние 8 байтов 
\code{random} отличаются от указанных выше.
%
Если условие не выполнено, т.~е. сервер следует данному стандарту и, таким образом,
поддерживает TLS 1.3, то клиент должен прервать Handshake с оповещением 
\token[ALERT.Err.ip]{illegal_parameter}. 

% skip: This mechanism provides limited protection against downgrade attacks 
% over and above what is provided by the Finished exchange: because the 
% ServerKeyExchange, a message present in TLS 1.2 and below, includes a signature 
% over both random values, it is not possible for an active attacker to modify 
% the random values without detection as long as ephemeral ciphers are used. It 
% does not provide downgrade protection when static RSA is used. 

% skip: This is a change from [RFC 5246], so in practice many TLS 1.2 clients 
% and servers will not behave as specified above.

Клиент TLS, который \doubt{переустанавливает} связь в соответствии с TLS версии 
1.2 и при этом получает \code{ServerHello} \doubt{версии 1.3}, должен прервать 
Handshake с оповещением \token[ALERT.Err.pv]{protocol_version}.

% todo: разобраться с "which receives a TLS 1.3 ServerHello" и renegotiation.

\subsection{Сообщение \code{HelloRetryRequest}}\label{HS.HRR} 

Сервер отправляет сообщение \code{HelloRetryRequest}, если присланное 
\token[HS.CH]{ClientHello} позволяет согласовать параметры и сформировать общие 
ключи, но не содержит всех необходимых для этого сведений.

Сообщение \code{HelloRetryRequest} имеет такой же формат, 
как~\token[HS.SH]{ServerHello}, повторяется семантика полей 
\code{legacy_version}, \code{legacy_session_id_echo}, \code{cipher_suite} и 
\code{legacy_compression_method}.
%
Тем не менее, в настоящем стандарте \code{HelloRetryRequest} считается 
отдельным сообщением. 

Сообщение \code{HelloRetryRequest} должно содержать расширение 
\token{supported_versions} (см.~\ref{HS.Ext.sv}). Серверу следует дополнительно 
включить в \code{HelloRetryRequest} минимальный набор расширений, который 
позволит скорректировать \token[HS.CH]{ClientHello}.
%
Как и для \token[HS.SH]{ServerHello}, сообщение \code{HelloRetryRequest} не
должно содержать расширений, которые не были включены в
\token[HS.CH]{ClientHello}. Единственным исключением является необязательное
расширение \token{cookie} (см.~\ref{HS.Ext.c}).

Кроме \token[HS.Ext.sg]{supported_versions} и \token[HS.Ext.c]{cookie} в 
настоящем стандарте разрешается включать в \code{HelloRetryRequest} только 
расширение \token[HS.Ext.ks]{key_share} (см.~\ref{HS.Ext.ks}).

Получив \code{HelloRetryRequest}, клиент должен проверить поля
\code{legacy_version}, \code{legacy_session_id_echo}, 
\code{cipher_suite} и \code{legacy_compression_method} по правилам,
установленным в~\ref{HS.SH}.
%
После этого клиент должен обработать расширения, начиная с 
\token[HS.Ext.sv]{supported_versions}. 
%
Клиент должен прервать Handshake с оповещением
\token[ALERT.Err.ip]{illegal_parameter}, если учет расширений
\code{HelloRetryRequest} не изменит \token[HS.CH]{ClientHello}.

Если клиент получает второе сообщение \code{HelloRetryRequest} в том же
соединении, т.~e. после своего повторного \token[HS.CH]{ClientHello}, то он
должен прервать рукопожатие с оповещением
\token[ALERT.Err.um]{unexpected_message}.

% use: https://www.rfc-editor.org/errata/eid6136

При успешной обработке \code{HelloRetryRequest} клиент должен отправить повторное сообщение 
\token[HS.CH]{ClientHello}. 

Клиент должен прервать Handshake, если выбранный сервером криптонабор не был 
указан в перечне клиента. 
%
Сервер должен повторить свой выбор в \token[HS.CH]{ServerHello} после 
повторного \token[HS.CH]{ClientHello} (если сервер выбирает криптонабор в 
первую очередь, то это будет происходить автоматически). 
%
Клиент должен проверить, что в \token[HS.SH]{ServerHello} указан то же
криптонабор, что в \code{HelloRetryRequest}. Клиент должен прервать Handshake с 
оповещением \token[ALERT.Err.ip]{illegal_parameter} при нарушении данного
условия.

Клиенту следует исключить из повторного \token[HS.SH]{ClientHello} те предварительно
согласованные секреты, которые не соответствуют алгоритму хэширования 
выбранного криптонабора.

% skip: This allows the client to avoid having to compute partial hash 
% transcripts for multiple hashes in the second ClientHello.

Клиент должен проверить, что версия TLS, указанная сервером в сообщении 
\code{HelloRetryRequest} через расширение 
\token[HS.Ext.sv]{supported_versions}, повторяется в 
\token[HS.SH]{ServerHello}. Клиент должен прервать Handshake с оповещением 
\token[ALERT.Err.ip]{illegal_parameter} при нарушении данного условия.



