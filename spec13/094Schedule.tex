\section{Расписание ключей}\label{CRYPTO.Schedule}

В протоколе Handshake стороны определяют один или несколько общих секретов, по  
которым затем строят подчиненные общие ключи. К общим секретам относятся 
предварительно согласованный секрет PSK и ключ DHE, выработанный по протоколу 
Диффи~--- Хеллмана. Перечень общих секретов определяется выбранным режимом: PSK, 
DHE или PSK+DHE (см.~\ref{COMMON.Components}). Если тот или иной общий секрет не 
формируется, то он полагается равным пустому слову.

При построении подчиненных ключей используются стенограммы Handshake. Поскольку
приветственные сообщения Handshake содержат случайные данные, подчиненные ключи
будут волатильны, даже если общие секреты повторяются, например когда в
нескольких соединениях используется один и тот же секрет PSK, не сопровождаемый
ключом DHE.

Подчиненные ключи строятся в соответствии с расписанием, определенным на 
рисунке~\ref{Fig.CRYPTO.Schedule}. Здесь $\code{Finished}_S$~--- сообщение 
$\token[HS.F]{Finished}$ сервера, $\code{Finished}_C$~--- клиента.

\begin{figure}[hbt]
\begin{align*}
K_1 &\gets\code{HKDF-Extract}(\text{PSK}, \perp),\\
\code{binder_key}&\gets\code{Derive-Secret}(K_1,
  \text{\str{ext binder}/\str{res binder}},\\
  &\hspace{40pt}\perp),\\
%
\code{client_early_traffic_secret}&\gets\code{Derive-Secret}(K_1,
  \str{c e traffic},\\
  &\hspace{40pt}(\code{ClientHello})),\\
%
\code{early_exporter_master_secret}&\gets\code{Derive-Secret}(K_1,
  \str{e exp master},\\
  &\hspace{40pt}(\code{ClientHello})),\\
%
K_2 &\gets\code{Derive-Secret}(K_1, \str{derived}, \perp),\\
K_3 &\gets\code{HKDF-Extract}(\text{DHE}, K_2),\\
%
\code{client_handshake_traffic_secret}&\gets\code{Derive-Secret}(K_3,
  \str{c hs traffic},\\
  &\hspace{40pt}(\code{ClientHello},\dots,\code{ServerHello})),\\
%
\code{server_handshake_traffic_secret}&\gets\code{Derive-Secret}(K_3,
  \str{s hs traffic},\\
  &\hspace{40pt}(\code{ClientHello},\dots,\code{ServerHello})),\\
%
K_4 &\gets\code{Derive-Secret}(K_3, \str{derived}, \perp),\\
K_5 &\gets\code{HKDF-Extract}(\perp, K_4).\\
%
\code{client_application_traffic_secret}&\gets\code{Derive-Secret}(K_5,
  \str{c ap traffic},\\
  &\hspace{40pt}(\code{ClientHello},\dots,\code{Finished}_S)),\\
%
\code{server_application_traffic_secret}&\gets\code{Derive-Secret}(K_5,
  \str{s ap traffic},\\
  &\hspace{40pt}(\code{ClientHello},\dots,\code{Finished}_S)),\\
%
\code{exporter_master_secret}&\gets\code{Derive-Secret}(K_5,
  \str{exp master},\\
  &\hspace{40pt}(\code{ClientHello},\dots,\code{Finished}_S)),\\
%
\code{resumption_master_secret}&\gets\code{Derive-Secret}(K_5,
  \str{res master},\\
  &\hspace{40pt}(\code{ClientHello},\dots,\code{Finished}_C)).\\
\end{align*}
\caption{Расписание ключей}\label{Fig.CRYPTO.Schedule}
\end{figure}

Ключи $K_1,\ldots,K_5$ являются вспомогательными, они не используются за 
пределами расписания. По~$K_1$ (ранний секрет), $K_3$ (секрет Handshake) 
и $K_5$ (главный секрет) строятся ключи,  
представленные в таблице~\ref{Table.CRYPTO.Schedule}. При построении 
явно учитываются стенограммы Handshake. 
%
Ключи таблицы используются для построения подчиненных ключей и других объектов
за пределами расписания. При этом стенограммы продолжают учитываться, но только
опосредованно.

\begin{table}[hbt]
\caption{Назначение ключей}\label{Table.CRYPTO.Schedule}
\begin{tabular}{|l|p{9cm}|}
\hline
Ключ & Подчиненные объекты (строятся по ключу)\\
\hline
\hline
\code|binder_key| & 
PSK-скрепка (см.~\ref{CRYPTO.Binder})\\
%
\hline
\code|client_early_traffic_secret| &
Ключ и начальная синхропосылка ранней защиты (см.~\ref{CRYPTO.TKeys})\\
%
\hline
\code|early_exporter_master_secret| &
Ранний экспортер (см.~\ref{CRYPTO.Exp})\\
%
\hline
\code|client_handshake_traffic_secret| &
Ключ и начальная синхропосылка защиты сообщений Handshake, отправляемых 
клиентом (см.~\ref{CRYPTO.TKeys})\\
%
%
\hline
\code|server_handshake_traffic_secret| &
Ключ и начальная синхропосылка защиты сообщений Handshake, отправляемых 
сервером (см.~\ref{CRYPTO.TKeys})\\
%
\hline
\code|client_application_traffic_secret| &
Ключ и начальная синхропосылка защиты прикладных данных, отправляемых 
клиентом (см.~\ref{CRYPTO.TKeys})\\
%
\hline
\code|server_application_traffic_secret| &
Ключ и начальная синхропосылка защиты прикладных данных, отправляемых 
сервером (см.~\ref{CRYPTO.TKeys})\\
%
\hline
\code|exporter_master_secret| &
Экспортер (см.~\ref{CRYPTO.Exp})\\
%
\hline
\code|resumption_master_secret| &
PSK для возобновления связи (см.~\ref{CRYPTO.Resume})\\
\hline
\end{tabular}
\end{table}

При построении \code{binder_key} должна использоваться строка \str{ext binder}, 
если используется внешний PSK, согласованный за пределами TLS, 
и строка \str{res binder}, если PSK~--- это секрет для возобновления связи,
построенный по \code{resumption_master_secret} в одном из предыдущих 
соединений. 

Клиент вычисляет ключ~$K_1$ для каждого PSK, который он намеревается 
использовать. Это необходимо потому, что по~$K_1$ нужно построить сначала ключ 
\code{binder_key}, а затем PSK-скрепку для включения в 
\token[HS.CH]{ClientHello}. После ответа сервера клиент оставляет только один  
ключ~$K_1$~--- тот, который соответствует PSK, выбранному сервером. Если сервер 
отказался от использования режима PSK, то клиенту придется еще раз 
вычислить~$K_1$, на этот раз для пустого PSK.

Ключ расписания, по которому уже построены все подчиненные ключи, следует 
уничтожить.



