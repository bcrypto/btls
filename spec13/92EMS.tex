\begin{appendix}{Б}{обязательное}{Расширение \code{extended_master_secret}}
\label{EMS} 

\hiddensection{Назначение}\label{EMS.Intro}

В настоящем приложении определяется расширение \code{extended_master_secret},
введенное в~\cite{RFC7627}. Расширение защищает реализации TLS версии 1.2 и 
ниже от атаки <<Triple Handshake>>~\cite{Triple}.
%
Использование расширения \code{extended_master_secret} является дополнительным 
обязательным требованием СТБ 34.101.65.

Протокол TLS версии 1.3 по построению защищен от атаки <<Triple Handshake>>
и в применении расширения \code{extended_master_secret} нет необходимости.
%
Тем не менее, для информирования о защите, реализациям TLS версии 
1.3 и ниже следует сообщать прикладным протоколам об использовании расширения 
в тех случаях, когда установлено соединение TLS~1.3.

Тем не менее расширение может использоваться, чтобы информировать о защите 
прикладные протоколы.

\hiddensection{Вычисление мастер-ключа}\label{TLS12.MS}

Если клиент и сервер согласовали применение расширения
\code{extended_master_secret}, то они должны использовать следующий алгоритм
вычисления мастер-ключа:
%
\begin{enumerate}
\item
Построить стенограмму \code{handshake_messages}, включив в нее сообщения
Handshake, начиная с \code{ClientHello} и заканчивая \code{ClientKeyExchange}.
\item
Определить алгоритм хэширования \code{Hash}, используемый в алгоритме генерации 
псевдослучайных чисел (см. СТБ 34.101.65 [пункт~8.16]), и вычислить с его помощью 
хэш-значение \code{session_hash = Hash(handshake_messages)}
стенограммы \code{handshake_messages}.
\item
Вычислить мастер-ключ
\begin{codeblock}
master_secret = PRF(pre_master_secret, 
  "extended master secret", session_hash)[0..47];
\end{codeblock}
\end{enumerate}

Если применение расширения \code{extended_master_secret} не было согласовано,
то клиент и сервер должны использовать алгоритм вычисления мастер-ключа, 
определенный в СТБ 34.101.65 (пункт~8.16).

\hiddensection{Установка связи}\label{EMS.FullHS}

Действия сторон при установке связи:
\begin{enumerate}
\item
Клиент, поддерживающий  расширение \code{extended_master_secret},
должен отправить его в сообщении \code{ClientHello}.

\item
Сервер, получивший расширение \code{extended_master_secret} и
поддерживающий его, должен включить расширение в сообщение
\code{ServerHello}.

\item
Если сервер получает \code{ClientHello} без расширения 
\code{extended_master_secret}, то ему следует прервать 
Handshake. Если сервер все-таки решает продолжить протокол, 
то он не должен включать расширение в \code{ServerHello}.

\item
Если клиент получает \code{ServerHello} без расширения 
\code{extended_master_secret}, то ему следует прервать Handshake.
\end{enumerate}

\hiddensection{Сокращенная установка связи}\label{EMS.AbbrHS}

Действия сторон при сокращенной установке связи:
\begin{enumerate}
\item
Если в предыдущем сеансе Handshake не использовалось расширение
\code{extended_master_secret}, то клиенту не следует возобновлять такой
сеанс. Вместо этого клиенту следует инициировать полную установку связи.

\item
Если сервер получает \code{ClientHello} с расширением
\code{extended_master_secret}, но расширение не использовалось в предыдущем
сеансе, то сервер должен прервать Handshake.

\item
Если сервер получает \code{ClientHello} без расширения
\code{extended_master_secret}, но расширение использовалось в предыдущем 
сеансе, то сервер должен прервать Handshake.

\item
Если сервер получает \code{ClientHello} без расширения
\code{extended_master_secret} и расширение не использовалось в предыдущем 
сеансе, то серверу следует прервать Handshake.

\item
Если сервер получает \code{ClientHello} с расширением
\code{extended_master_secret} и расширение использовалось в предыдущем 
сеансе, то сервер должен включить расширение в \code{ServerHello}.

\item
Если клиент получает \code{ServerHello} с расширением
\code{extended_master_secret}, но расширение не использовалось в предыдущем 
сеансе, то клиент должен прервать Handshake.

\item
Если клиент получает \code{ServerHello} без расширения
\code{extended_master_secret}, но расширение использовалось в предыдущем 
сеансе, то клиент должен прервать Handshake.
\end{enumerate}

\hiddensection{Идентификатор}\label{EMS.Id}

Расширению \code{extended_master_secret} назначается идентификатор 23. 
Он указывается в поле \code{extension_type} структуры \code{Extension}. 
Соответствующее поле \code{extension_data} должно быть пустым.

\end{appendix}
