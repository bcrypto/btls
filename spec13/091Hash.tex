\section{Алгоритмы \code{Hash} и \code{HMAC}}\label{CRYPTO.Hash}

Стороны TLS используют алгоритм хэширования \code{Hash}, установленный в криптонаборе.
%
Входными данными~\code{Hash} является сообщение~$X\in\{0,1\}^*$, выходными~--- 
хэш-значение~$Y\in\{0,1\}^n$. 

Должны соблюдаться следующие условия:
\begin{itemize}
\item
длина $n$ кратна~$8$ и $n/8\leq 255$;
\item
сообщение~$X$ обрабатывается блоками по~$b$ битов;
\item
длина~$b$ кратна~$8$ и $b\geq n$.
\end{itemize}

Используется алгоритм имитозащиты \code{HMAC}. Алгоритм построен 
по одноименной схеме, установленной в СТБ 34.101.47. В \code{HMAC} в качестве 
базового алгоритма хэширования используется~\code{Hash}.

Входными данными~\code{HMAC} являются ключ~$K\in\{0,1\}^*$ и 
сообщение~$X\in\{0,1\}^*$, выходными~--- имитовставка~$Y\in\{0,1\}^n$.
%
Размерности~$n$ и~$b$ базового алгоритма~\code{Hash} определяют размерности выхода 
и внутренних структур данных~\code{HMAC}.

Алгоритмы~\code{Hash} и~\code{HMAC}, а также  длина их выходов~$n$ используются 
в других алгоритмах настоящего раздела. Число~$n/8$, длина выходов в байтах, 
обозначается через \code{hash_length}.


