\section{Сообщение \token{EndOfEarlyData}}\label{HS.EOED}

Сообщение \token{EndOfEarlyData} информирует о завершении передачи ранних 
прикладных данных в рамках механизма 0-RTT.
   
Формат сообщения \token{EndOfEarlyData} описывается одноименным типом:
%
\begin{codeblock}
struct {} EndOfEarlyData;
\end{codeblock}

Если сервер согласился с использованием механизма 0-RTT, отправив расширение 
\token[HS.Ext.ed]{early_data} в сообщении \token[HS.EE]{EncryptedExtensions}, 
то клиент должен отправить \token{EndOfEarlyData} после приема сообщения
\token[HS.F]{Finished} от сервера. 
%
Если сервер не включил \token[HS.Ext.ed]{early_data} в 
\token[HS.EE]{EncryptedExtensions}, то клиент не должен высылать 
\token{EndOfEarlyData}.

Сообщение \token{EndOfEarlyData} защищается на ключе, построенном по 
\code{client_early_traffic_secret} (см.~\ref{CRYPTO.Schedule}). После отправки 
\token{EndOfEarlyData} клиент защищает свои сообщения на ключе, построенном по 
\code{client_handshake_traffic_secret}.

Сервер не должен отправлять сообщение \token{EndOfEarlyData}.
Если клиент получает данное сообщение, то он должен закрыть соединение
с оповещением \token[ALERT.Err.um]{unexpected_message}. 



