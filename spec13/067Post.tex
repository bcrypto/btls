\section{Сообщения после Handshake}\label{HS.Post} 
        
\subsection{Сообщение \token{NewSessionTicket}}\label{HS.NST} 

В сообщении \token{NewSessionTicket} сервер высылает клиенту билет, который
ссылается на секрет, построенный по общему ключу 
\code{resumption_master_secret} (см.~\ref{CRYPTO.Schedule}).
%
Используя \code{resumption_master_secret} и данные билета, клиент может 
построить секрет самостоятельно и использовать его в будущих соединениях в 
качестве PSK.

В одном соединении сервер может отправить клиенту несколько билетов:
последовательно друг за другом либо после определенных событий.
%
Например, сервер может отправлять билет после аутентификации клиента по 
завершении Handshake, фиксируя факт аутентификации. 
%
Располагая несколькими билетами, клиент может открывать несколько параллельных 
соединений с сервером.

Сервер может отправить сообщение \token{NewSessionTicket} в любой момент после  
получения сообщения \token[HS.F]{Finished} клиента при условии, что 
\token[HS.CH]{ClientHello} содержит расширение 
\token[HS.Ext.pkem]{psk_key_exchange_mode} (см.~\ref{HS.Ext.pkem}).

% use: https://www.rfc-editor.org/errata/eid7003

Сообщение \token{NewSessionTicket} защищается на ключе, построенном по 
\code{server_application_traffic_secret} (см.~\ref{CRYPTO.Schedule}). 

Клиент может использовать PSK, на который ссылается билет в 
\token{NewSessionTicket}, для возобновления связи. Для этого клиент должен 
включить билет в качестве идентификатора PSK в 
расширение~\token[HS.Ext.psk]{pre_shared_key} (см.~\ref{HS.Ext.psk}).
%
Если клиент не поддерживает возобновление связи, то он должен игнорировать 
сообщение \token{NewSessionTicket}.

% use: https://www.rfc-editor.org/errata/eid7250

При возобновлении связи сервер должен принимать билет, только если он был 
построен с помощью алгоритма хэширования текущего криптонабора.

При использовании расширения \token[HS.Ext.sn]{server_name} клиенту
не следует возобновлять связь, если билет был согласован в соединении с именем 
сервера, которое отличается от текущего имени.
%
% skip: The latter is a performance optimization: normally, there is no reason 
% to expect that different servers covered by a single certificate would be 
% able to accept each other's tickets; hence, attempting resumption in that 
% case would waste a single-use ticket.
%
Однако клиент может возобновить связь, если он был проинформирован об изменении 
имени сервера за пределами TLS.
%
Клиент не должен возобновлять связь, если текущее имя сервера не соответствует 
данным в сертификате сервера, который был использован в первоначальном соединении.
%
Если при возобновлении связи требуется сообщить имя сервера прикладному 
протоколу, то следует указать текущее имя, а не имя в первоначальном соединении.
%
% skip: Note that if a server implementation declines all PSK identities with 
% different SNI values, these two values are always the same.

Формат сообщения \token{NewSessionTicket} определяется одноименным типом:
%
\begin{codeblock}
struct {
  uint32 ticket_lifetime;
  uint32 ticket_age_add;
  opaque ticket_nonce<0..255>;
  opaque ticket<1..2^16-1>;
  Extension extensions<0..2^16-2>;
} NewSessionTicket;
\end{codeblock}

Поля \code{NewSessionTicket} имеют следующее значение:

\begin{itemize}
\item
\code{ticket_lifetime}~-- срок действия билета в секундах, начиная с момента 
его выпуска.
%
% skip:  as a 32-bit unsigned integer in network byte order
% * Задается 32-битовым беззнаковым целым, в котором старшие байты идут  
%   первыми (сетевой порядок байтов, big-endian).
%
Срок действия не должен быть больше 604800 секунд (7 суток). Нулевое значение
\code{ticket_lifetime} означает, что билет следует уничтожить немедленно 
\doubt{после использования}.
%
% use: https://mailarchive.ietf.org/arch/msg/tls/5hBdR59T3U9qSFJlCANqFObFPQs/
% 
Клиент не должен использовать билет по прошествии 7 суток после его выпуска,
независимо от значения, установленного в \code{ticket_lifetime}.
%
Срок действия билета может сокращаться относительно \code{ticket_lifetime}
в соответствии с локальной политикой пользователя.
%
Сервер может аннулировать билет до окончания срока его действия; 

% use: https://www.rfc-editor.org/errata/eid6142

\item
\code{ticket_age_add}~--- случайное 32-битовое число, которое используется для 
зашумления времени жизни билета. Клиент получает зашумленное время жизни, 
складывая актуальное время в миллисекундах с \code{ticket_age_add} по 
модулю~$2^{32}$. Клиент высылает зашумленное время в расширении 
\token[HS.Ext.psk]{pre_shared_key} (см.~\ref{HS.Ext.psk}). При формировании  
очередного билета сервер должен генерировать \code{ticket_age_add} заново;

\item
\code{ticket_nonce}~--- синхропосылка. Синхропосылки текущего соединения не 
должны повторяться;

\item
\code{ticket}~--- билет. Выступает в роли идентификатор ассоциированного PSK. 
Билетом может быть ключ базы данных или (аутентифицированный) шифртекст, 
который проверяется при получении;

% info: https://mailarchive.ietf.org/arch/msg/tls/2TGQGMbdjArgmG41gnb9LJ3W_TU/
% info: https://datatracker.ietf.org/doc/html/rfc5077#section-4

\item
\code{extensions}~--- перечень расширений. Клиент должен игнорировать 
нераспознанные расширения.
\end{itemize}

Настоящий стандарт разрешает включать в перечень
\code{NewSessionTicket.extensions} только расширение 
\token[HS.Ext.ed]{early_data} (см.~\ref{HS.Ext.ed}). Включая расширение, сервер 
объявляет, что секрет, на который ссылается выпущенный билет, может 
использоваться в механизме \mbox{0-RTT}.

В поле \code{max_early_data_size} расширения \token[HS.Ext.ed]{early_data} 
сервер указывает, сколько байтов ранних прикладных данных разрешается передать 
клиенту в рамках механизма \mbox{0-RTT}. 
%
Если сервер получает более \code{max_early_data_size} байтов ранних прикладных
данных, то ему следует закрыть соединение с оповещением
\token[ALERT.Err.um]{unexpected_message}.

Лимит \code{max_early_data_size} не учитывает однобайтовый тип содержимого и 
незначащие нулевые байты в \code{TLSInnerPlaintext} (см.~\ref{RECORD.Enc}).
%
Сервер может проверять соблюдение лимита, не снимая с данных защиту, и поэтому 
клиенту не следует дополнять записи Record большим числом незначащих нулей.

Секрет, связанный с билетом, вычисляется по правилам, установленным 
в~\ref{CRYPTO.Resume}.

\begin{note}
Если сервер не запрашивает аутентификацию клиента по сертификату, то он может 
выслать билет в~\token{NewSessionTicket} сразу после своего сообщения 
\token[HS.F]{Finished}.
%
Для этого сервер самостоятельно вычисляет сообщение 
\token[HS.F]{Finished} клиента, определяет полную стенограмму Handshake, 
строит по ней сначала ключ \code{resumption_master_secret}, а затем билет.
\end{note}

% skip: This might be appropriate in cases where the client is expected to open 
% multiple TLS connections in parallel and would benefit from the reduced 
% overhead of a resumption handshake, for example.

% todo: security?

\begin{note}
Сформировав один раз общие ключи, сервер может продлевать время их жизни, 
последовательно выпуская новые и новые билеты. Рекомендуется контролировать и 
ограничивать время жизни такого ключевого материала. 
%
В частности, если первоначальные ключи построены с использованием сертификатов, 
то при контроле следует учитывать срок действия сертификатов,  
возможность их отзыва. 
\end{note}

\subsection{Отложенная аутентификация}\label{HS.Post.Auth} 

Если клиент отправил расширение \token[HS.Ext.pha]{post_handshake_auth} 
(см.~\ref{HS.Ext.pha}), то он может быть аутентифицирован не только в протоколе 
Handshake, но и по его завершении.

% use: https://www.rfc-editor.org/errata/eid6401

Сервер запрашивает отложенную аутентификацию, отправляя сообщение 
\token[HS.CR]{CertificateRequest}. 

Клиент должен ответить сообщениями аутентификации (см.~\ref{HS.Auth}). 
%
Если клиент согласен аутентифицироваться, то он должен выслать сообщения
\token[HS.CT]{Certificate}, \token[HS.CV]{CertificateVerify} и 
\token[HS.F]{Finished}. 
%
Если клиент отказывается, то он должен выслать сообщение 
\token[HS.CT]{Certificate}, которое не содержит сертификатов,
а после него сообщение~\token[HS.F]{Finished}. 
%
Сообщения должны идти в оговоренном порядке и не должны перемежаться 
сообщениями других типов \addendum{или от других запросов}.

% use: https://mailarchive.ietf.org/arch/msg/tls/5hBdR59T3U9qSFJlCANqFObFPQs/

Сообщение \token[HS.CR]{CertificateRequest} и ответные сообщения клиента
защищаются на ключах, построенных по \code{server_application_traffic_secret}
и \code{client_application_traffic_secret} соответственно
(см.~\ref{CRYPTO.Schedule}). 

Если клиент не отправлял расширение \token[HS.Ext.pha]{post_handshake_auth},
но получил \token[HS.CR]{CertificateRequest} после Handshake, то он должен 
закрыть соединение с оповещением \token[ALERT.Err.um]{unexpected_message}.

Поскольку аутентификация клиента может включать запрос пользователю (владельцу 
личного ключа), сервер должен быть готов к задержке между 
\token[HS.CR]{CertificateRequest} и \token[HS.F]{Finished}.
Во время задержки сервер может получать сообщения, не относящиеся к 
аутентификации.
%
Кроме этого, порядок ответов клиента на несколько подряд идущих запросов 
\token[HS.CR]{CertificateRequest} не обязательно совпадает с порядком запросов.
%
Синхропосылка \code{certificate_request_context} в 
сообщениях~\token[HS.CR]{CertificateRequest} и \token[HS.CR]{Certificate}   
помогает серверу связать запросы с ответами.

\subsection{Сообщение \token{KeyUpdate}}\label{HS.KU} 

Сообщение \token{KeyUpdate} информирует об обновлении ключа защиты 
отправляемых прикладных данных. Правила обновления определены 
в~\ref{CRYPTO.Upd}. 

Формат сообщения \token{KeyUpdate} описывается одноименным типом:
%
\begin{codeblock}
enum {
  update_not_requested(0), update_requested(1), (255)
} KeyUpdateRequest;

struct {
  KeyUpdateRequest request_update;
} KeyUpdate;
\end{codeblock}

В поле \code{request_update} отправитель сообщения информирует о 
необходимости отправки встречного \token{KeyUpdate} или, другими словами,
необходимости обновления ключа защиты принимаемых прикладных данных.
%
Значение \code{update_requested} означает необходимость отправки встречного 
сообщения, значение \code{update_not_requested}~--- отсутствие необходимости.
%
Получатель должен закрыть соединение с оповещением 
\token[ALERT.Err.um]{unexpected_message}, если в поле указано недопустимое 
значение.

Сообщение \token{KeyUpdate} может отправить любая сторона после
того, как она выслала сообщение \token[HS.F]{Finished} и завершила,
таким образом, Handshake. 
%
Если клиент или сервер получает \token{KeyUpdate} перед 
\token[HS.F]{Finished}, то он должен закрыть соединение с оповещением 
\token[ALERT.Err.um]{unexpected_message}. 

После передачи \token{KeyUpdate} отправитель должен использовать обновленный ключ 
защиты отправляемых прикладных данных. 
%
При этом само сообщение \token{KeyUpdate} защищается на необновленном ключе.
%
Стороны не должны принимать записи, защищенные на обновленном ключе, вплоть до 
получения \token{KeyUpdate}, защищенного на необновленном.
%
% skip: Failure to do so may allow message truncation attacks.
%
Ключи защиты строятся по \code{server_application_traffic_secret} или 
\code{client_application_traffic_secret}, в зависимости от того, 
кто является отправителем (см.~\ref{CRYPTO.TKeys}).

Если в поле \code{request_update} сообщения \token{KeyUpdate} указано значение 
\code{update_requested}, то получатель должен ответить своим собственным
\token{KeyUpdate} прежде чем отправлять прикладные данные. 
%
В поле \code{request_update} встречного \token{KeyUpdate} должно быть указано 
значение \code{update_not_requested}.
%
Допустима ситуация, когда сторона получает несколько \token{KeyUpdate} с 
запросом на встречное сообщение и отвечает только одним таким сообщением.
%
Сообщение \token{KeyUpdate} с запросом запрещено отправлять повторно до 
получения искомого встречного \token{KeyUpdate}.

\begin{note}
После отправки \token{KeyUpdate} с запросом на встречное сообщение, сторона 
может получить это сообщение не сразу, а после других сообщений, которые 
находятся в процессе доставки. При этом в процессе доставки могут находиться 
запаздывающие сообщения \token{KeyUpdate}, не связанные с отправленным.
\end{note}

% skip: However, because send and receive keys are derived from independent 
% traffic secrets, retaining the receive traffic secret does not threaten the  
% forward secrecy of data sent before the sender changed keys.

\begin{note}
Если стороны независимо одновременно отправляют друг друг сообщения 
\token{KeyUpdate} с запросами на встречные сообщения, то, ответив на запросы, 
каждая сторона в конце концов обновит ключ отправляемых ею данных дважды.
\end{note}

% skip: With a 128-bit key as in AES-128, rekeying 2^64 times has a high
% probability of key reuse within a given connection.  Note that even
% if the key repeats, the IV is also independently generated, so the
% chance of a joint key/IV collision is much lower.  In order to
% provide an extra margin of security, sending implementations MUST NOT
% allow the epoch -- and hence the number of key updates -- to exceed
% 2^48-1.  In order to allow this value to be changed later -- for
% instance for ciphers with more than 128-bit keys -- receiving
% implementations MUST NOT enforce this rule.  If a sending
% implementation receives a KeyUpdate with request_update set to
% "update_requested", it MUST NOT send its own KeyUpdate if that would
% cause it to exceed these limits and SHOULD instead ignore the
% "update_requested" flag.  This may result in an eventual need to
% terminate the connection when the limits in Section 5.5 are reached.
