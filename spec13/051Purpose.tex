\section{Назначение}\label{COMMON.Purpose}

Настоящий стандарт определяет криптографический протокол TLS версии 1.3,
далее просто TLS.
%
Протокол предназначен для защиты соединений между клиентами и серверами в сети 
Интернет.
%
Протокол определяется в соответствии со спецификацией~\cite{RFC8446}.
%
Действия сторон протокола и форматы пересылаемых между сторонами сообщений
представлены с такой степенью детализации, которая позволяет разрабатывать
полностью совместимые между собой реализации TLS.

TLS встраивается в стек коммуникационных протоколов поверх транспортного
уровня и обеспечивает защиту данных, передаваемых через этот уровень
прикладными протоколами клиента и сервера. TLS выполняется независимо от
прикладных протоколов и прозрачен для них.

С помощью TLS клиент и сервер создают защищенное соединение, в рамках которого
проводится аутентификация сторон, обеспечиваются конфиденциальность и контроль 
целостности и подлинности передаваемых данных.
%
Для организации соединения достаточно обеспечить надежную передачу данных на 
транспортном уровне.

% skip: These properties should be true even in the face of an attacker who
% has complete control of the network, as described in [RFC3552].  See
% Appendix F for a more complete statement of the relevant security
% properties.

Протокол TLS предыдущей версии 1.2 установлен в СТБ 34.101.65. 
%
Настоящий стандарт не отменяет СТБ~34.101.65~--- оба стандарта, как и 
соответствующие версии TLS, могут применяться одновременно независимо друг от 
друга.

% skip: \begin{note*}
% Настоящий стандарт выполнен в виде отдельного от СТБ~34.101.65 нормативного 
% документа, поскольку различия между версиями 1.2 и 1.3 существенны.
% \end{note*}

Настоящий стандарт уточняет СТБ~34.101.65 в двух аспектах. Во-первых, в 
приложении~\ref{EMS} определяется обязательное к применению расширение,
которое защищает от атаки, обнаруженной после введения СТБ~34.101.65 в действие.
%
Во-вторых, в необходимых случаях детализируется взаимодействие сторон TLS 1.2 
c клиентами и серверами, которые поддерживают версию 1.3 протокола.

