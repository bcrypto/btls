\section{PSK-скрепки}\label{CRYPTO.Binder} 

PSK-скрепка устанавливает связь между секретом PSK и текущим сеансом Handshake. 
Если секрет был согласован в предыдущем соединении (с помощью 
\token[HS.NST]{NewSessionTicket}), то дополнительно устанавливается связь с 
этим соединением. 

PSK-скрепки вычисляются при подготовке расширения 
\token[HS.Ext.psk]{pre_shared_key} в момент отправки первого или повторного 
сообщения \token[HS.CH]{ClientHello}. 
%
Расширение описывается типом~\code{OfferedPsks}, определенным 
в~\ref{HS.Ext.psk}, и состоит из полей \code{identities} и \code{binders}.
%
Скрепки сохраняются в последнем поле.
%
Расширение \token[HS.Ext.psk]{pre_shared_key} является последним в списке 
расширений \token[HS.CH]{ClientHello} и, следовательно, завершает сообщение.

При вычислении PSK-скрепки используется текущая стенограмма Handshake. Она 
заканчивается текущим сообщением \token[HS.CH]{ClientHello}: первым или 
повторным.
%
Стенограмма сокращается: в расширении \token[HS.Ext.psk]{pre_shared_key} 
завершающего \token[HS.CH]{ClientHello} исключается поле~\code{binders}.

Сокращенная стенограмма хэшируется. В результате получается хэш-значение
\begin{align*}
H&\gets\code{Transcript-Hash}(\code{Truncate}(\code{ClientHello})),
\intertext{если Handshake выполняется без перезапуска, и хэш-значение}
H&\gets\code{Transcript-Hash}(\code{ClientHello},\code{HelloRetryRequest}, 
\code{Truncate}(\code{ClientHello}_2)),
\end{align*}
если с перезапуском. 
%
Здесь \code{Truncate}~--- функция исключения \code{binders}, 
$\code{ClientHello}_2$~--- повторное приветственное сообщение клиента.

После определения $H$ скрепка \code{binder} вычисляется следующим образом:
\begin{align*}
\code{finished_key}&\gets\code{HKDF-Expand-Label}
(\code{binder_key}, \str{finished}, \perp, \code{hash_length}),\\
\code{binder}&\gets\code{HMAC}(\code{finished_key}, H).
\end{align*}

% skip: The full ClientHello1/ClientHello2 is included in all other handshake
% hash computations.  Note that in the first flight, Truncate(ClientHello1) is 
% hashed directly, but in the second flight, ClientHello1 is hashed and then 
% reinjected as a message_hash message, as described in Section 4.4.1.  
% Note that the message_hash will be hashed with the negotiated function, which 
% may or may match the hash associated with the PSK.  This is consistent with 
% how the transcript is calculated for the rest of the handshake.
