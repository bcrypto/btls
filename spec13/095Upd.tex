\section{Обновление ключей}\label{CRYPTO.Upd}

По завершении Handshake стороны формируют общие ключи
\code{client_application_traffic_secret} и
\code{server_application_traffic_secret}, а затем строят по ним ключи защиты
прикладных данных.

Впоследствии клиент может обновить \code{client_application_traffic_secret}
и построить по нему новый ключ защиты данных, отправляемых серверу.
%
Соответственно, сервер может обновить \code{server_application_traffic_secret}
и построить новый ключ защиты данных, отправляемых клиенту.

Правила обновления:
\begin{align*}
\code{client_application_traffic_secret}&\gets\code{HKDF-Expand-Label}(\\
  &\hspace{36pt}\code{client_application_traffic_secret},\\
  &\hspace{36pt}\str{traffic upd}, \perp, \code{hash_length}),\\
\code{server_application_traffic_secret}&\gets\code{HKDF-Expand-Label}(\\
  &\hspace{36pt}\code{server_application_traffic_secret},\\
  &\hspace{36pt}\str{traffic upd}, \perp, \code{hash_length}).
\end{align*}

Перед обновлением ключей сторона-инициатор должна отправить противоположной 
стороне сообщение \token[HS.KU]{KeyUpdate} (см.~\ref{HS.KU}).

После обновления следует удалить предыдущие ключи.
