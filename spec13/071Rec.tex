\section{Записи}\label{RECORD.Rec}

Протокол Record предназначен для передачи данных. Данные разбиваются на 
фрагменты, фрагменты помещаются в записи, записи пересылаются противоположной 
стороне. При получении записей их фрагменты объединяются и передаются 
протоколам верхних уровней. Записи могут пересылаться в защищенной форме.

В записи указывается ее тип. Различные протоколы верхних уровней могут 
передавать свои сообщения через общий протокол Record, используя записи 
различных типов.

В настоящем стандарте определены 4 типа записей:
\begin{itemize}
\item
\code{handshake}~--- сообщения протокола Handshake;
\item
\code{application_data}~--- сообщения прикладных протоколов;
\item
\code{alert}~--- сообщения протокола Alert;
\item
\code{change_cipher_spec}~--- используется для совместимости с промежуточными 
сетевыми устройствами, ориентированными на работу с сообщениями TLS 1.2.
\end{itemize}

Реализации TLS не должны отправлять записи, типы которых отличаются от
перечисленных выше. Исключение могут составить типы, которые будут
устанавливаться для расширений TLS за рамками настоящего стандарта. Если
реализация получает запись неустановленного типа, то она должна прервать
протокол с оповещением \token[ALERT.Err.um]{unexpected_message}.

Запись типа \code{change_cipher_spec} может быть получена в процессе выполнения
Handshake после отправки или получения первого сообщения
\token[HS.CH]{ClientHello} (отличного от \token[HS.CH]{ClientHello}, которое
следует после \token[HS.HRR]{HelloRetryRequest}) и до того, как получено
сообщение \token[HS.F]{Finished} от противоположной стороны.
%
Запись должна быть незащищена и должна содержать однобайтовый фрагмент 
\code{0x01}.
%
% skip: Note that this record may appear at a point at the handshake where the 
% implementation is expecting protected records, and so it is necessary 
% to detect this condition prior to attempting to deprotect the record.
%
При соблюдении данных условий полученная запись должна игнорироваться.
%
При нарушении любого из условий протокол Handshake должен быть прерван с 
оповещением \token[ALERT.Err.um]{unexpected_message}.
%
Протокол должен быть прерван с тем же оповещением, если обнаружено, 
что запись \code{change_cipher_spec} получена до первого 
\token[HS.CH]{ClientHello} или после ответного \token[HS.F]{Finished}.
%
Сервер без состояния, лишенный возможности обнаруживать нарушение 
порядка записей, может игнорировать последнее требование.

