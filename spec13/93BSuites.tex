\begin{appendix}{В}{обязательное}{Использование алгоритмов СТБ}\label{BSUITES}

\mbox{}

\hiddensection{Назначение}\label{BSUITES.Intro}

В настоящем приложении определяются правила использования в TLS 
криптографических алгоритмов, установленных в стандартах серии СТБ~34.101.  
%
Правила касаются криптонаборов (\ref{BSUITES.Suites}), 
протокола Диффи~--- Хеллмана (\ref{BSUITES.DH}),  
групп точек эллиптической кривой (\ref{BSUITES.DH}), 
алгоритмов ЭЦП (\ref{BSUITES.DS}).

Требования настоящего приложения являются обязательными только для тех 
реализаций TLS, в которых используются алгоритмы СТБ.

\hiddensection{Криптонаборы}\label{BSUITES.Suites}

Должны использоваться криптонаборы из следующего списка 
(см. таблицу \ref{Table.BSUITES.Suites}): 
\begin{enumerate}
\item
\code|TLS_BELT_CHE256_BELT_HASH|~--- алгоритмы аутентифицированного шифрования
\code{belt-che256} в связке с алгоритмом хэширования \code{belt-hash}. Алгоритмы
определены в СТБ 34.101.31.
\item
\code|TLS_BASH_PRG_AE2561_BASH256|~--- алгоритмы аутентифицированного шифрования
\code{bash-prg-ae2561} в связке с алгоритмом хэширования \code{bash256}.
Алгоритмы определены в СТБ 34.101.77.
\end{enumerate}

Криптонабор \code|TLS_BASH_PRG_AE2561_BASH256| обязателен к реализации.

\begin{table}[h]
\caption{Криптонаборы}\label{Table.BSUITES.Suites}
\begin{tabular}{|l|c|c|}
\hline
Криптонабор & 
Идентификатор\\
\hline
\hline
\code|TLS_BELT_CHE256_BELT_HASH| & \{255, 29\}\\
\hline
\code|TLS_BASH_PRG_AE2561_BASH256| & \{255, 30\}\\
\hline
\end{tabular}
\end{table}

Алгоритмы криптонаборов имеют следующие размерности:
%
\begin{itemize}
\item
\code{key_length}, длина в байтах ключа алгоритмов аутентифицированного 
шифрования, равняется 32;
\item
\code{iv_length}, длина в байтах синхропосылки алгоритмов аутентифицированного 
шифрования, равняется 16;
\item
\code{hash_length}, длина в байтах выхода алгоритма хэширования, равняется 32.
\end{itemize}
%
% В алгоритмах \code{bash-prg-ae2561} параметр $d$ (емкость) равняется 1.

При аутентифицированном шифровании в протоколе Record ассоциированные 
данные составляются из полей \code{opaque_type}, 
\code{legacy_record_version} и \code{length} структуры 
\token[RECORD.Enc]{TLSCiphertext}. Размер ассоциированных данных~--- 5 байтов.

В алгоритмах \code{bash-prg-ae2561} ассоциированные данные~$I\in\{0,1\}^{40}$ 
переносятся в анонc~$A$, объединяясь с синхропосылкой $S\in\{0,1\}^{128}$:
$$
A\gets S\parallel I\parallel 0^{24}.
$$
Шаг 2.1 с вызовом $\alpha.\code{absorb}(I)$ исключается и при установке защиты, 
и при ее снятии.

\begin{note}
Анонс дополняется нулями, чтобы соответствовать требованиям 
\code{bash-prg-ae2561}: длина анонса в битах должна быть кратна 32.
\end{note}

В алгоритмах \code{belt-che256} ассоциированные данные и синхропосылка 
используются обычным образом, в соответствии с интерфейсом алгоритмов.

Имитовставки \code{belt-che256} состоят из 8~байтов, имитовставки 
\code{bash-prg-ae2561}~--- из~32. При использовании алгоритмов в TLS
имитовставки не должны сокращаться. Имитовставка~$T$ добавляется к 
шифртексту~$Y$, образуя аутентифицированный шифртекст $Y\parallel T$,
который размещается в~\code{TLSCiphertext.encrypted_record}.

Ключ \code{belt-che256} должен обновляться при достижении порога в $2^{32}$ 
защищенных записей. Ключ \code{bash-prg-ae2561} используется безлимитно.

\begin{note}
Максимальное число записей \code{belt-che256} определено в соответствии с 
СТБ~34.101.31 (приложение В) на среднем уровне гарантий. 
%
На этом уровне квота ключа составляет $2^{48}\sqrt{2/(5D+7)}$ 16-байтовых блоков
аутентифицированного шифртекста, где $D$~--- максимальное число блоков, 
обрабатываемых за один вызов \code{belt-che256}. 
%
В контексте Record величина $D=1+1025+1=1027$: 
один неполный блок ассоциированных данных, не более 
$\lceil(2^{14}+1)/16\rceil=1025$ блоков в \code{TLSInnerPlaintext} и один 
служебный блок \code{belt-che256}.
%
При таком~$D$ квота ключа составляет $2^{42.33}$ блоков.
%
Аутентифицированный шифртекст в \code{TLSCiphertext.encrypted_record}
состоит из не более чем $1025+1=1026$ блоков: не более 
$\lceil(2^{14}+1)/16\rceil=1025$ блоков шифртекста и один неполный блок с 
имитовставкой.
%
Поэтому без смены ключа разрешается защитить как минимум 
$2^{42.33}/1026>2^{32}$ записей Record.
\end{note}

С секретом, предварительно согласованным за пределами TLS, должен быть 
ассоциирован алгоритм хэширования (см.~\ref{HS.Ext.psk}). В качестве такого 
алгоритма должен использоваться либо \code{belt-hash}, либо \code{bash256}.
%
По умолчанию, если ассоциированный алгоритм хэширования не задан явно, должен 
использоваться \code{belt-hash}.

\hiddensection{Протокол Диффи~--- Хеллмана}\label{BSUITES.DH}

Протокол Диффи~--- Хеллмана должен быть реализован в соответствии со следующими 
правилами, установленными в СТБ 34.101.66 и основанными на соглашениях и 
алгоритмах~СТБ~34.101.45.

\begin{enumerate}
\item
В качестве базовой циклической группы должна использоваться группа точек 
эллиптической кривой, которая удовлетворяет соглашениям СТБ~34.101.45.

\item
Одноразовые личный и открытый ключи должны генерироваться с помощью алгоритма
\code{bign-genkeypair} (СТБ 34.101.45).

\item
Стороны должны проверять открытые ключи друг друга с помощью алгоритма
\code{bign-valpubkey} (СТБ 34.101.45).

\item
Стороны должны вычислять общий ключ с помощью алгоритма \code{bake-dh} (СТБ
34.101.66): общий ключ $K=uV$, где~$u$~--- одноразовый личный ключ (скаляр),
$V$~--- одноразовый открытый ключ противоположной стороны (точка кривой).

\item
Общий ключ~$K$, аффинная точка эллиптической кривой, должен кодироваться строкой
октетов по правилам, заданным в СТБ 34.101.45 (пункт 5.4). Кодируются обе
координаты~$K$, код $x$-координаты указывается первым. При кодировании координат
используются правила <<от младших к старшим>> (little-endian), незначащие нули
не отбрасываются. При использовании эллиптической кривой уровня стойкости~$\ell$
кодовое представление~$K$ состоит из $\ell/2$ октетов.
\end{enumerate}

\hiddensection{Группы точек эллиптической кривой}\label{BSUITES.Groups}

В протоколе Диффи~--- Хеллмана должны использоваться следующие группы точек 
эллиптической кривой, которые удовлетворяют соглашениям СТБ 34.101.45 и 
определяются стандартными параметрами, установленными в приложении Б этого 
стандарта:
\begin{itemize}
\item
\code{bign-curve256v1}~--- стандартные параметры уровня стойкости $\ell=128$;
\item
\code{bign-curve384v1}~--- стандартные параметры уровня стойкости $\ell=192$;
\item
\code{bign-curve512v1}~--- стандартные параметры уровня стойкости $\ell=256$.
\end{itemize}

Идентификаторы групп добавляются в перечисление~\code{NamedGroup}
(см.~\ref{HS.Ext.sg}):
\begin{codeblock}
NamedGroup += {
  bign_curve256v1(0xFE01),
  bign_curve384v1(0xFE02),
  bign_curve512v1(0xFE03)
};
\end{codeblock}

Идентификаторы используются в расширениях \token[HS.Ext.sg]{supported_groups} 
и \token[HS.Ext.ks]{key_share}.

\hiddensection{Алгоритмы электронной цифровой подписи}\label{BSUITES.DS}

Должны использоваться следующие алгоритмы выработки и проверки ЭЦП,
установленные в СТБ 34.101.45 и сопровождаемые алгоритмами хэширования 
СТБ 34.101.31 и СТБ 34.101.77:
\begin{itemize}
\item
\code{bign-with-hbelt}~--- алгоритмы ЭЦП в связке c алгоритмом хэширования 
\code{belt-hash} (СТБ 34.101.31);
\item
\code{bign-with-bash384}~--- алгоритмы ЭЦП в связке c алгоритмом хэширования 
\code{bash384} (СТБ 34.101.77);
\item
\code{bign-with-bash512}~--- алгоритмы ЭЦП в связке c алгоритмом хэширования 
\code{bash512} (СТБ 34.101.77).
\end{itemize}

В алгоритмах \code{bign-with-hbelt} должны использоваться параметры
\code{bign-curve256v1}, в алгоритмах \code{bign-with-bash384}~--- 
параметры \code{bign-curve384v1}, в алгоритмах \code{bign-with-bash512}~--- 
параметры \code{bign-curve512v1}.

Идентификаторы алгоритмов ЭЦП добавляются в перечисление~\code{SignatureScheme}
(см.~\ref{HS.Ext.sa}):
\begin{codeblock}
SignatureScheme += {
  bign_with_hbelt(0xFE01),
  bign_with_bash384(0xFE02),
  bign_with_bash512(0xFE03)
};
\end{codeblock}

Идентификаторы используются в расширениях \token[HS.Ext.sa]{signature_algorithms} 
и \token[HS.Ext.sa]{signature_algorithms_cert}.

\end{appendix}
\mbox{}
\vfill
\mbox{}
\clearpage
