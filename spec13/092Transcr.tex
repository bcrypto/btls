\section{Стенограммы}\label{CRYPTO.Transcr} 

При криптографических вычислениях используются стенограммы Handshake.
Стенограммы составляются из сообщений Handshake, переданных между сторонами к 
определенному моменту. 
%
Другими словами, стенограмма~--- это подпоследовательность следующей 
последовательности:
\begin{align*}
&\code{ClientHello}, \code{HelloRetryRequest}, \code{ClientHello}_2, 
\code{ServerHello},\code{EncryptedExtensions},\\ 
%
&\hspace{12pt}\code{CertificateRequest}_S, \code{Certificate}_S, 
\code{CertificateVerify}_S, \code{Finished}_S,\\
%
&\hspace{12pt}\code{EndOfEarlyData}, \code{Certificate}_C, 
\code{CertificateVerify}_C, \code{Finished}_C.
\end{align*}
Здесь нижний индекс~$2$ означает повторное сообщение, 
индекс $S$~--- сообщение сервера, $C$~--- клиента.
%
В самом начале Handshake стенограмма представляет собой пустую 
последовательность~$\perp$. Первое сообщение стенограммы~--- это 
обязательно~\token[HS.CH]{ClientHello}.

В сообщениях стенограммы сохраняются поля типа и длины (см.~\ref{HS.Msg}). 
Заголовки записей Record, в которые сообщения Handshake инкапсулируются, не 
учитываются. Сообщения интерпретируются как слова из~$\{0,1\}^{8*}$. 

Стенограммы хэшируется с помощью алгоритма \code{Transcript-Hash}. 
Алгоритм получает на вход стенограмму $M=(M_1,M_2,\ldots,M_m)$
и возвращает хэш-значение $Y\in\{0,1\}^n$.

Шаги алгоритма:
\begin{enumerate}
\item
Если $M$ содержит сообщение~\token[HS.HRR]{HelloRetryRequest}, то 
\begin{enumerate}
\item
$M_1\gets\itob{\code{message_type}}_8\parallel 0^{16}\parallel
\itob{\code{hash_length}}_8\parallel\code{Hash}(M_1)$,
где \code{message_type}~--- константа, заданная в перечислении
\code{HandshakeType} (см.~\ref{HS.Msg}).
\end{enumerate}
\item
Установить $Y\gets\code{Hash}(M_1\parallel M_2\parallel\ldots M_m)$.
\item
Возвратить~$Y$.
\end{enumerate}

\begin{note*}
На шаге 1.1 алгоритма $M_1=\code{ClientHello}$, $M_2=\code{HelloRetryRequest}$.
%
Сообщение \token[HS.CH]{ClientHello} заменяется искусственным сообщением типа 
\code{message_type}, которое содержит $\code{Hash}(\code{ClientHello})$.
%
Замена позволяет серверу без состояния перезапускать Handshake, не сохраняя
\token[HS.CH]{ClientHello}, а пересылая его хэш-значение в расширении
\token[HS.Ext.c]{cookie} сообщения~\token[HS.HRR]{HelloRetryRequest}
(см.~\ref{HS.Ext.c}).
\end{note*}

% skip: In general, implementations can implement the transcript by keeping a  
% running transcript hash value based on the negotiated hash.  Note,
% however, that subsequent post-handshake authentications do not
% include each other, just the messages through the end of the main
% handshake.
