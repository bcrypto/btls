\section{Компоненты}\label{COMMON.Components}

TLS состоит из трех субпротоколов: Handshake, Record и Alert 
(см. разделы~\ref{HS}~--- \ref{ALERT}). 

Протокол Handshake позволяет клиенту и серверу аутентифицировать друг 
друга, согласовать параметры, в том числе версию TLS и перечень 
криптографических алгоритмов, сформировать общие ключи.
%
Аутентификация сервера в Handshake обязательна, клиента~--- нет.
%
Протокол противостоит вмешательству со стороны противника.
%
В частности, противник не может навязать сторонам параметры, 
отличные от тех, которые были бы согласованы при отсутствии вмешательства.

В Handshake предусмотрены три режима формирования общих ключей:
\begin{enumerate}[label=\arabic*)]
\item
DHE~--- протокол Диффи~--~Хеллмана c одноразовыми ключами; 
\item
PSK~--- на основе предварительно согласованных секретов;
\item
PSK+DHE~--- составной режим, композиция двух предыдущих.
\end{enumerate}

Параметры и ключевой материал режимов передается в специальных расширениях 
сообщений Handshake. В расширениях могут передаваться другие параметры и 
настройки TLS. Перечень расширений может пополняться.

Аутентификация основана на знании личных ключей, связанных с сертификатами 
открытых ключей, и (или) на знании предварительно согласованных секретов.
%
Используются сертификаты, содержание и формат которых определены в 
СТБ~34.101.19.

Протокол Record использует согласованные при выполнении Handshake параметры и 
сформированные общие ключи для защиты данных обмена между клиентом и сервером.
%
Данные разбиваются на фрагменты, которые независимо обрабатываются на ключах 
защиты.

Протокол Alert извещает о закрытии соединений и об ошибках в процессе 
выполнения Handshake и Record.

Защита данных в протоколе Record выполняется с помощью алгоритмов
аутентифицированного шифрования. Их ключи формируются с помощью алгоритмов схемы
HKDF (HMAC-based Key Derivation Function), введенной в~\cite{Kra10,RFC5869}.
%
Алгоритмы HKDF, правила формирования ключей с их помощью, другие
криптографические аспекты TLS описаны в разделе~\ref{CRYPTO}.

В HKDF используются алгоритмы имитозащиты схемы HMAC, установленной в СТБ 
34.101.47. Параметром HMAC и, следовательно, HKDF является алгоритм хэширования. 
Этот алгоритм в связке с алгоритмами аутентифицированного шифрования образует 
криптонабор TLS.
%
В TLS предусмотрена возможность расширения перечня криптонаборов. В 
приложении~\ref{BSUITES} определены криптонаборы, основанные на 
алгоритмах, введенных в стандартах серии СТБ 34.101.

Криптонабор определяет правила криптографических вычислений за пределами Record. 
%
Во-первых, сообщения Handshake и Alert могут защищаться по той же схеме, что и 
фрагменты Record. 
%
Во-вторых, алгоритм имитозащиты HMAC, определяемый алгоритмом хэширования 
криптонабора, может использоваться в Handshake самостоятельно, не в качестве 
компонента HKDF.

Клиент и сервер согласуют используемый криптонабор в начале выполнения Handshake. 
%
Кроме этого, стороны информируют друг друг о своих криптографических 
возможностях: поддерживаемых циклических группах для протокола 
Диффи~--- Хеллмана, поддерживаемых алгоритмах ЭЦП для аутентификации по 
сертификату и для проверки сертификата.

