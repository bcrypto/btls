\section{Алгоритмы построения ключей}\label{CRYPTO.HKDF}

\subsection{Назначение}\label{CRYPTO.HKDF.Intro}

Для построения ключей используются алгоритмы \code{HKDF-Extract} и 
\code{HKDF-Expand} схемы HKDF (см.~\ref{COMMON.Components}).
%
C помощью \code{HKDF-Extract} по энтропийному слову $X$ произвольной длины
(ключевому материалу) и синхропосылке строится ключ~$K$ фиксированной длины.
%
Затем с помощью \code{HKDF-Expand} по ключу~$K$ и еще одной синхропосылке
строится набор подчиненных ключей.
%
Алгоритмы \code{HKDF-Extract} и \code{HKDF-Expand} можно вызывать рекурсивно,
используя построенные ключи в качестве ключевого материала и синхропосылок
при новых вызовах.
%
В результате будет получен набор ключей, иерархически связанных друг с другом.
%
Связь такова, что по подчиненному ключу вычислительно трудно определить как ключ, 
по которому он построен, так и другие подчиненные ключи.

Для учета специфики TLS алгоритм \code{HKDF-Expand} расширяется сначала до алгоритма
\code{HKDF-Expand-Label}, а затем до \code{Derive-Secret}. При расширении
усложняется интерфейс \code{HKDF-Expand}~--- вместо единой синхропосылки на вход
подаются ее части. 
%
В первой части указывается метка (название) ключа, который строится, во второй~--- 
контекст, в котором построение выполняется. В \code{Derive-Secret} в качестве 
контекста выступает стенограмма протокола Handshake. Контекст может быть пустым. 

\subsection{Интерфейс}\label{CRYPTO.HKDF.IFace}

Входными данными \code{HKDF-Extract} являются ключевой
материал~$X\in\{0,1\}^{8*}$ и синхропосылка~$S\in\{0,1\}^{8*}$. Выходными
данными является ключ~$K\in\{0,1\}^n$.

Входными данными \code{HKDF-Expand} являются ключ~$K\in\{0,1\}^{8*}$,
синхропосылка~$S\in\{0,1\}^{8*}$ и неотрицательное целое~$m$~--- длина выхода в 
байтах. Выходными данными является подчиненный ключ~$Y\in\{0,1\}^{8m}$.

Входными данными \code{HKDF-Expand-Label} являются 
ключ~$K\in\{0,1\}^{8*}$, метка подчиненного ключа~$L\in\{0,1\}^{8*}$,
контекст~$C\in\{0,1\}^{8*}$ и длина~$m$. 
%
Выходными данными является подчиненный ключ~$Y\in\{0,1\}^{8m}$.

Входными данными \code{Derive-Secret} являются ключ~$K\in\{0,1\}^{8*}$, 
метка подчиненного ключа~$L\in\{0,1\}^{8*}$ и стенограмма $M$, 
которая представляет собой последовательность сообщений Handshake.
%
Выходными данными является подчиненный ключ~$Y\in\{0,1\}^n$.

Должны соблюдаться следующие ограничения на входные данные:
\begin{itemize}
\item
длина~$m$ не должна превосходить~$255n/8$;
\item
длина метки~$L$ в байтах должна лежать в диапазоне от~$1$ до~$249$;
\item
длина контекста~$C$ в байтах не должна превосходить~$255$.
\end{itemize}

Метки ключей кодируются строками ASCII с окаймлением двойными кавычками 
(см.~\ref{DEFS}). В алгоритме \code{HKDF-Expand-Label} к меткам добавляется 
префикс \str{tls13 }.

\subsection{Алгоритм \code{HKDF-Extract}}\label{CRYPTO.HKDF.Extract}

Вычисление $\code{HKDF-Extract}(X,S)$ выполняется следующим образом:

\begin{enumerate}
\item
Если $S=\perp$, то $S\gets 0^n$.
\item
Установить~$K\gets\code{HMAC}(S,X)$.
\item
Возвратить~$K$.
\end{enumerate}

\subsection{Алгоритм \code{HKDF-Expand}}\label{CRYPTO.HKDF.Expand}

Вычисление $\code{HKDF-Expand}(K,S,m)$ выполняется следующим образом:

\begin{enumerate}
\item
Установить~$(Y,T)\gets(\perp,\perp)$ (пустые строки).
\item
Для $i=1,2,\ldots,\lceil 8m/n\rceil$:
\begin{enumerate}
\item
$T\gets\code{HMAC}(K,T\parallel S\parallel\itob{i}_8)$;
\item
$Y\gets Y\parallel T$.
\end{enumerate}
\item
Установить~$Y\gets\Lo(Y,8m)$.
\item
Возвратить~$Y$.
\end{enumerate}

\subsection{Алгоритм \code{HKDF-Expand-Label}}\label{CRYPTO.HKDF.Label}

В алгоритме \code{HKDF-Expand-Label} используется структура данных \code{HkdfLabel}:
\begin{codeblock}
struct {
  uint16 length;
  opaque label<7..255>;
  opaque context<0..255>;
} HkdfLabel;
\end{codeblock}

Вычисление $\code{HKDF-Expand-Label}(K,L,C,m)$ выполняется следующим образом: 

\begin{enumerate}
\item
Построить структуру \code{HkdfLabel}, заполнив ее поля следующим образом:
\begin{enumerate}
\item
$\code{HkdfLabel.length}\gets m$;
\item
$\code{HkdfLabel.label}\gets \str{tls13 } \parallel L$;
\item
$\code{HkdfLabel.context}\gets C$.
\end{enumerate}
\item
Установить~$Y\gets\code{HKDF-Expand}(K, \code{HkdfLabel}, m)$.
\item
Возвратить~$Y$.
\end{enumerate}

\subsection{Алгоритм \code{Derive-Secret}}\label{CRYPTO.HKDF.Derive}

Вычисление $\code{Derive-Secret}(K,L,M)$ выполняется следующим образом: 
%
\begin{enumerate}
\item
Установить $H\gets\code{Transcript-Hash}(M)$.
\item
Установить~$Y\gets\code{HKDF-Expand-Label}(K,L,H,\code{hash_length})$.
\item
Возвратить~$Y$.
\end{enumerate}

