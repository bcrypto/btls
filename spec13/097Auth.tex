\section{Аутентификаторы}\label{CRYPTO.Auth} 

Для подтверждения своей подлинности и (или) демонстрации владения общими ключами 
стороны вычисляют аутентификаторы~--- подписи и имитовставки стенограмм Handshake. 
%
Подпись размещается в поле \code{signature} сообщения 
\token[HS.CV]{CertificateVerify}, имитовставка~--- в поле \code{verify_data} 
сообщения \token[HS.F]{Finished}.
%
Клиент может отправлять сообщения \token[HS.CV]{CertificateVerify} и 
\token[HS.F]{Finished} как в процессе выполнения Handshake, так и после 
(см.~\ref{HS.Post.Auth}).

Подписывается результат конкатенации следующих объектов: 
\begin{enumerate}[label=\arabic*)]
\item
строка из 64 октетов $\hex{20}$ (пробел в таблице ASCII);
\item
строка \str{TLS 1.3, server CertificateVerify}, если подписантом 
является сервер, или строка \str{TLS 1.3, client CertificateVerify}, если 
клиент; 
%
% use: https://www.rfc-editor.org/errata/eid6141
%
\item
октет $\hex{00}$ (разделитель);
\item
хэш-значение 
$\code{Transcript-Hash}(\code{ClientHello},\ldots,\code{Certificate})$, 
где \token[HS.CT]{Certificate}~--- сообщение подписанта, непосредственно 
предшествующее его же сообщению \token[HS.CV]{CertificateVerify}, в котором 
будет размещаться подпись.
\end{enumerate}

\begin{note*}
Выбранный формат подписываемых данных защищает от угроз для сервера, который
поддерживает соединения TLS 1.2, используя тот же сертификат и соотвественно
личный ключ. При выборе в TLS 1.2 режима DHE сервер подписывает и отправляет в
сообщении \code{ServerKeyExhchange} данные, начальные 32 байта которых 
контролирует клиент.
\end{note*}

% info: https://mailarchive.ietf.org/arch/msg/tls/AYSDPvSRV8bvwis8HTGTzwLh1IM/

\begin{example*}
Если хэш-значение состоит из 32 октетов $\hex{01}$, то сервер будет подписывать 
следующее двоичное слово:
\begin{align*}
&\hexz{2020202020202020202020202020202020202020202020202020202020202020}\\
&\hexz{2020202020202020202020202020202020202020202020202020202020202020}\\
&\hexz{544c5320312e332c20736572766572204365727469666963617465566572696679}\\
&\hexz{00}\\
&\hex{0101010101010101010101010101010101010101010101010101010101010101}.
\end{align*}
\end{example*}

Стенограмма, для которой вычисляется имитовставка, завершается сообщением,
указанным в таблице~\ref{Table.CRYPTO.TranscrMac}.
%
В таблице нижний индекс $S$ означает сообщения сервера, индекс $C$~--- клиента. 
%
В определенных случаях имеется несколько вариантов завершающего сообщения.
%
Следует выбирать сообщение, которое отправляется в Handshake позднее других.

\begin{table}[bht]
\caption{Имитовставки стенограмм}\label{Table.CRYPTO.TranscrMac}
\begin{tabular}{|l|l|l|}
\hline
Сторона & Окончание стенограммы & Ключ \code|base_key|\\
\hline
\hline
Сервер 
  & $\code|EncryptedExtensions|$ & \code|server_handshake_traffic_secret|\\
  & $\code|CertificateRequest|_S$ & \\
\hline
Клиент
  & $\code|Finished|_S$ & \code|client_handshake_traffic_secret|\\
(Handshake)
  & $\code|EndOfEarlyData|$ & \\
  & $\code|CertificateVerify|_C$ & \\
\hline
Клиент
  & $\code|CertificateVerify|_C$ & \code|client_application_traffic_secret|\\
(после Handshake) 
  &&\\
\hline
\end{tabular}
\end{table}

% check: https://www.rfc-editor.org/errata/eid6151
% * client_application_traffic_secret -> [sender]_application_traffic_secret?

При вычислении имитовставки стенограммы используется ключ \code{base_key}, 
также представленный в таблице~\ref{Table.CRYPTO.TranscrMac}.
%
Вычисления выполняются следующим образом:
\begin{align*}
H&\gets\code{Transcript-Hash}(M),\\
%
\code{finished_key}
  &\gets\code{HKDF-Expand-Label}
(\code{base_key}, \str{finished}, \perp, \code{hash_length}),\\
%
\code{verify_data}
  &\gets\code{HMAC}(\code{finished_key}, H).
\end{align*}
Здесь $M$~--- исходная стенограмма, \code{verify_data}~--- искомая 
имитовставка.

% skip: HMAC [RFC 2104] uses the Hash algorithm for the handshake. As noted 
% above, the HMAC input can generally be implemented by a running hash, i.e., 
% just the handshake hash at this point. 

% skip: In previous versions of TLS, the verify_data was always 12 octets long. 
% In TLS 1.3, it is the size of the HMAC output for the Hash used for the 
% handshake.

% skip: Alerts and any other non-handshake record types are not handshake 
% messages and are not included in the hash computations.
