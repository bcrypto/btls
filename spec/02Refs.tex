\chapter{Нормативные ссылки}\label{REFS}

В настоящем стандарте использованы ссылки на следующие технические 
нормативные правовые акты в области технического нормирования и 
стандартизации (далее~--- ТНПА):  

СТБ~34.101.19-2012 Информационная технологии и безопасность. Форматы 
сертификатов и списков отозванных сертификатов инфраструктуры открытых 
ключей 

СТБ~34.101.31-2011 Информационные технологии. Защита информации. 
Криптографические алгоритмы шифрования и контроля целостности 

СТБ~34.101.45-2013 Информационные технологии и безопасность. Алгоритмы  
электронной цифровой подписи и транспорта ключа на основе эллиптических 
кривых 

СТБ~34.101.47-2012 Информационные технологии и безопасность. 
Криптографические алгоритмы генерации псевдослучайных чисел 

ГОСТ~34.973-91 (ИСО 8824-87) Информационная технология. Взаимосвязь 
открытых систем. Спецификация абстрактно-синтаксической нотации версии 1 
(АСН.1) 

ГОСТ~34.974-91 (ИСО 8825-87) Информационная технология. Взаимосвязь 
открытых систем. Описание базовых правил кодирования для 
абстрактно-синтаксической нотации версии 1 (АСН.1) 

ГОСТ~27463-87 Системы обработки информации. 7-битные кодированные наборы 
символов 

\begin{note*}
При пользовании настоящим стандартом целесообразно проверить действие ТНПА по
каталогу, составленному по состоянию на 1 января текущего года, и по
соответствующим информационным указателям, опубликованным в текущем году.

Если ссылочные ТНПА заменены (изменены), то при пользовании настоящим стандартом
следует руководствоваться замененными (измененными) ТНПА. Если ссылочные ТНПА
отменены без замены, то положение, в котором дана ссылка на них, применяется в
части, не затрагивающей эту ссылку.
\end{note*}

