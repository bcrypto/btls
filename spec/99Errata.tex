\clearpage
\chapter*{\mbox{}\hfill Поправка к официальной редакции\hfill\mbox{}}

\mbox{}

\begin{center}
\begin{tabular}{|p{3.5cm}|p{6cm}|p{6cm}|}
\hline
В каком месте & Напечатано & Должно быть\\
\hline
\hline
Приложение~\ref{BSUITES},\par
таблица~\ref{Table.BSUITES.1},\par
вторая колонка
&
\{192, 21\}\par
\{192, 22\}\par
\{192, 23\}\par
\{192, 24\}\par
\{192, 25\}\par
\{192, 26\}\par
\{192, 27\}\par
\{192, 28\}
&
\{255, 21\}\par
\{255, 22\}\par
\{255, 23\}\par
\{255, 24\}\par
\{255, 25\}\par
\{255, 26\}\par
\{255, 27\}\par
\{255, 28\}
\\
\hline
Приложение~\ref{BSUITES},\par
пункт~\ref{BSUITES.1},\par
последний абзац
&
Во всех криптонаборах используется один и тот же алгоритм генерации
псевдослучайных чисел (см.~В.2.4). Отличаются алгоритмы шифрования (см.~В.2.1),
алгоритмы имитозащиты (см.~В.2.2) и алгоритмы формирования общего ключа
(см.~В.2.5). Алгоритмы шифрования и имитозащиты могут быть совмещены
(см.~В.2.3).
&
Во всех криптонаборах используется один и тот же алгоритм генерации
псевдослучайных чисел (см.~\ref{BSUITES.2.2}). Отличаются алгоритмы шифрования и
имитозащиты (см.~\ref{BSUITES.2.1}), алгоритмы формирования общего ключа
(см.~\ref{BSUITES.2.3}).
\\
\hline
\multirow{3}{*}{\shortstack[l]{
Приложение~\ref{BSUITES},\\
пункт~\ref{BSUITES.2},\\[3pt]
заголовки}}
&
В.2.3~Одновременное шифрование и имитозащита
&
В.2.1.3~Одновременное шифрование и имитозащита
\\
\cline{2-3}
&
В.2.4~Алгоритм генерации псевдослучайных чисел
&
В.2.2~Алгоритм генерации псевдослучайных чисел
\\
\cline{2-3}
&
В.2.5~Алгоритмы формирования общего ключа
&
В.2.3~Алгоритмы формирования общего ключа
\\
\cline{2-3}
&
В.2.5.1~Алгоритм DHE\_BIGN
&
В.2.3.1~Алгоритм DHE\_BIGN
\\
\cline{2-3}
&
В.2.5.2~Алгоритм DHT\_BIGN
&
В.2.3.2~Алгоритм DHT\_BIGN
\\
\cline{2-3}
&
В.2.5.3~Алгоритм\par DHE\_PSK\_BIGN
&
В.2.3.3~Алгоритм\par DHE\_PSK\_BIGN
\\
\cline{2-3}
&
В.2.5.4~Алгоритм\par DHT\_PSK\_BIGN
&
В.2.3.4~Алгоритм\par DHT\_PSK\_BIGN
\\
\hline
Приложение~\ref{BSUITES},\par
пункт~\ref{BSUITES.3.2} 
&
\lstinline|ClientCertificateType +=|\par
\lstinline| {bign128_auth(121),|\par
\lstinline|  bign192_auth(122),|\par
\lstinline|  bign256_auth(123)};|
&
\lstinline|ClientCertificateType +=|\par
\lstinline| {bign128_auth(231),|\par
\lstinline|  bign192_auth(232),|\par
\lstinline|  bign256_auth(233)};|
\\
\hline
Приложение~\ref{BSUITES},\par
пункт~\ref{BSUITES.3.3} 
&
\lstinline|HashAlgorithm +=|\par
\lstinline| {belt_hash(121)};|\par
\lstinline|SignatureAlgorithm +=|\par
\lstinline| {bign_sign(121)};|
&
\lstinline|HashAlgorithm +=|\par
\lstinline| {belt_hash(231)};|\par
\lstinline|SignatureAlgorithm +=|\par
\lstinline| {bign_sign(231)};|
\\
\hline
\end{tabular}
\end{center}

\vfill

\mbox{}

\begin{center}
\begin{tabular}{|p{3.5cm}|p{6cm}|p{6cm}|}
\hline
В каком месте & Напечатано & Должно быть\\
\hline
%
\hline
Приложение~\ref{EXTS} &
Отсутствует &
Добавлено
\\
%
\hline
Библиография, ссылка~\cite{RFC6961} &
&
The Transport Layer Security (TLS) Multiple Certificate Status 
Request Extension\par
{\small (Расширение TLS для нескольких запросов о статусе сертификатов)}
\\
%
\hline
Библиография, ссылка~\cite{RFC7627} &
&
RFC 7627:2015
Transport Layer Security (TLS) Session Hash and Extended Master Secret 
Extension\par
{\small (Расширение TLS для хэширования сеансовых данных и построения 
расширенного мастер-ключа)}
\\
%
\hline
Библиография, ссылка~\cite{RFC7366} &
&
RFC 7366:2014
Encrypt-then-MAC for Transport Layer Security (TLS) and Datagram Transport 
Layer Security (DTLS)\par
{\small (Расширение TLS и DTLS для переключения на схему 
Зашифрование-затем-имитозащита)}
\\
\hline
\end{tabular}
\end{center}