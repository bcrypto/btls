\chapter{Протокол Record}\label{RECORD}

\section{Общие сведения}\label{RECORD.1}

Протокол Record получает данные для передачи, разбивает их на фрагменты,
при необходимости сжимает, вычисляет имитовставку, зашифровывает и передает
полученный результат. После получения данные расшифровываются, проверяется
их целостность, при необходимости данные преобразуются из сжатой формы в
исходную, объединяются и доставляются протоколам верхних уровней.

Протокол Record используется протоколами Handshake, Change~Cipher~Spec и
Alert, которые описываются в следующих разделах, а также прикладными
протоколами. Каждый фрагмент данных при передаче протоколом Record
дополняется полями, которые указывают на тип содержимого и длину фрагмента.

Тип содержимого является элементом перечисления
\begin{lstlisting}
enum {
  change_cipher_spec(20), alert(21), handshake(22),
  application_data(23), (255)
} ContentType;
\end{lstlisting}

Значения \lstinline{change_cipher_spec}, \lstinline{alert}, \lstinline{handshake}
и \lstinline{application_data} указывают на то, что данные отправлены
протоколами Change~Cipher~Spec, Alert, Handshake и прикладным
протоколом соответственно.

Перечисление \lstinline{ContentType} может быть расширено в будущем. Но пока 
реализации TLS НЕ ДОЛЖНЫ посылать данные, типы содержимого которых отличны 
от определенных выше. Если сторона TLS получает сообщение с содержимым 
неопределенного типа, то она ДОЛЖНА выслать критическое сигнальное 
сообщение \lstinline{unexpected_message} (см.~\ref{ALERT.3}). 

Любой протокол, предназначенный для использования поверх TLS, должен быть 
тщательно продуман на предмет защиты от возможных атак. Это значит, что 
разработчик протокола верхнего уровня должен быть осведомлен о том, какие 
механизмы безопасности поддерживает протокол TLS, а какие нет, и не 
полагаться на последние. В частности, разработчик должен учитывать, что в 
TLS тип содержимого и длина фрагмента не защищаются шифрованием. Поэтому, 
если эти данные являются критическими, разработчику следует принять меры 
для минимизации утечки информации при передаче (например, следует 
дополнять фрагменты неиспользуемыми байтами или маскировать трафик). 

\section {Состояния соединения}\label{RECORD.2}

Логически соединение TLS описывается четырьмя состояниями: имеются текущие 
(активные) состояния чтения и записи, а также ожидаемые состояния чтения и 
записи. Состояние~--- это структура данных, которая определяет используемые 
алгоритмы сжатия, шифрования и имитозащиты, а также состояния и параметры 
этих алгоритмов. В том числе состояние определяет ключи шифрования и 
имитозащиты, которые используются или будут использоваться для защиты 
соединения в одном из направлений <<чтения или записи>>.  

Протокол Record обрабатывает отправляемые данные с использованием текущего 
состояния записи, а принимаемые~--- с использованием текущего состояния 
чтения. Начальные текущие состояния устанавливаются пустыми. В этих 
состояниях шифрование, сжатие и имитозащита не используются. Пустое 
состояние соответствует криптонабору \lstinline{TLS_NULL_WITH_NULL_NULL}. 

Состояния чтения и записи разделяют общие параметры защиты. Параметры 
защиты для ожидаемых состояний устанавливаются протоколом Handshake. 
Протокол Change Cipher Spec может переводить ожидаемое состояние в 
текущее с переносом параметров защиты, при этом ранее установленное 
текущее состояние заменяется ожидаемым, а ожидаемое состояние сбрасывается 
в пустое. Если для состояния не установлены параметры защиты, то его 
нельзя сделать текущим. 

Параметры защиты описываются следующей структурой:
\begin{lstlisting}
struct {
  ConnectionEnd          entity;
  PRFAlgorithm           prf_algorithm;
  BulkCipherAlgorithm    bulk_cipher_algorithm;
  CipherType             cipher_type;
  uint8                  enc_key_length;
  uint8                  block_length;
  uint8                  fixed_iv_length;
  uint8                  record_iv_length;
  MACAlgorithm           mac_algorithm;
  uint8                  mac_length;
  uint8                  mac_key_length;
  CompressionMethod      compression_algorithm;
  opaque                 master_secret[48];
  opaque                 client_random[32];
  opaque                 server_random[32];
} SecurityParameters;
\end{lstlisting}

Поля структуры \lstinline{SecurityParameters} имеют следующее значение:
\begin{itemize}
\item[--] 
\lstinline{entity}~--- сторона соединения (клиент или сервер);

\item[--]
\lstinline{prf_algorithm}~--- алгоритм генерации псевдослучайных чисел;

\item[--]
\lstinline{bulk_cipher_algorithm}~--- алгоритм шифрования (зашифрования 
или расшифрования); 

\item[--]
\lstinline{cipher_type}~--- тип алгоритма шифрования;

\item[--]
\lstinline{enc_key_length}~--- длина ключа алгоритма шифрования; 

\item[--]\lstinline{block_length}~--- длина блока алгоритма шифрования 
(только для алгоритмов блочного шифрования);

\item[--]\lstinline{fixed_iv_length}~--- длина неявной части 
синхропосылки, используемой при шифровании; 

\item[--]
\lstinline{record_iv_length}~--- длина явной части синхропосылки, 
используемой при шифровании; 

\item[--]
\lstinline{mac_algorithm}~--- алгоритм имитозащиты; 

\item[--]
\lstinline{mac_length}~--- длина имитовставки алгоритма имитозащиты;

\item[--]
\lstinline{mac_key_length}~--- длина ключа алгоритма имитозащиты;

\item[--]
\lstinline{compression_algorithm}~--- алгоритм, который используется для 
сжатия данных; 

\item[--]
\lstinline{master_secret}~--- мастер-ключ, который является общим для двух 
сторон соединения;  
 
\item[--]
\lstinline{client_random}~--- случайное число клиента;

\item[--]
\lstinline{server_random}~--- случайное число сервера.
\end{itemize}

Используемые в \lstinline{SecurityParameters} типы 
\lstinline{PRFAlgorithm}, \lstinline{BulkCipherAlgorithm}, 
\lstinline{CipherType}, \lstinline{MACAlgorithm} описаны выше. Оставшиеся 
типы определяются следующим образом: 
\begin{lstlisting}
enum {server, client} ConnectionEnd;
enum {null(0), (255)} CompressionMethod;
\end{lstlisting}

Элементы \lstinline{server} и \lstinline{client} перечисления 
\lstinline{ConnectionEnd} соответствуют серверу и клиенту. Элемент 
\lstinline{null} перечисления CompressionMethod соответствует <<пустому>> 
алгоритму сжатия, который не изменяет поступающие на его вход данные. 

Протокол Record использует параметры защиты для формирования следующих объектов:
\begin{itemize}
\item[--]
\lstinline{client_write_MAC_key}~--- ключ имитозащиты клиента;

\item[--]
\lstinline{server_write_MAC_key}~--- ключ имитозащиты сервера;

\item[--]
\lstinline{client_write_key}~--- ключ шифрования клиента;

\item[--]
\lstinline{server_write_key}~--- ключ шифрования сервера;

\item[--]
\lstinline{client_write_IV}~--- неявная часть синхропосылки клиента;

\item[--]
\lstinline{server_write_IV}~--- неявная часть синхропосылки сервера.
\end{itemize}
                                                            
Объекты строятся по параметрам защиты с помощью алгоритма, определенного 
в~\ref{CRYPTO.2.3}. Не все объекты могут быть задействованы и поэтому 
некоторые из них могут быть пустыми. Ключи и синхропосылки клиента 
используются клиентом при отправке данных, а сервером при получении 
данных, и наоборот.  

После того, как параметры защиты согласованы, необходимые ключи и 
синхропосылки построены, ожидаемые состояния соединения могут быть 
переведены в текущие. Смена состояний ДОЛЖНА быть учтена при обработке 
всех следующих фрагментов данных. 

Каждое состояние соединения включает следующие элементы:
\begin{itemize}
\item[--] 
состояние алгоритма сжатия;

\item[--] 
состояние алгоритма шифрования, в том числе ключ шифрования для данного 
соединения. Для алгоритма поточного шифрования в этом состоянии сохраняется 
вся информация, необходимая для продолжения шифрования следующих фрагментов 
данных; 

\item[--]
состояние алгоритма имитозащиты, в том числе ключ имитозащиты;

\item[--]
порядковый номер \lstinline{seq_num}.
\end{itemize}

Для состояний чтения и записи должны поддерживаться независимые порядковые 
номера. Порядковый номер ДОЛЖЕН сбрасываться в 0 всякий раз, когда 
состояние соединения становится текущим, и увеличиваться на единицу после 
обработки (передачи или приема) каждого фрагмента данных. Порядковый 
номер является числом типа \mbox{\lstinline{uint64}~и} не может превышать 
значения $2^{64}-1$. Порядковые номера не должны повторяться. Поэтому при 
достижении максимального порядкового номера реализации TLS должны 
переустанавливать соединение. 

\section{Выполнение протокола}\label{RECORD.3}

\subsection{Фрагментация}\label{RECORD.3.1}

Протокол Record принимает неструктурированные данные от протоколов 
верхнего уровня непустыми блоками произвольной длины. Протокол 
преобразует блоки в структуры типа \lstinline{TLSPlaintext}, содержащие фрагменты из 
не более чем $2^{14}$ байтов. Протокол Record не обязательно сохраняет размер 
поступающих блоков, т. е. несколько блоков данных, полученных от одного и 
того же протокола верхнего уровня, МОГУТ быть объединены в один фрагмент, 
или, наоборот, один блок МОЖЕТ быть разбит на несколько фрагментов. 
Тип \lstinline{TLSPlaintext} определяется следующим образом:
\begin{lstlisting}
struct {
  ContentType type;
  ProtocolVersion version;
  uint16 length;
  opaque fragment[TLSPlaintext.length];
} TLSPlaintext;
\end{lstlisting}

Поля структуры \lstinline{TLSPlaintext} имеют следующее значение:
\begin{itemize}
\item[--]
\lstinline{type}~--- тип содержимого передаваемых данных (см.~\ref{RECORD.1});

\item[--]
\lstinline{version}~--- используемая версия TLS;

\item[--]
\lstinline{length}~--- длина в байтах поля 
\lstinline{TLSPlaintext.fragment}. Длина НЕ ДОЛЖНА быть больше $2^{14}$; 

\item[--]
\lstinline{fragment}~--- фрагмент данных от протокола верхнего уровня, 
заданного полем \lstinline{TLSPlaintext.type}. 
\end{itemize}

Тип \lstinline{ProtocolVersion} определяется следующим образом:
\begin{lstlisting}
struct {
  uint8 major;
  uint8 minor;
} ProtocolVersion;
\end{lstlisting}

Поля \lstinline{major} и \lstinline{minor} задают старшую и младшую части 
номера версии протокола. Настоящий стандарт определяет протокол TLS версии 
1.2, которому соответствует значение \lstinline|{3, 3}|. 

Реализации TLS НЕ ДОЛЖНЫ посылать фрагменты нулевой длины с типом 
содержимого \lstinline{change_cipher_spec}, \lstinline{alert} и 
\lstinline{handshake}. Фрагменты нулевой длины с типом содержимого 
\lstinline{application_data} МОГУТ высылаться, например, для  
анализа помех в канале связи. 

Фрагменты с различными типами содержимого МОГУТ чередоваться. Фрагменты 
прикладных протоколов, как правило, имеют более низкий приоритет для 
передачи по сравнению с фрагментами других типов. Тем не менее фрагменты 
ДОЛЖНЫ доставляться на сетевой уровень в том же порядке, в каком к ним 
применялась защита протоколом Record. Если сторона получает фрагмент 
прикладного протокола во время выполнения Handshake, то этот фрагмент 
ДОЛЖЕН обрабатываться с использованием параметров защиты, установленных по 
завершении предыдущего сеанса Handshake. 
                                                      
\subsection {Сжатие и восстановление сжатых данных}\label{RECORD.3.2} 

Все фрагменты сжимаются с помощью алгоритма сжатия, определенного в 
текущем состоянии сеанса. Всегда существует активный алгоритм сжатия. 
Первоначально он определяется как <<пустой>> и задается идентификатором 
\lstinline{CompressionMethod.null}.  
Алгоритм сжатия преобразует структуру типа \lstinline{TLSPlaintext} в 
структуру типа \lstinline{TLSCompressed}:
\begin{lstlisting}
struct {
  ContentType type;
  ProtocolVersion version;
  uint16 length;
  opaque fragment[TLSCompressed.length];
} TLSCompressed;
\end{lstlisting}

Поля структуры \lstinline{TLSCompressed} имеют следующее значение:
\begin{itemize}
\item[--]
\lstinline{type}~--- поле, аналогичное \lstinline{TLSPlaintext.type};

\item[--]
\lstinline{version}~--- поле, аналогичное \lstinline{TLSPlaintext.version};

\item[--]
\lstinline{length}~--- длина в байтах поля 
\lstinline{TLSCompressed.fragment}. Длина НЕ ДОЛЖНА быть больше, чем 
$2^{14} + 1024$; 
\item[--]
\lstinline{fragment}~--- cжатое поле \lstinline{TLSPlaintext.fragment}.
\end{itemize}

Состояние алгоритма сжатия инициализируются значениями по умолчанию в тот 
момент, когда состояние соединения становится активным. 

Сжатие должно выполняться без потери информации. При сжатии длина 
содержимого не может увеличиваться (за счет дополнительных заголовков, 
таблиц сжатия и др.) более чем на 1024 байтов. Если объем данных, 
восстановленных из \lstinline{TLSCompressed.fragment}, превысит $2^{14}$ 
байтов, то ДОЛЖНО быть отправлено критическое сигнальное сообщение 
\lstinline{decompression_failure} (см.~\ref{ALERT.3}). 

Восстановление сжатых данных ДОЛЖНО быть реализовано так, чтобы 
восстановление не могло привести к переполнению внутренних буферов памяти. 

\begin{note}
Некоторые алгоритмы сжатия для TLS приводятся в~\cite{RFC3749}. 
\end{note}

\begin{note}
Известны атаки на прикладные протоколы, в которых злоумышленник знает формат
фрагмента открытых данных и даже частично управляет его содержимым.
Злоумышленник использует уровень сжатия (разницу между
\lstinline{TLSPlaintext.length} и \lstinline{TLSCompressed.length}) для
определения недостающих частей фрагмента. Следует учитывать возможность таких
атак при проектировании пакетов прикладных протоколов и планировании сжатия в
TLS.
\end{note}
 
\subsection{Защита данных}\label{RECORD.3.3}
    
Алгоритмы зашифрования и имитозащиты преобразуют структуру типа 
\lstinline{TLSCompressed} в структуру типа \lstinline{TLSCiphertext}. 
Алгоритм расшифрования выполняет обратный процесс. 

Тип \lstinline{TLSCiphertext} определяется следующим образом:

\begin{lstlisting}
struct {
  ContentType type;
  ProtocolVersion version;
  uint16 length;
  select (SecurityParameters.cipher_type) {
    case stream: GenericStreamCipher;
    case block:  GenericBlockCipher;
    case aead:   GenericAEADCipher;
  } fragment;
} TLSCiphertext;
\end{lstlisting}
 
Поля структуры \lstinline{TLSCiphertext} имеют следующее значение:
\begin{enumerate} 
\item[--]
\lstinline{type}~--- поле, аналогичное \lstinline{TLSCompressed.type};

\item[--]
\lstinline{version}~--- поле, аналогичное \lstinline{TLSCompressed.version};

\item[--]
\lstinline{length}~--- длина в байтах поля 
\lstinline{TLSCiphertext.fragment}. Длина не должна быть больше, чем 
$2^{14} + 2048$; 

\item[--]
\lstinline{fragment}~--- зашифрованное поле 
\lstinline{TLSCompressed.fragment} вместе с имитовставкой. 
\end{enumerate}
                
\subsubsection{Поточное шифрование}\label{RECORD.3.3.1}

Алгоритмы поточного шифрования преобразуют структуру 
\lstinline{TLSCompressed.fragment} в структуру 
\lstinline{TLSCiphertext.fragment} типа \lstinline{GenericStreamCipher} и 
наоборот.   

Тип \lstinline{GenericStreamCipher} определяется следующим образом:
\begin{lstlisting}
stream-ciphered struct {
  opaque content[TLSCompressed.length];
  opaque MAC[SecurityParameters.mac_length];
} GenericStreamCipher;
\end{lstlisting}

Поля структуры TLSCompressed имеют следующее значение:
\begin{itemize}
\item[--] 
\lstinline{content}~--- поле, аналогичное 
\lstinline{TLSCiphertext.fragment}; 

\item[--]
\lstinline{MAC}~--- имитовставка, вычисленная с помощью алгоритма, 
заданного полем \lstinline{SecurityParameters.mac_algorithm}. 
\end{itemize}

Имитовставка вычисляется на ключе имитозащиты от составной строки 
\begin{lstlisting}
seq_num + TLSCompressed.type + TLSCompressed.version + 
  TLSCompressed.length + TLSCompressed.fragment, 
\end{lstlisting}
где \lstinline{seq_num}~--- порядковый номер фрагмента. Имитовставка 
вычисляется до зашифрования отправляемого фрагмента и после расшифрования 
принятого.  

Зашифровывается составная строка \lstinline{content + mac}. Длина 
\lstinline{TLSCiphertext.length}  равна сумме значений полей 
\lstinline{TLSCompressed.length} и  \lstinline{SecurityParameters.mac_length}.  

\subsubsection{Блочное шифрование}\label{RECORD.3.3.2} 

Алгоритмы блочного шифрования используются в режиме сцепления блоков. 
Они преобразуют структуру \lstinline{TLSCompressed.fragment} в структуру 
\lstinline{TLSCiphertext.fragment} типа \lstinline{GenericBlockCipher} 
и наоборот. 

Тип \lstinline{GenericBlockCipher} определяется следующим образом:
\begin{lstlisting}
struct {
  opaque IV[SecurityParameters.record_iv_length];
  block-ciphered struct {
    opaque content[TLSCompressed.length];
    opaque MAC[SecurityParameters.mac_length];
    uint8 padding[GenericBlockCipher.padding_length];
    uint8 padding_length;
  };
} GenericBlockCipher;
\end{lstlisting}

Поля структуры \lstinline{GenericBlockCipher} имеют следующее значение:
\begin{itemize}
\item[--]
\lstinline{IV}~--- синхропосылка, которую СЛЕДУЕТ выбирать случайно и которая ДОЛЖНА 
быть непредсказуемой. Для алгоритмов блочного шифрования длина 
синхропосылки определяется полем \lstinline{SecurityParameters.record_iv_length}, 
значение которого ДОЛЖНО совпадать со значением 
поля~\lstinline{SecurityParameters.block_size}; 

\item[--] 
\lstinline{content}~--- поле, аналогичное \lstinline{TLSCompressed.fragment};

\item[--]
\lstinline{MAC}~--- имитовставка, которая вычисляется в соответствии 
с~\ref{RECORD.3.3.1};

\item[--]
\lstinline{padding}~--- дополнение, которое добавляется к открытым данным для 
выравнивания их на границу блока алгоритма шифрования. Дополнение МОЖЕТ 
быть любой длины, не превосходящей 255, при этом длина дополнения ДОЛЖНА 
выбираться так, чтобы значение поля \lstinline{TLSCiphertext.length} 
(т.~е. размер структуры типа \lstinline{GenericBlockCipher}) было кратно 
длине блока алгоритма шифрования. Дополнения, размеры которых больше 
минимально необходимых, могут использоваться для предотвращения атак, 
основанных на анализе длин передаваемых сообщений. Каждый байт дополнения 
ДОЛЖЕН быть заполнен значением длины дополнения. Получатель ДОЛЖЕН 
проверить дополнение и ДОЛЖЕН отправить критическое сигнальное сообщение 
\lstinline{bad_record_mac} (см.~\ref{ALERT.3}), если дополнение 
некорректно;

\item[--]
\lstinline{padding_length}~--- длина дополнения в байтах. Может принимать 
значения от 0 до 255 включительно.  
\end{itemize}

При блочном шифровании значение поля \lstinline{TLSCiphertext.length} на 
единицу больше, чем сумма значений полей 
\lstinline{SecurityParameters.block_length},  
\lstinline{TLSCompressed.length}, 
\lstinline{SecurityParameters.mac_length} и \lstinline{padding_length}.  

\begin{example*}
Пусть длина блока (в байтах) алгоритма шифрования равна 8, 
длина содержимого (\lstinline{TLSCompressed.length}) равна 61, 
а длина имитовставки~--- 20. 
Тогда суммарная длина полей \lstinline{content}, \lstinline{MAC}, 
\lstinline{padding_length} равняется 82. Для того, чтобы общая длина 
зашифровываемых данных была кратна длине блока, необходимо выбрать 
дополнение, длина которого должна принимать одно из следующих значений: 6, 
14, 22, \ldots, 254. Дополнение минимальной длины будет  
состоять из 6 байтов, каждый из которых содержит значение 6. Таким 
образом, последние 8 байтов структуры \lstinline{GenericBlockCipher} до их 
зашифрования будут иметь следующий вид: XX 06 06 06 06 06 06 06, где XX~--- 
последний байт имитовставки (поле MAC).
\end{example*}

\begin{note}
При шифровании в режиме сцепления блоков весь фрагмент открытого текста ДОЛЖЕН
быть известен до передачи какой-либо части соответствующего шифртекста.
\end{note}

\begin{note}
В \cite{PwdInterception} описана атака на механизм дополнения данных перед
зашифрованием, основанная на замерах времени вычисления имитовставки
расшифрованных данных. Для защиты от данной атаки реализации ДОЛЖНЫ обрабатывать
зашифрованные фрагменты данных за одно и то же время, вне зависимости от того,
какое дополнение получено при расшифровании, корректное или нет. Для этого
рекомендуется вычислять имитовставку даже в том случае, когда дополнение
некорректно, и только после этого браковать фрагмент. Для защиты от атаки,
предложенной в~\cite{Lucky13}, рекомендуется и при корректном, и при некрректном
дополнении вычислять имитовставку за максимальное время. Это время, за которые
были бы выполнены вычисления, если бы дополнение имело нулевую длину.
\end{note}

\subsubsection{Одновременное шифрование и имитозащита}\label{RECORD.3.3.3}

Алгоритм установки защиты преобразует структуру 
\lstinline{TLSCompressed.fragment} в структуру 
\lstinline{TLSCiphertext.fragment} типа \lstinline{GenericAEADCipher}, а 
алгоритм снятия защиты выполняет обратное преобразование. 

Тип \lstinline{GenericAEADCipher} определяется следующим образом:
\begin{lstlisting}
struct {
  opaque nonce_explicit[SecurityParameters.record_iv_length];
  aead-ciphered struct {
    opaque content[TLSCompressed.length];
  };
} GenericAEADCipher;
\end{lstlisting}

При установке защиты клиент использует ключ шифрования \lstinline{client_write_key}, 
а сервер~--- \lstinline{server_write_key}. Ключи имитозащиты не используются. 
Каждый криптонабор, поддерживающий алгоритмы одновременного шифрования и 
имитозащиты, ДОЛЖЕН определять, каким образом формируется синхропосылка. 

Синхропосылка состоит из двух частей~--- явной, которая передается вместе с 
фрагментами данных, и неявной, которая определяется по параметрам защиты 
и не передается вместе с данными.   

Явная часть синхропосылки размещается в поле 
\lstinline{GenericAEAEDCipher.nonce_explicit}. Неявную часть СЛЕДУЕТ 
определять по строке \lstinline{client_write_iv} (при отправке данных 
клиентом) или \lstinline{server_write_iv} (при отправке данных сервером). 
Формирование этих строк описывается в~\ref{RECORD.4}.  

Критическими данными алгоритмов одновременного шифрования и имитозащиты 
является поле \lstinline{TLSCompressed.fragment}. Дополнительные открытые 
данные, обозначаемые \lstinline{additional_data}, определяются следующим 
образом:
\begin{lstlisting}
additional_data = seq_num + TLSCompressed.type + 
  TLSCompressed.version + TLSCompressed.length,
\end{lstlisting}
где \lstinline{seq_num}~--- порядковый номер фрагмента.

Длина выходных данных алгоритма установки защиты, как правило, больше, чем 
длина критических данных (т.~е. больше значения поля 
\lstinline{TLSCompressed.length}). Увеличение длины определяется спецификой 
используемого алгоритма. Так как алгоритмы одновременного шифрования и 
имитозащиты могут включать в себя механизмы выравнивания данных на 
границу блока, длина выходных данных алгоритма установки защиты может 
меняться в зависимости от значения поля \lstinline{TLSCompressed.length}. 
При установке защиты длина критических данных НЕ ДОЛЖНА увеличиваться 
более чем на 1024 байтов. 

Если при обработке данных алгоритм снятия защиты возвратил признак 
нарушения целостности данных, то ДОЛЖНО быть отправлено критическое 
сигнальное сообщение \lstinline{bad_record_mac} (см.~\ref{ALERT.3}). 

\section{Формирование ключей}\label{RECORD.4}

Протокол Record использует алгоритм генерации псевдослучайных чисел для 
формирования ключей текущего состояния соединения. С помощью этого 
алгоритма могут также формироваться синхропосылки алгоритмов 
одновременного шифрования и имитозащиты. 

При генерации используются параметры защиты (см.~\ref{RECORD.2}), выработанные 
сторонами по протоколу Handshake. По мастер-ключу и случайным данным 
клиента и сервера генерируется строка байтов \lstinline{key_block}: 
\begin{lstlisting}
key_block = PRF(SecurityParameters.master_secret, 
  "key expansion", SecurityParameters.server_random + 
     SecurityParameters.client_random);
\end{lstlisting}

Ключи и синхропосылки формируются по этой строке в следующем порядке 
(объекты нулевой длины считаются пустыми): 
\begin{itemize}
\item[--] 
ключ \lstinline{client_write_MAC_key} длины 
\lstinline{SecurityParameters.mac_key_length}; 

\item[--]
ключ \lstinline{server_write_MAC_key} длины 
\lstinline{SecurityParameters.mac_key_length}; 

\item[--]
ключ \lstinline{client_write_key} длины 
\lstinline{SecurityParameters.enc_key_length}; 

\item[--]
ключ \lstinline{server_write_key} длины 
\lstinline{SecurityParameters.enc_key_length}; 

\item[--]
синхропосылка \lstinline{client_write_IV} длины 
\lstinline{SecurityParameters.fixed_iv_length}; 

\item[--]
синхропосылка \lstinline{server_write_IV} длины 
\lstinline{SecurityParameters.fixed_iv_length}. 
\end{itemize}

Генерируется столько байтов \lstinline{key_block}, сколько требуется для 
построения всех ключей и синхропосылок. 


