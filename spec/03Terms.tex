\chapter{Термины и определения}\label{TERMS}

В настоящем стандарте применяют термины, установленные в СТБ~34.101.19, 
СТБ~34.101.31, СТБ~34.101.45 и СТБ~34.101.47, а также следующие термины с 
соответствующими определениями: 

{\bf \thedefctr~алгоритм сжатия}:
Алгоритм, который выполняет обратимое сжатие данных для уменьшения их 
размера при передаче по каналам связи; в настоящем стандарте алгоритм 
сжатия используется всегда вместе с алгоритмом восстановления сжатых 
данных. 

{\bf \thedefctr~аутентификация}:
Проверка подлинности стороны.

{\bf \thedefctr~клиент (client)}:
Сторона, которая инициирует выполнение протокола TLS.

{\bf \thedefctr~ключ имитозащиты клиента (client write MAC key)}:
Криптографический ключ, используемый для имитозащиты данных, отправляемых 
клиентом. 

{\bf \thedefctr~ключ имитозащиты сервера (server write MAC key)}:
Криптографический ключ, используемый для имитозащиты данных, отправляемых 
сервером. 

{\bf \thedefctr~ключ шифрования клиента (client write encryption key)}:
Криптографический ключ, используемый для шифрования данных, отправляемых 
клиентом. 

{\bf \thedefctr~ключ шифрования сервера (server write encryption key)}:
Криптографический ключ, используемый для шифрования данных, отправляемых 
сервером. 

{\bf \thedefctr~криптонабор (cipher suite)}:
Криптоопределение, дополненное алгоритмом формирования общего ключа, 
который используется в протоколе Handshake для построения предварительного 
мастер-ключа. 

{\bf \thedefctr~криптоопределение (cipher spec)}:
Точно определенный перечень алгоритмов шифрования, имитозащиты, генерации 
псевдослучайных чисел, которые используются в протоколе Record для 
обеспечения конфиденциальности и контроля целостности данных. 

{\bf \thedefctr~мастер-ключ (master secret)}:
Криптографический ключ, который включается в параметры защиты и 
используется для построения ключей соединений. 

{\bf \thedefctr~неявная часть синхропосылки (implicit nonce)}:
Часть синхропосылки, которая не передается вместе с обработанными на ней 
данными, а определяется по контексту, в котором обработка выполнялась 
(например, по состоянию соединения). 

{\bf \thedefctr~параметры защиты (security parameters)}:
Параметры, которые являются частью состояния сеанса и используются для 
формирования состояний соединения; создаются при согласовании параметров 
связи с использованием согласованного алгоритма формирования общего ключа 
и с применением согласованных методов аутентификации. 

{\bf \thedefctr~переустановка связи (renegotiate)}:
Согласование параметров связи, которое выполняется во время соединения и 
заканчивается привязкой к созданному ранее сеансу (возможно текущему), 
формированием по параметрам защиты этого сеанса новых состояний соединения 
и переходом к этим состояниям. 

{\bf \thedefctr~предварительный мастер-ключ (preliminary master secret)}:
Криптографический ключ, который формируется при согласовании параметров 
связи и используется для построения мастер-ключа. 

{\bf \thedefctr~прикладной протокол (application protocol)}:
Протокол, который выполняется поверх протокола TLS.

{\bf \thedefctr~протокол защиты транспортного уровня; протокол TLS 
(transport layer security protocol)}: 
Определеяемый в настоящем стандарте криптографический протокол, который 
обеспечивает взаимную аутентификацию сторон протокола, конфиденциальность 
и контроль целостности данных, передаваемых между сторонами на 
транспортном коммуникационном уровне. 

{\bf \thedefctr~протокол Alert}:
Протокол обмена сигнальными сообщениями; часть протокола TLS. 

{\bf \thedefctr~протокол Change Cipher Spec}:
Протокол оповещения о переходе к новым состояниям соединения после 
согласования параметров связи; часть протокола TLS. 

{\bf \thedefctr~протокол Handshake}:
Протокол установки, возобновления или переустановки связи; часть протокола 
TLS. 

{\bf \thedefctr~протокол Record}:
Протокол, обеспечивающий обратимое сжатие, конфиденциальность и контроль 
целостности данных, передаваемых на транспортном коммуникационном уровне; 
часть протокола TLS; используется протоколами Handshake, Change Cipher 
Spec и Alert. 

{\bf \thedefctr~сеанс (session)}:
Логическая связь между клиентом и сервером, которая описывается 
идентификатором, параметрами защиты и другими согласованными между 
сторонами данными, которые могут быть использованы в нескольких 
соединениях. 

{\bf \thedefctr~сервер (server)}:
Сторона, которая выполняет протокол TLS с клиентами по их запросам. 

{\bf \thedefctr~сигнальное сообщение (alert message)}:
Сообщение о закрытии соединения или сообщение о внештатной
ситуации во время выполнения протокола TLS. 

{\bf \thedefctr~согласование параметров связи (negotiate)}:
Согласование между клиентом и сервером алгоритма сжатия, криптонаборов, 
метода аутентификации, параметров защиты; часть протокола Handshake. 

{\bf \thedefctr~соединение (connection)}:
Cвязь между сторонами на транспортном коммуникационном уровне. 

{\bf \thedefctr~состояние соединения (connection state)}:
Набор параметров, определяющих способ обработки принимаемых или 
отправляемых в рамках соединения данных, в том числе порядковый номер 
обрабатываемого фрагмента данных, состояния алгоритмов криптоопределения, 
состояние алгоритма сжатия; состояние соединения формируется по 
параметрам защиты сеанса. 

{\bf \thedefctr~сокращенная установка связи; возобновление связи 
(abbreviated handshake; session resume)}: 
Согласование параметров связи, которое выполняется в начале соединения и 
заканчивается привязкой к созданному ранее сеансу, формированием по 
параметрам защиты этого сеанса новых состояний соединения и переходом к 
этим состояниям. 

{\bf \thedefctr~установка связи (handshake; full handshake)}:
Согласование параметров связи, которое выполняется в начале соединения и 
заканчивается созданием нового сеанса, формированием по параметрам защиты 
этого сеанса новых состояний соединения и переходом к этим состояниям. 

{\bf \thedefctr~фрагмент (record)}:
Порция данных, отправляемых или принимаемых на транспортном 
коммуникационном уровне. 

{\bf \thedefctr~эфемерный ключ (ephemeral key)}:
Криптографический ключ, который генерируется, используется и уничтожается 
при установке связи. 

{\bf \thedefctr~явная часть синхропосылки (explicit nonce)}:
Часть синхропосылки, которая передается вместе с обработанными на ней 
данными. 


