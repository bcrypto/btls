\chapter{Общие положения}\label{COMMON}

Настоящий стандарт определяет криптографический протокол, предназначенный 
для защиты соединений между клиентом и сервером в сети Интернет. 
Данный протокол соответствует спецификации~\cite{RFC5246}, 
ее расширениям~\cite{RFC4279}, \cite{RFC5746}  и, следуя этим документам, 
обозначается TLS. Действия сторон протокола и форматы пересылаемых между 
сторонами сообщений определяются с такой степенью детализации, которая 
позволяет разрабатывать полностью совместимые между собой реализации TLS. 

TLS обеспечивает аутентификацию сторон протокола, конфиденциальность и 
контроль целостности передаваемых между сторонами данных. TLS 
встраивается в стек коммуникационных протоколов поверх транспортного 
уровня и обеспечивает защиту данных этого уровня. TLS выполняется 
независимо от протоколов верхнего уровня и прозрачен для них. 

Для организации защиты используются криптографические алгоритмы, которые 
оформляются в виде криптонаборов. В TLS предусмотрена возможность 
расширения перечня криптонаборов. Расширяемость TLS избавляет от 
необходимости создавать и реализовывать новый протокол для каждого нового 
набора криптографических алгоритмов. В приложении~\ref{BSUITES} определены 
криптонаборы, основанные на криптографических алгоритмах действующих ТНПА. 
  
В TLS предусмотрена возможность сохранения состояний сеансов, что 
позволяет восстанавливать связь между сторонами, а не устанавливать ее 
каждый раз заново. Возможность сохранения состояний сеансов снижает 
нагрузку на сервер и уменьшает объем пересылаемых между сторонами данных. 

TLS является объединением нескольких субпротоколов, разбитых на два 
уровня. На нижнем уровне действует протокол Record, который обеспечивает 
защищенный транспорт данных, поступающих от прикладных протоколов. На 
верхнем уровне действуют протоколы Handshake, Change~Cipher~Spec, Alert и 
прикладные протоколы. 

Протокол Record обеспечивает конфиденциальность и контроль целостности 
транспортируемых данных. Для обеспечения конфиденциальности используются 
симметичные алгоритмы шифрования, а для контроля целостности~--- алгоритмы 
имитозащиты. Для каждого соединения стороны вырабатывают уникальные общие 
ключи шифрования и имитозащиты. Ключи строятся по секретным данным, 
согласованным с помощью других протоколов (как правило, Handshake). 
Протокол Record может выполняться без шифрования и имитозащиты. Однако 
режимы, обеспечивающие и конфиденциальность, и контроль целостности, 
являются основными. 

Протокол Handshake позволяет клиенту и серверу аутентифицировать друг 
друга, а также согласовать криптографические алгоритмы и общие ключи до 
того, как прикладной протокол начнет прием или передачу данных. 
Аутентификация сторон, как правило, производится с помощью асимметричных 
алгоритмов (алгоритмов с открытым ключом). Аутентификация может быть 
необязательной, но, как правило, хотя бы одна сторона проверяет 
подлинность другой. Общие ключи согласуются так, чтобы их не мог 
определить злоумышленник, который перехватывает все сообщения протокола. 
Более того, в режимах с аутентификацией общие ключи недоступны 
злоумышленнику, даже если он выдает себя за одну из сторон протокола. 
Протокол Handshake строится так, что всякое изменение пересылаемых между 
сторонами данных будет выявлено этими сторонами. 

Протокол Change~Cipher~Spec сообщает о смене параметров защиты на новые, 
согласованные при выполнении протокола Handshake. 

Протокол Alert извещает о закрытии соединений и об ошибках, произошедших 
при выполнении TLS.  


                          

