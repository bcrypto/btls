\thispagestyle{empty}

\noindent
{\bf ГОСУДАРСТВЕННЫЙ СТАНДАРТ} \hfill {\bf\draftlogo}\\
\noindent
{\bf РЕСПУБЛИКИ~БЕЛАРУСЬ}\\[-9pt]
\hrule height 1pt
\vskip0.4mm
\hrule height 2pt

\vskip2cm
\noindent
{\bf\Large Информационные технологии и безопасность}\\[10pt]
{\bf\large ПРОТОКОЛ ЗАЩИТЫ ТРАНСПОРТНОГО УРОВНЯ (TLS)}\\

\vskip2cm
\noindent
{\bf\Large Iнфармацыйныя тэхналогii i бяспека}\\[10pt]
{\bf\large ПРАТАКОЛ АХОВЫ ТРАНСПАРТНАГА ЎЗРОЎНЮ (TLS)}\\ 

\vskip9cm
\hrule height 1pt
\vskip0.4mm
\hrule height 2pt
\noindent
\begin{tabular}{p{5cm}cp{4cm}}
\vtop{\null\hbox{{\includegraphics[width=2.6cm]{../figs/stb}}}} & \hspace{6cm} & 
\mbox{}\newline\mbox{}\newline\newline Госстандарт\newline Минск\\
\end{tabular}

\pagebreak


\hrule
\vskip2mm

УДК~004.056.5.057.4(083.74)(476)\hfill
МКС~35.240.01\hfill
КП~05

\vskip0.5mm
 
{\bf Ключевые слова}: транспортный уровень, криптографический протокол, 
шифрование, контроль целостности, аутентификация

\vskip0.5mm

\hrule

\rule{0pt}{5mm}

\centerline{\bf Предисловие} 

Цели, основные принципы, положения по государственному регулированию и 
управлению в области технического нормирования и стандартизации 
установлены Законом Республики Беларусь <<О техническом нормировании и 
стандартизации>>.  

\vskip0.2cm

1~РАЗРАБОТАН учреждением Белорусского государственного университета 
<<Науч\-но-исследовательский  институт прикладных проблем математики и 
информатики>> (НИИ ППМИ)

ВНЕСЕН Оперативно-аналитическим центром при Президенте Республики Беларусь 

2~УТВЕРЖДЕН И ВВЕДЕН В ДЕЙСТВИЕ постановлением Госстандарта Республики 
Беларусь от 22 мая 2014 г.~\No~23

3~Настоящий стандарт разработан на основе документа RFC 5246:2008 The 
Transport Layer Security (TLS) Protocol. Version 1.2 (Протокол защиты 
транспортного уровня. Версия 1.2). Документ RFC 5246 разработан 
Специальной комиссией интернет-разработок (Internet Engineering Task 
Force)

4~ВВЕДЕН ВПЕРВЫЕ

\vfill
\hrule
\vskip1mm
Издан на русском языке

\pagebreak
