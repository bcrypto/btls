\begin{appendix}{А}{обязательное}
{Трактование ключевых слов}
\label{KEYWORDS}

\mbox{}

В настоящем приложении приводится разъяснение значений ключевых слов 
<<ДОЛЖЕН>>, <<НЕЛЬЗЯ>>, <<СЛЕДУЕТ>>, <<НЕ СЛЕДУЕТ>>, <<РЕКОМЕНДУЕТСЯ>> и <<МОЖЕТ>> 
используемых в настоящем стандарте. 

Ключевое слово <<ДОЛЖЕН>> означает, что действия, к которым применено данное 
ключевое слово, необходимо в точности выполнять. 

Ключевое слово <<НЕЛЬЗЯ>> выражает абсолютный запрет на выполнение 
соответствующих действий. 

Ключевые слова <<СЛЕДУЕТ>> и <<РЕКОМЕНДУЕТСЯ>> необходимо понимать так, что в 
некоторых случаях существует реальная причина их игнорировать, но 
последствия таких действий должны быть очевидными и хорошо взвешенными. 

Ключевое слово <<НЕ СЛЕДУЕТ>> употребляется в тех случаях, когда действие, к 
которому применено данное ключевое слово, будет в некоторых случаях 
правильным и даже полезным, однако при этом его последствия должны быть 
очевидными и хорошо взвешенными. 

Ключевое слово <<МОЖЕТ>> применяется к действиям (предметам), выполнение или 
невыполнение (наличие или отсутствие) которых не влияет на ситуацию в 
целом. Это означает, что программы, работающие с чем-то, помеченным 
данными ключевыми словами, должны учитывать обе ситуации и корректно их 
обрабатывать. 

Данные ключевые слова введены, в первую очередь, для выражения требований 
к действиям, которые влияют на безопасность и надежность рассматриваемых 
объектов, а также в интересах унификации последних. 

\end{appendix}

\mbox{}
\vfill
\mbox{}
\clearpage
