\chapter{Криптонаборы и методы аутентификации}\label{CRYPTO}

\section{Криптонаборы}\label{CRYPTO.1}

Криптонабор определяет:
\begin{itemize}
\item[--] 
алгоритмы шифрования (зашифрования и расшифрования), которые 
используются для обеспечения конфиденциальности данных, передаваемых между 
клиентом и сервером; 

\item[--] 
алгоритм имитозащиты, который используется для контроля целостности 
данных; 

\item[--]
алгоритм генерации псевдослучайных чисел, который используется для 
генерации ключей и синхропосылок по мастер-ключу, а также для верификации 
сообщений; 

\item[--]
алгоритм формирования общего ключа, который используется для 
согласования или передачи предварительного мастер-ключа с последующим 
построением мастер-ключа. 
\end{itemize}

Включенные в криптонабор алгоритмы шифрования, имитозащиты и генерации 
псевдослучайных чисел в совокупности составляют криптоопределение. 
Каждому криптонабору назначается уникальный идентификатор типа 
\lstinline{CipherSuite}: 
\begin{lstlisting}
uint8 CipherSuite[2];
\end{lstlisting}

Идентификатор \lstinline|{0, 0}| зарезервирован для криптонабора 
\lstinline|TLS_NULL_WITH_NULL_NULL|: 
\begin{lstlisting}
CipherSuite TLS_NULL_WITH_NULL_NULL = {0, 0};
\end{lstlisting}

В этом криптонаборе все алгоритмы являются <<пустыми>> (обозначаются 
\lstinline{null}), т.~е. не выполняют никаких вычислений. Для <<пустых>> 
алгоритмов длины ключей, синхропосылок и имитовставок полагаются равными~0.  

\section{Алгоритмы криптонаборов}\label{CRYPTO.2}

\subsection{Алгоритмы шифрования}\label{CRYPTO.2.1}

Для шифрования могут использоваться алгоритмы поточного шифрования, алгоритмы 
блочного шифрования или алгоритмы одновременного шифрования и имитозащиты. 
Тип алгоритма задается с помощью перечисления
\begin{lstlisting}
enum {stream, block, aead} CipherType;
\end{lstlisting}

Элемент \lstinline{stream} этого перечисления указывает на поточное шифрование, 
элемент \lstinline|block|~--- на блочное, 
элемент \lstinline|aead|~--- на одновременное шифрование и имитозащиту.

При поточном зашифровании открытый текст побитово 
суммируется с гаммой (двоичной последовательностью), полученной с помощью 
ключезависимого генератора псевдослучайных чисел. Гамма может 
вырабатываться для каждого отдельного фрагмента, передаваемого в рамках 
соединения, или для всех фрагментов сразу. В первом случае при генерации 
гаммы должна использоваться синхропосылка, уникальная для обрабатываемого 
фрагмента (например, номер фрагмента). Во втором случае состояние 
генератора гаммы должно сохраняться после зашифрования очередного 
фрагмента и использоваться при зашифровании следующего. 

При блочном зашифровании каждый блок открытого текста 
преобразуется в блок шифртекста. Шифрование выполняется в режиме 
сцепления блоков. Перед зашифрованием открытый текст дополняется 
незначащими байтами, чтобы получить строку байтов, длина которой кратна 
длине блока алгоритма. При зашифровании используется синхропосылка, 
которая передается вместе с шифртекстом. 

Алгоритмы одновременного шифрования и имитозащиты~--- это алгоритм установки 
защиты и алгоритм снятия защиты. Алгоритм установки защиты берет на вход ключ, 
синхропосылку, критические данные, для которых будет обеспечиваться 
шифрование и имитозащита, и открытые данные, для которых будет 
обеспечиваться только имитозащита. Алгоритм установки защиты возвращает 
зашифрованные критические данные и имитовставку открытых и критических 
данных. Алгоритм снятия защиты берет на вход ключ, синхропосылку, 
имитовставку, зашифрованные критические данные и открытые данные. 
Алгоритм снятия защиты либо расшифровывает критические данные, либо 
возвращает признак нарушения целостности данных. 

Допустимые алгоритмы шифрования задаются типом 
\lstinline{BulkCipherAlgorithm}: 
\begin{lstlisting}
enum {null,...} BulkCipherAlgorithm; 
\end{lstlisting}

Элемент \lstinline{null} соответствует <<пустому>> алгоритму шифрования, 
который не изменяет поступающие на его вход данные. Алгоритм 
\lstinline{null} классифицируется как алгоритм поточного шифрования. 

\subsection{Алгоритмы имитозащиты}\label{CRYPTO.2.2}

Допустимые алгоритмы имитозащиты задаются типом \lstinline{MACAlgorithm}: 
\begin{lstlisting}
enum {null,...} MACAlgorithm;
\end{lstlisting}

Элемент \lstinline{null} соответствует <<пустому>> алгоритму имитозащиты, 
который не вычисляет имитовставку. 

Если в качестве алгоритмов шифрования выбраны алгоритмы типа \lstinline{aead}, 
то в качестве алгоритма имитозащиты должен быть выбран \lstinline{null}. 

\subsection{Алгоритмы генерации псевдослучайных чисел}\label{CRYPTO.2.3}

Алгоритм генерации псевдослучайных чисел принимает на вход три параметра, 
которые обозначаются \lstinline{secret}, \lstinline{label} и \lstinline{seed}, и 
возвращает строку байтов требуемой длины, которая обозначается 
\lstinline{PRF(secret, label, seed)}. Входные параметры 
\lstinline{secret}, \lstinline{label} и \lstinline{seed}~--- это строки байтов 
произвольной длины, причем \lstinline{label} можно задать строкой символов 
(см.~\ref{SYNTAX.2}, приложение~\ref{SYNTAX}). 

Вычисление \lstinline{PRF(secret, label, seed)} состоит в обращении к 
алгоритму генерации псевдослучвйных чисел в режиме HMAC, определенному в 
СТБ~34.101.47 (подраздел~\ref{CRYPTO.3}), с передачей \lstinline{secret} в 
качестве ключа и \lstinline{label + seed} в качестве синхропосылки. 
Используемый алгоритм СТБ~34.101.47 строится на основе алгоритма HMAC, 
который, в свою очередь, строится на некотором алгоритме хэширования. 

Могут использоваться различные базовые алгоритмы хэширования. Используемый 
алгоритм ДОЛЖЕН однозначно определяться в криптонаборе. СЛЕДУЕТ 
использовать криптографически стойкие алгоритмы. 

Выходные данные \lstinline{PRF(secret, label, seed)} формируются блоками 
по~$n$ байтов, где~$n$~--- длина хэш-значения 
(определяется используемым алгоритмом хэширования). Для генерации~$m$ 
байтов следует сформировать~$k$ блоков, где~$k$~--- минимальное целое 
такое, что $nk\geq m$.  
При необходимости в последнем 
блоке следует отбросить последние байты. 

Допустимые алгоритмы генерации псевдослучайных чисел задаются типом 
\lstinline{PRFAlgorithm}: 
\begin{lstlisting}
enum {...} PRFAlgorithm;
\end{lstlisting}

\subsection{Алгоритмы формирования общего ключа}\label{CRYPTO.2.4}

Алгоритм формирования общего ключа является интерактивным. Это значит, что
его выполняют совместно клиент и сервер, обмениваясь между собой
сообщениями, которые содержат промежуточные результаты вычислений.
Сообщения алгоритма~--- это сообщения протокола Handshake, описанные 
в~\ref{HANDSHAKE.3}. По завершении алгоритма стороны формируют 
предварительный мастер-ключ, известный только им. По этому ключу и 
случайным данным, вырабатываемым каждой из сторон, строится окончательный 
мастер-ключ. 

В криптонаборе, вообще говоря, можно определить любой алгоритм 
формирования общего ключа. Тем не менее в основной спецификации~\cite{RFC5246} 
и ее расширении~\cite{RFC4279} описаны семь типов таких алгоритмов, 
которые покрывают большинство существующих на сегодняшний день решений. 
Для описания этих типов используются обозначения, близкие к обозначениям 
СТБ~34.101.45:  
$G$~--- элемент аддитивной (алгебраической) группы, который порождает 
циклическую группу $\langle G \rangle$ порядка $q$; 
$d_S, d_C \in \{1, 2,\ldots, q - 1\}$~--- личные ключи сервера и клиента 
соответственно;  
$Q_S = d_S G$, $Q_C = d_C G$~--- открытые ключи сторон. 

{\bf DH\_anon} (протокол Диффи~--~Хеллмана без аутентификации сторон). Сервер 
в сообщении \lstinline{ServerKeyExchange} переcылает клиенту описание группы 
$\langle G \rangle$ и свой открытый ключ $Q_S$. Клиент в сообщении 
\lstinline{ClientKeyExchange} пересылает серверу свой открытый ключ $Q_C$. 
Стороны вычисляют общий ключ $d_S d_C G = d_S Q_C = d_C Q_S$, по 
которому строится предварительный мастер-ключ. Сертификаты не 
используются. 

{\bf DH\_fixed} (протокол Диффи~--~Хеллмана cо статическим ключом). Сервер в 
сообщении \lstinline{Certificate} передает свой сертификат, который содержит 
описание $\langle G \rangle$ и открытый ключ $Q_{S}$. Сообщение 
\lstinline{ServerKeyExchange} не передается. Клиент в сообщении 
\lstinline{Certificate} (по запросу сервера) передает в своем сертификате 
статический (неизменяемый) открытый ключ $Q_{C}$. Если запроса от сервера 
нет, то клиент передает эфемерный (одноразовый) открытый ключ $Q_{C}$ в 
сообщении \lstinline{ClientKeyExchange}. Стороны вычисляют общий ключ 
$d_{S}d_{C}G$, по которому определяется предварительный мастер-ключ. 

{\bf DHE} (протокол Диффи~--~Хеллмана c эфемерными ключами). Сервер в 
сообщении \lstinline{Certificate} передает клиенту сертификат, открытый ключ 
которого можно использовать для проверки ЭЦП. Затем сервер в сообщении 
\lstinline{ServerKeyExchange} передает описание группы $\langle G \rangle$, 
свой эфемерный открытый ключ $Q_{S}$ и подписывает эти данные, а также 
случайные данные обеих сторон, на личном ключе, который соответствует 
переданному сертификату. Клиент проверяет ЭЦП и в сообщении 
\lstinline{ClientKeyExchange} передает свой открытый ключ $Q_{C}$. Стороны 
вычисляют общий ключ $d_{S}d_{C}G = d_{S}Q_{C} = d_{C}Q_{S}$, по которому 
строится предварительный мастер-ключ.  

{\bf T} (транспорт). Сервер в сообщении \lstinline{Certificate} передает свой 
сертификат, открытый ключ которого можно использовать для шифрования. 
Клиент выполняет на этом ключе зашифрование предварительного мастер-ключа 
и передает зашифрованный ключ в сообщении \lstinline{ClientKeyExchange}. 
Сервер выполняет расшифрование на своем личном ключе. Сообщение 
\lstinline{ServerKeyExchange} не передается. 

{\bf PSK} (на основе предварительного распределения секретов, от 
английского pre-shared key). Клиент и сервер предварительно распределяют 
между собой набор общих секретов. Клиент выбирает секрет из набора и в 
сообщении \lstinline{ClientKeyExchange} передает идентификатор выбранного 
секрета. Для помощи при выборе секрета сервер в 
\lstinline{ServerKeyExchange} может передать подсказку (например, номер 
секрета). Если подсказка не нужна, то сообщение 
\lstinline{ServerKeyExchange} не передается. Предварительный мастер-ключ 
строится по выбранному секрету. Сертификаты не используются.  

{\bf DHE\_PSK} (совмещение PSK и DHE). Сервер в сообщении 
\lstinline{ServerKeyExchange} передает описание группы $\langle G \rangle$ и 
свой эфемерный открытый ключ $Q_{S}$. Дополнительно сервер может передать 
в \lstinline{ServerKeyExchange} PSK-подсказку. Клиент в 
\lstinline{ClientKeyExchange} передает свой эфемерный открытый ключ $Q_{С}$ и 
идентификатор выбранного PSK-секрета. Стороны определяют общий ключ Диффи 
~--~Хеллмана $d_{S}d_{C}G = d_{S}Q_{C} = d_{C}Q_{S}$ и общий PSK-секрет. 
Предварительный мастер-ключ является результатом конкатенации этих общих 
секретных данных. Сертификаты не используются. 

{\bf T\_PSK} (совмещение PSK и транспорта). Дополнительно к сообщениям 
механизма PSK, сервер в сообщении \lstinline{Certificate} посылает клиенту 
свой сертификат, открытый ключ которого можно использовать для 
шифрования. Клиент выполняет на этом ключе зашифрование предварительного 
мастер-ключа и передает зашифрованный ключ в сообщении 
\lstinline{ClientKeyExchange} вместе с идентификатором выбранного 
PSK-секрета. Сервер выполняет расшифрование на своем личном ключе и 
объединяет полученный предварительный мастер-ключ с общим PSK-секретом. 

\begin{note}
Примечание 1~---~Описание группы $\langle G \rangle$ может 
задаваться явно или косвенно. Явное описание задается набором параметров, 
описывающих структуру группы, ее порядок, правила представления элементов 
и др. Клиент ДОЛЖЕН проверять
корректность присланного ему явного описания группы $\langle G \rangle$. 
Косвенное описание задается ссылкой на фиксированные параметры, известные 
клиенту и серверу, например, на параметры из ТНПА или из сертификата 
сервера.
\end{note}

\begin{note}
Примечание 2~--- В алгоритмах типа DH\_anon не используются ни 
сертификаты, ни общие секретные данные и, таким образом, не проверяется 
подлинность сторон. Поэтому алгоритмы типа DH\_anon не обеспечивают защиту 
от атак <<противник посередине>>, и их рекомендуется использовать только в 
специальных случаях.
\end{note}

\begin{note}
Примечание 3~---~В алгоритмах типа DH\_fixed, T при компрометации 
личного ключа сервера все сообщения предыдущих соединений TLS могут быть 
раскрыты. Данные алгоритмы не обеспечивают защиту от атак по <<чтению 
назад>> и их рекомендуется использовать только в специальных случаях. 
Защиту от <<чтения назад>> не обеспечивают также алгоритмы типа PSK (при 
компрометации PSK-секрета) и T\_PSK (при компрометации PSK-секрета и 
личного ключа сервера).
\end{note}
 
\begin{note}
Примечание 4~---~Алгоритмы типа PSK не обеспечивают защиту от 
словарных атак по подбору PSK-секрета со стороны злоумышленника, который 
перехватывает все сообщения протокола. Алгоритмы типа DHE\_PSK, Т\_PSK 
защищают от таких атак, но не обеспечивают защиту от словарных атак 
злоумышленника, который выдает себя за сервер (DHE\_PSK) или является 
таковым (T\_PSK) и пытается узнать PSK-секрет у клиента.
\end{note}

\begin{note}
Примечание 5~---~Если клиент $C$ взаимодействует с сервером $A$, 
контролируемым злоумышленником, и использует алгоритм формирования общего 
ключа типа T, то злоумышленник может провести атаку, описанную 
в~\cite{TripleHandshake}. Злоумышленник может организовать защищенное 
соединение между $C$ и другим сервером $S$,  при котором $С$ и $S$ будут 
считать, что взаимодействуют с $A$ и не смогут обнаружить, что 
взаимодействуют между собой. СЛЕДУЕТ учитывать возможность данной атаки 
при использовании алгоритмов типа T.  
\end{note}

\section{Методы аутентификации}\label{CRYPTO.3}

\subsection{Аутентификация сервера}\label{CRYPTO.3.1}

Аутентификация сервера основана, как правило, на проверке сертификата 
открытого ключа сервера и на проверке владения сервером соответствующим 
личным ключом. Успешное завершение протокола Handshake означает, что 
аутентификация завершена успешно: сертификат сервера действителен и 
сервер действительно владеет личным ключом. 
В алгоритмах формирования общего ключа на основе предварительно 
распределенных секретов проводится неявная аутентификация сервера, 
основанная на проверке владения PSK-секретом. Успешное завершение 
протокола Handshake означает, что сервер действительно владеет этим 
секретом. Если секрет распределялся по защищенным каналам только клиенту 
и серверу, то владение секретом доказывает клиенту подлинность сервера. 

\subsection{Аутентификация клиента}\label{CRYPTO.3.2}

Для аутентификации клиента сервер, как правило, запрашивает его сертификат 
в сообщении \lstinline{CertificateRequest}. Сертификат должен содержать 
открытый ключ определенного алгоритма ЭЦП. Сервер указывает в 
\lstinline{CertificateRequest} список подходящих типов открытых ключей. 
Данные типы задаются одним байтом и  
называются методами аутентификации клиента. 
Допустимые методы аутентификации клиента задаются типом 
\lstinline{ClientCertificateType}:
\begin{lstlisting}
enum {..., (255)} ClientCertificateType;
\end{lstlisting}

В ответ на запрос сервера клиент представляет сертификат одного из 
запрашиваемых типов и подписывает на личном ключе, соответствующем 
открытому ключу сертификата, определенные данные. При этом используется 
пара <<алгоритм хэширования, алгоритмы ЭЦП>>. Данная пара используется 
также в некоторых алгоритмах формирования общего ключа. 
В алгоритмах формирования общего ключа на основе предварительно 
распределенных секретов проводится неявная аутентификация клиента. При 
соблюдении мер защиты PSK-секрета успешное завершение 
Handshake означает, что клиент действительно знает этот 
секрет, т.~е. является подлинным.  

\subsection{Алгоритмы хэширования и электронной цифровой 
подписи}\label{CRYPTO.3.3}

Пара <<алгоритм хэширования, алгоритмы ЭЦП>> задается следующим типом:
\begin{lstlisting}
struct {
  HashAlgorithm hash;
  SignatureAlgorithm signature;
} SignatureAndHashAlgorithm;
\end{lstlisting}

Поле \lstinline{hash} определяет алгоритм хэширования и описывается 
следующим типом:  
\begin{lstlisting}
enum {none(0),..., (255)} HashAlgorithm;
\end{lstlisting}

Элемент \lstinline{none} означает, что алгоритмы ЭЦП не требуют 
хэширования данных перед выработкой или проверкой подписи.  

Поле \lstinline{signature} определяет алгоритмы ЭЦП и описывается 
следующим типом: 
\begin{lstlisting}
enum {anonymous(0),..., (255)} SignatureAlgorithm;
\end{lstlisting}

Элемент \lstinline{anonymous} означает, что подпись не вырабатывается.


