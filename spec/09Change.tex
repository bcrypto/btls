\chapter{Протокол Change Cipher Spec}\label{CHANGE}

Протокол Change~Cipher~Spec предназначен для оповещения о смене 
криптоопределений и ключей. В этом протоколе используется единственное 
сообщение \lstinline{ChangeCipherSpec}, которое защищается и сжимается с 
использованием текущего состояния соединения. Сообщение состоит из одного 
байта со значением 1: 
\begin{lstlisting}
struct {
  enum {change_cipher_spec(1), (255)} type;
} ChangeCipherSpec;
\end{lstlisting}

Сообщение \lstinline{ChangeCipherSpec} может отправлять как клиент, так и 
сервер. Данное сообщение уведомляет принимающую сторону о том, что последующие 
фрагменты данных будут защищены с использованием вновь согласованных 
криптоопределения и ключей. Сообщение \lstinline{ChangeCipherSpec} 
отправляется во время установки связи после того, как согласованы 
параметры защиты, но перед отправкой сообщения \lstinline{Finished}. 

При получении данного сообщения ожидаемое состояние чтения ДОЛЖНО быть 
переведено в активное состояние чтения. Сразу после отправки этого 
сообщения ожидаемое состояние записи ДОЛЖНО быть переведено в активное 
состояние записи (cм.~\ref{RECORD.2}). 

Если во время передачи данных по соединению инициируется переустановка
связи, то взаимодействующие стороны могут продолжить обмен данными, 
используя старое криптоопределение. Однако сразу после отправки 
сообщения~\lstinline{ChangeCipherSpec} отправитель ДОЛЖЕН использовать новое 
криптоопределение. Так как при получении сообщения \lstinline{ChangeCipherSpec} 
принимающей стороне требуется время для вычисления новых ключей, то МОЖЕТ 
существовать определенный временной интервал, в течение которого 
получатель должен буферизировать данные. На практике, как правило, этот 
интервал довольно короткий.


